\documentclass[12pt,a4paper]{article}
% PACKAGES
\usepackage[margin=1in]{geometry}
\usepackage{amsmath, amssymb, amsthm}
\usepackage{authblk}
\usepackage[utf8]{inputenc}
\usepackage{fontenc}
\usepackage{hyperref}
\hypersetup{
colorlinks=true,
linkcolor=blue,
filecolor=magenta,
urlcolor=cyan,
pdftitle={The Farey Sequence as a Realization of the Arithmetic of Order},
pdfpagemode=FullScreen,
}
% MATH OPERATORS AND ENVIRONMENTS
\DeclareMathOperator{\im}{Im}
\DeclareMathOperator{\re}{Re}
\DeclareMathOperator{\SL}{SL}
\DeclareMathOperator{\PSL}{PSL}
\DeclareMathOperator{\GL}{GL}
\DeclareMathOperator{\gcd_func}{gcd}
\newtheorem{theorem}{Theorem}[section]
\newtheorem{proposition}[theorem]{Proposition}
\newtheorem{lemma}[theorem]{Lemma}
\newtheorem{corollary}[theorem]{Corollary}
\theoremstyle{definition}
\newtheorem{definition}[theorem]{Definition}
\newtheorem{example}[theorem]{Example}
\theoremstyle{remark}
\newtheorem{remark}[theorem]{Remark}
% TITLE AND AUTHOR
\title{Physics as Quantized Measurement: The Farey Sequence as a Realization of the Arithmetic of Order}
\author{Faysal El Khettabi}
\affil{Ensemble AIs\ \texttt{faysal.el.khettabi@gmail.com}}
\date{July 2025}
\begin{document}
\maketitle
\begin{center}
\vspace{1em}
\emph{Dedicated to the memory of John Horton Conway (1937–2020),} \
\emph{whose playful genius made visible the hidden symmetries of numbers, games, and the infinite.}
\vspace{1em}
\end{center}
\tableofcontents
\begin{abstract}
This article formalizes the Farey sequence hierarchy (Fn​) and the universal Stern-Brocot tree (Tn​) as not merely analogies but explicit constructive realizations of the ``Arithmetic of Order'' (AoO) framework. It synthesizes the generative principles of constraint-guided differentiation, the progression 1→n→n+1, and the nested powerset structure, demonstrating that the Farey system embodies the finitistic, emergent continuum perspective central to the AoO thesis. We propose that the Farey numbers, generated by a finite, deterministic process, provide a natural model for physical quantities and measurements, replacing the need for an a priori infinite continuum. This synthesis reinforces a unified finitistic approach to foundational mathematics, hypercomplex number theory, projective geometries, and the algorithmic design of intelligent systems.
\end{abstract}
\section{Introduction: A Finitistic Recasting of the Continuum}
The core argument of this report is that the Farey system provides a complete and rigorous mathematical prototype for the "Arithmetic of Order" framework \cite{ElKhettabi2025AoO}. This is not a relationship of analogy, but of instantiation, where the abstract principles of the latter find their direct, operational expression in the former.
The foundations of modern mathematics and physics have been shaped by a foundational crisis that emerged in the late 19th and early 20th centuries \cite{WikipediaFoundations}. A central feature of the resolution to this crisis was the widespread adoption of infinitary concepts, most notably the continuum of real and complex numbers. The imaginary unit i=−1​∈C, for instance, is now central to the formulation of quantum theory \cite{ElKhettabi2024HCN, ElKhettabi2025AoO}. Yet, this reliance introduces a profound conceptual paradox: how can a finite physical system—a quantum register, a molecule, or any object composed of a finite number of components—fundamentally require an infinite mathematical construct for its description? \cite{ElKhettabi2025AoO}.
The "Arithmetic of Order" (AoO) framework directly confronts this paradox by proposing a mathematics built from the finite and observable, where complexity and structure emerge constructively \cite{ElKhettabi2025AoO}. This approach aligns with the philosophical traditions of finitism and constructivism \cite{SEP_Finitism, SEP_Constructivism}. Finitism, in its various forms, questions or rejects the existence of actual infinite objects, such as the set of all natural numbers, proposing that mathematics should be grounded in objects that are, at least in principle, finite \cite{WikipediaFarey, SEP_Finitism}. Constructivism insists that mathematical existence is tied to algorithmic constructibility; to prove an object exists, one must provide a method of finding (“constructing”) such an object \cite{SEP_Constructivism, Zukin2016}. The AoO framework synthesizes these views by critiquing the traditional reliance on a priori infinities and proposing a new foundation where the continuum itself is not a given axiom but an emergent property \cite{ElKhettabi2025AoO}.
This report demonstrates that the Farey sequence system offers a perfect, non-trivial model for this finitistic philosophy. The Farey sequence, generated by a simple, finite, and deterministic algorithm, provides a concrete pathway to the rational numbers and, by extension, to the real continuum. Its generation does not presuppose the existence of the continuum but rather builds it step-by-step, deriving it as an asymptotic limit of a sequence of finite structures \cite{WikipediaFarey, Zukin2016}. This aligns perfectly with the AoO's reinterpretation of the continuum as a limit of a nested hierarchy of finite sets \cite{ElKhettabi2025AoO}.
The structure of this analysis will proceed as follows. First, the abstract principles of the AoO framework will be detailed. Second, a rigorous exposition of the Farey system will be provided. The central sections will then establish the formal mapping between the generative processes and hierarchical structures of these two domains. This is followed by a deep analysis of their corresponding algebraic structures, a discussion of the hierarchical perception of the continuum, and finally, an exploration of the implications of this synthesis.
\section{The "Arithmetic of Order": A Framework of Emergent Complexity}
The "Arithmetic of Order" (AoO) is a foundational framework that seeks to re-establish mathematics on finite, constructive principles. It posits that the intricate structures of mathematics are not arbitrary inventions but are natural consequences of the most fundamental process of ordered progression \cite{ElKhettabi2025AoO}.
\subsection{The Generative Progression 1→n→n+1}
The central axiom of the AoO framework is that all of mathematics can be understood as a "natural revelation of the intrinsic structure embedded in the progression 1→n→n+1" \cite{ElKhettabi2025AoO}. This progression is not merely a representation of counting; it embodies the fundamental process of incrementally adding a new degree of freedom to a system \cite{ElKhettabi2025AoO, ElKhettabi2024HCN}. In physical terms, each step n→n+1 can be interpreted as the introduction of a new quantized degree of freedom — an additional state, bit, or mode that refines the system’s resolution of measurable quantities. For example, each new fraction in the Farey hierarchy refines an interval with finite rational steps, mirroring how physical measurement adds finite resolution at each scale. This reflects the same principle that underlies Planck’s constant in quantum mechanics: physical observables do not vary continuously in theory but are constrained by finite quanta of action and resolution. In the context of physics, a system's state is defined by a finite number of such degrees of freedom, and understanding how the system's properties change as this number increases is paramount \cite{ElKhettabi2024HCN}.
The framework emphasizes that it is the ordered nature of the underlying set Ωn​={1,2,…,n} that is crucial \cite{ElKhettabi2025AoO}. This ordering ensures that the hierarchy of structures built upon it is well-defined, recursive, and nested. Each step from n to n+1 represents a deterministic expansion of the system's potential, opening up new combinatorial possibilities and enabling the emergence of higher-order symmetries and more complex structures \cite{ElKhettabi2025AoO}.
\subsection{The Powerset Hierarchy P(Ωn​) as the Canonical State Space}
Within the AoO, the canonical state space of a system with n degrees of freedom is identified with the powerset of Ωn​, denoted P(Ωn​) \cite{ElKhettabi2025AoO}. The powerset is the set of all subsets of Ωn​, including the empty set and Ωn​ itself \cite{WikipediaPowerset}. Each element of P(Ωn​)—that is, each subset S⊆Ωn​—represents a distinct configuration of the system. This configuration can be encoded by a characteristic function, a binary vector indicating which degrees of freedom are "active" or "present" in that state \cite{ElKhettabi2025AoO, WikipediaPowerset}.
The generative progression 1→n→n+1 manifests directly as a nested hierarchy of these powerset state spaces: P(Ωn​)⊂P(Ωn+1​) \cite{ElKhettabi2024HCN}. The construction of the next level of the hierarchy is both recursive and fully deterministic. Given P(Ωn​), the powerset P(Ωn+1​) is formed by taking all existing subsets in P(Ωn​) and adding to them a new collection of subsets, each formed by taking an element of P(Ωn​) and adjoining the new element {n+1} \cite{WikipediaPowerset}. Formally, this is expressed as:

The cardinality of the powerset, ∣P(Ωn​)∣=2n, grows exponentially, highlighting the rapid increase in structural complexity that arises from the addition of each new degree of freedom \cite{ElKhettabi2024HCN}.
\subsection{The Continuum as an Asymptotic Limit}
A cornerstone of the AoO framework is its explicit rejection of the continuum as an a priori entity \cite{ElKhettabi2025AoO}. Instead, the real continuum is reinterpreted as an asymptotic limit of the nested powerset hierarchy. As the number of degrees of freedom n tends towards infinity, the combinatorial and topological properties of the finite space P(Ωn​) approach those traditionally ascribed to continuous systems \cite{ElKhettabi2025AoO}.
This perspective aligns with finitist philosophies that are skeptical of actual, completed infinities but are comfortable with the concept of potential infinity or unbounded processes \cite{WikipediaFarey, SEP_Finitism}. The framework provides a concrete model for how this limit is approached. The cardinality of the powerset, 2n, naturally approaches the cardinality of the continuum, 2ℵ0​, as n approaches the first infinite cardinal, ℵ0​. The Cantor space, which is homeomorphic to the set of infinite binary sequences {0,1}N, serves as a topological bridge, as it is in one-to-one correspondence with P(N) and has the cardinality of the continuum \cite{WikipediaPowerset}.
\subsection{Constraint-Guided Differentiation and Emergent Structures}
A key generative mechanism within the AoO framework is termed "constraint-guided differentiation" \cite{ElKhettabi2025AoO}. This principle posits that by applying specific combinatorial rules or constraints to the powerset hierarchy, certain "optimal" mathematical structures can emerge deterministically. Examples cited within the framework's literature include the emergence of exceptional structures like the Golay code G24​ and the Leech lattice Λ24​ from P(Ω24​) when specific constraints are applied \cite{ElKhettabi2025AoO}.
\section{The Farey System: A Deterministic Path to the Rationals}
The Farey sequence and its related structures, such as the Stern-Brocot tree, form a cornerstone of elementary number theory, providing a constructive method for generating the set of rational numbers.
\subsection{The Farey Sequence Fn​: Definition and Hierarchical Construction}
The Farey sequence of order n, denoted Fn​, is formally defined as the sequence of irreducible fractions a/b in the closed interval (0,1) such that the denominator b is a positive integer satisfying 1≤b≤n, arranged in ascending order of magnitude \cite{WikipediaFarey, Zukin2016}. Each sequence conventionally starts with 0/1 and ends with 1/1 \cite{WikipediaFarey}.
The first few Farey sequences are \cite{Zukin2016, RanickiFareyProject}:
\begin{itemize}
\item F1​:{10​,11​}
\item F2​:{10​,21​,11​}
\item F3​:{10​,31​,21​,32​,11​}
\item F4​:{10​,41​,31​,21​,32​,43​,11​}
\end{itemize}
A fundamental property of this construction is its hierarchical nature: Fn​⊂Fn+1​ for all n≥1 \cite{WikipediaFarey, Zukin2016}. The number of new fractions added at step n+1 is precisely given by Euler's totient function, ϕ(n+1) \cite{WikipediaFarey, Zukin2016}.
\subsection{Mediant Refinement: The Algorithmic Engine}
The generation of Farey sequences is driven by the mediant operation \cite{RanickiFareyProject, NumberanalyticsFarey}. The mediant of two fractions a/b and c/d is defined as (a+c)/(b+d). If a/b<c/d, their mediant always lies strictly between them \cite{KnottFarey, RanickiFareyProject, NumberanalyticsFarey}.
A fundamental theorem states that the sequence Fn+1​ can be constructed from Fn​ by identifying all adjacent pairs of fractions a/b and c/d in Fn​ and inserting their mediant (a+c)/(b+d) between them if and only if the denominator of the mediant satisfies b+d=n+1 \cite{DUMMIT, Zukin2016, JNSFarey}. This provides a fully algorithmic and constructive method for generating the entire hierarchy \cite{Zukin2016, JNSFarey}.
\subsection{The Stern-Brocot Tree: The Universal Genealogy of Rationals}
The mediant operation, when applied recursively without the denominator constraint, generates the Stern-Brocot tree, an infinite binary tree that contains every positive rational number exactly once \cite{WikipediaSternBrocot, CPAlgorithmsSternBrocot}. It is constructed by starting with the "ancestors" 0/1 and 1/0 (representing 0 and infinity) and iteratively inserting the mediant between adjacent fractions \cite{WikipediaSternBrocot, CutTheKnotSternBrocot}. The Farey sequence Fn​ can be recovered by an in-order traversal of the tree, pruning any branch where a denominator exceeds n \cite{WikipediaSternBrocot, CPAlgorithmsSternBrocot}.
\subsection{Foundational Properties and their Significance}
\begin{itemize}
\item \textbf{The Unimodular Relation:} If two fractions a/b and c/d are adjacent in any Farey sequence Fn​, they satisfy the unimodular relation: bc−ad=1 \cite{DUMMIT, Zukin2016}. This invariant guarantees that any mediant formed from two neighbors is itself irreducible \cite{Zukin2016, JNSFarey}.
\item \textbf{Optimality in Diophantine Approximation:} Farey sequences contain the set of "best rational approximations" of the first kind to any real number for a given denominator bound n \cite{Zukin2016, JNSFarey}.
\item \textbf{The Emergent Continuum:} The union of all Farey sequences, ⋃n=1∞​Fn​, constitutes the set of all rational numbers in
$$\cite{CutTheKnotFarey}. The real continuum$$
is the topological closure of this constructively generated set \cite{Zukin2016}.
\end{itemize}
\section{The Isomorphism of Process: Unifying the Framework and its Realization}
The parallels between the AoO framework and the Farey system are not merely superficial. The core generative processes of both systems are structurally identical.
\subsection{Local Refinement as Constrained Expansion}
The fundamental mechanism of evolution in both systems is a process of local refinement governed by a global constraint.
\begin{itemize}
\item In the Farey system, an interval (a/b,c/d) in Fn​ is refined by the mediant (a+c)/(b+d) only when the denominator constraint b+d=n+1 is met \cite{Zukin2016, JNSFarey}.
\item In the powerset hierarchy, the transition from P(Ωn​) to P(Ωn+1​) introduces new subsets defined by the inclusion of the single new element {n+1} \cite{WikipediaPowerset, ElKhettabi2024HCN}.
\end{itemize}
The mediant insertion rule is a perfect instantiation of the AoO's abstract principle of "constraint-guided differentiation" \cite{ElKhettabi2025AoO}. The parameter n acts as a universal filter, determining which new elements are actualized at each stage.
\subsection{The Arrow of Complexity in Nested Hierarchies}
There is a formal mapping between the set inclusion Fn​⊂Fn+1​ and the powerset inclusion P(Ωn​)⊂P(Ωn+1​). Despite different growth rates (polynomial for ∣Fn​∣∼3n2/π2 vs. exponential for ∣P(Ωn​)∣=2n), both systems exhibit an irreversible "arrow of complexity" driven by the same underlying progression, 1→n→n+1 \cite{ElKhettabi2025AoO, WikipediaFarey, Zukin2016}.
\section{A Unification of Algebraic Structures}
In algebraic thinking, the “rules” are not just about solving for unknowns, but about determining the space of all possible solutions and how they interrelate. The correspondence between the Farey and powerset systems extends to the very algebraic structures they embody, revealing a profound duality.
\begin{table}[h!]
\centering
\caption{Structural Parallels between Farey and Powerset Systems}
\label{tab:duality}
\begin{tabular}{|p{2.5cm}|p{4cm}|p{4cm}|p{3.5cm}|}
\hline
\textbf{Aspect} & \textbf{Farey/Stern-Brocot System} & \textbf{Powerset Hierarchy} & \textbf{Framework Link} \
\hline
\textbf{Elements} & Irreducible fractions p/q & Subsets S⊆Ωn​ & Configurations of n degrees of freedom \cite{ElKhettabi2025AoO, Zukin2016} \
\hline
\textbf{Ordering} & Standard numerical order < & Subset inclusion ⊂ & Hierarchical nesting \cite{ElKhettabi2025AoO, Zukin2016} \
\hline
\textbf{Refinement Rule} & Mediant b+da+c​ if b+d≤n+1 \cite{Zukin2016, JNSFarey} & Add subsets containing n+1 \cite{WikipediaPowerset} & Constraint-guided differentiation \cite{ElKhettabi2025AoO} \
\hline
\textbf{Local Algebra} & Neighbors a/b,c/d form a matrix in $\SL(2,\mathbb{Z})$ \cite{Zukin2016, DUMMIT} & --- & Structure of local interactions \
\hline
\textbf{Global Algebra} & --- & (P(Ωn​),Δ) is isomorphic to F2n​ \cite{WikipediaPowerset} & Algebra of system states \cite{ElKhettabi2024HCN} \
\hline
\textbf{Geometric View} & Ford Circles, Farey Tessellation of H2 \cite{WikipediaFarey, Zukin2016} & Projective planes/spaces over F2​ \cite{ElKhettabi2025AoO} & Emergent geometry \cite{ElKhettabi2025AoO} \
\hline
\textbf{Asymptotic Limit} & The real continuum (0,1) \cite{Zukin2016} & The Cantor space 2N, cardinality of continuum \cite{WikipediaPowerset} & The emergent continuum \cite{ElKhettabi2025AoO} \
\hline
\end{tabular}
\end{table}
\subsection{The Modular Group $\SL(2,\mathbb{Z})$: The Invariant Algebra of Farey Adjacency}
The unimodular relation bc−ad=1 is the defining characteristic of the special linear group $\SL(2,\mathbb{Z})$ \cite{ConradSL2Z, WikipediaSL2Z}. For any pair of adjacent Farey neighbors a/b and c/d, the matrix (ab​cd​) is an element of $\SL(2,\mathbb{Z})$ \cite{Zukin2016, DUMMIT}. This group acts as the group of orientation-preserving automorphisms of the Farey graph, which tessellates the hyperbolic plane.\footnote{Conway's Farey tessellation and the modular group's action on it are beautifully detailed in his \emph{Topograph} \cite{ConwayTopograph1985}.} This connection is further deepened by the study of frieze patterns, where the work of Conway and Coxeter revealed combinatorial interpretations of the modular group related to polygon dissections and the Farey tessellation \cite{ConwayCoxeter1973}. $\SL(2,\mathbb{Z})$ is thus the infinite, non-abelian group that governs the local, dynamic structure of the Farey sequence \cite{ConradSL2Z}.
\subsection{The Vector Space F2n​: The Canonical Algebra of System Configurations}
The powerset P(Ωn​), when equipped with the operation of symmetric difference (Δ), forms a finite abelian group \cite{WikipediaPowerset, BooleanRing}. This group is isomorphic to the n-dimensional vector space over the finite field of two elements, F2n​ \cite{F2nVectorSpace, BooleanRing}. This algebraic structure is presented in the AoO framework as the canonical algebra for describing the global, static set of all possible configurations of a system with n binary degrees of freedom \cite{ElKhettabi2025AoO, ElKhettabi2024HCN}.
\subsection{The Duality of Local (Non-Abelian) and Global (Abelian) Algebra}
The synthesis of these systems reveals a profound duality. The AoO framework, by asserting that the Farey system is a concrete instantiation of its principles, implicitly predicts this juxtaposition of the infinite, non-abelian group $\SL(2,\mathbb{Z})$ with the sequence of finite, abelian groups F2n​. This key observation can be summarized as follows: The Farey system’s local, non-abelian symmetry ($\SL(2,\mathbb{Z})$) governs the dynamic refinement of the space \cite{Zukin2016, ConradSL2Z}, while the powerset’s global, abelian symmetry (F2n​) governs the static configuration of all possible states \cite{WikipediaPowerset, BooleanRing}. This duality between dynamic refinement and static configuration is a central feature of the AoO framework's explanatory power.
\section{Hierarchical Perception of the Continuum through Degrees of Freedom}
The relationship between the sets of mediants generated at different stages of the Farey hierarchy captures how systems with different degrees of freedom "perceive" the continuum. This section formalizes this concept, demonstrating that each degree of freedom unlocks a distinct, non-overlapping "vision" of the emergent continuum.
\subsection{Defining the "Vision" of a Degree of Freedom: The Sets Wn​ and Wm​}
For a system with a given degree of freedom, represented by the integer n, we can define its specific interval of focus and the set of rational numbers it generates to refine that interval.
\begin{itemize}
\item \textbf{Interval of Focus for degree n:} The interval In​=(n+11​,n1​) is defined by two fractions that become Farey neighbors in the sequence Fn+1​ \cite{Zukin2016}.
\item \textbf{Mediant Refinement Set Wn​:} We define Wn​ as the set of all rational numbers that recursively refine the interval In​. This set is generated by the iterative application of the mediant operation, a process identical to the construction of a local Stern-Brocot tree within that interval \cite{Zukin2016, NumberanalyticsFarey}.
\begin{itemize}
\item \textbf{Properties of Wn​:}
\item Wn​ is \textbf{dense} in (n+11​,n1​), as the mediant operation eventually fills every gap between rational numbers \cite{DUMMIT}.
\item Wn​ is \textbf{infinite} and \textbf{ordered} according to the structure of the Stern-Brocot process \cite{WikipediaSternBrocot, CPAlgorithmsSternBrocot}.
\end{itemize}
\end{itemize}
For a system with a higher degree of freedom m>n:
\begin{itemize}
\item \textbf{Interval of Focus for degree m:} Im​=(m+11​,m1​).
\item \textbf{Mediant Refinement Set Wm​:} Wm​ is similarly defined as the set of all mediants that recursively refine the interval Im​.
\end{itemize}
\begin{theorem}
Let In​=(n+11​,n1​) and define Wn​ as the set of all rationals generated by repeated mediant refinement within In​. Then for any m>n+1, the sets Wn​ and Wm​ are disjoint:
[
W_n \cap W_m = \emptyset.
]
Moreover, the rational continuum in (0,1) satisfies
[
\bigcup_{n=1}^{\infty} W_n = \mathbb{Q} \cap (0,1),
\quad \text{and} \quad
\overline{\bigcup_{n=1}^{\infty} W_n} = (0,1).
]
\end{theorem}
\begin{remark}
The non-overlapping property of the Wn​ sets formalizes the notion that a physical system with n quantized degrees of freedom cannot access or resolve measurement bands defined by higher m>n+1. This mirrors how discrete energy levels, finite sampling, or Planck-scale limits constrain physical measurement in quantum theory.
\end{remark}
\begin{corollary}
Within the Arithmetic of Order (AoO) framework, the concept of ``zero'' is not assumed as a primitive numerical constant but emerges constructively as a relational boundary condition. The ordered powerset hierarchy begins with the empty set {} as a symbolic holder, not a numerical zero. Its powerset P({})={{}} generates the first distinguishable container. The introduction of the first degree of freedom {1} yields P({1})={{},{1}}, whose cardinality 21=2=1+1 demonstrates that counting is rooted in nesting and ordering, not in a naked zero.
In parallel, the Farey sequence never invokes an a priori continuum zero but uses 0/1 as a finite measurement anchor. Its mediant operation,
[
\frac{a}{b} \oplus \frac{c}{d} = \frac{a+c}{b+d},
]
ensures that each refinement step preserves the integer basis for measurement, approaching the continuum limit only asymptotically via the nested bands Wn​.
Thus, the continuum’s nude zero'' appears solely as the relational closure of finitely constructed measurement steps: zero'' is never primitive but always contextualized by the ordered degrees of freedom that define its limit.
\end{corollary}
\subsection{Implications for the Arithmetic of Order Framework}
This model of hierarchical, disjoint perception has significant implications for the AoO framework.
\begin{enumerate}
\item \textbf{Degree-of-Freedom-Dependent Reality:} The continuum is not perceived uniformly but as a hierarchy of refinements, where each "vision" is tied to the observer’s available degrees of freedom (n or m) \cite{ElKhettabi2025AoO}. A physical system with n variables cannot "see" the finer structure revealed by m>n variables.
\item \textbf{Finitistic Continuum:} The "complete" continuum is the limit of all Wk​, but no system with a finite number of degrees of freedom can fully capture it. This aligns with the core philosophy of finitism \cite{ElKhettabi2025AoO, SEP_Finitism}.
\item \textbf{Algebraic Consistency:} The disjointness of Wn​ and Wm​ is a concrete example of the AoO’s principle of constraint-guided differentiation \cite{ElKhettabi2025AoO}. Each degree of freedom k introduces structure only in its designated interval (k+11​,k1​).
\end{enumerate}
\section{Conclusion and Outlook}
The rigorous analysis presented here confirms that the Farey sequence hierarchy Fn​ and the infinite Stern-Brocot tree provide a direct and operational instantiation of the core principles articulated in the \emph{Arithmetic of Order} framework. By demonstrating a formal isomorphism between the Farey system’s mediant-based local refinement and the AoO’s principle of \emph{constraint-guided differentiation}, this work extends the finitistic, constructivist agenda championed in El Khettabi’s foundational reports on powerset combinatorics, hypercomplex numbers, and emergent geometries \cite{ElKhettabi2025AoO, ElKhettabi2024HCN, ElKhettabi2024PLOS}.
Crucially, the Farey system shows that the real continuum --- historically assumed as an \emph{a priori} infinite structure --- can instead be understood as the asymptotic closure of a fully discrete, algorithmically generable process \cite{SEP_Finitism, Zukin2016}. Every rational in the unit interval is positioned within a nested, well-ordered hierarchy whose generation is entirely finitistic and transparent \cite{WikipediaFarey, Zukin2016}. The Farey mediant mechanism mirrors the XOR-bitwise operation on powersets: both systems employ a local, deterministic rule to refine an initial configuration space under a global constraint parameter, here the denominator bound n \cite{WikipediaPowerset, JNSFarey}.
The duality between local non-abelian modular symmetries ($\SL(2,\mathbb{Z})$) and the global abelian powerset algebra (F2n​) reinforces the AoO’s insight that the richness of mathematical and physical structure can emerge from the simple arithmetic of ordered degrees of freedom \cite{ElKhettabi2025AoO, ConradSL2Z, BooleanRing}. Just as the Golay code G24​ and the Leech lattice Λ24​ arise from sieving the powerset P(Ω24​), a process whose outputs were famously used by Conway and others to construct the Leech lattice \cite{ConwaySloane}, so too does the Farey hierarchy filter the Stern-Brocot tree into optimally ordered sets of best rational approximations \cite{DUMMIT}.
By placing the Farey system within this finitistic combinatorial paradigm, this article not only strengthens the AoO’s theoretical claims but demonstrates its flexibility and universality across number theory, coding theory, and the modeling of continuum phenomena within strictly finite means. In this sense, the Farey hierarchy provides an explicit model of physics as quantized measurement: each mediant, each denominator constraint, each combinatorial power represents a discrete quantum of possible states. The real continuum, as the limit of these finite steps, is not a physical prerequisite but an emergent idealization of finitely resolved measurements.
\textbf{Future Directions.} This synthesis invites further exploration along several lines. First, the deep interplay between Farey adjacency, modular tessellations, and hyperbolic geometry (Ford circles, Farey tessellation of H2) deserves to be mapped explicitly onto the projective geometries naturally emerging from the ordered powerset hierarchy \cite{ElKhettabi2025AoO, WikipediaFarey, FareyGraphSymmetry}. Second, the constructive mediant process suggests algorithmic avenues for designing finite, resource-bounded AI systems capable of performing Diophantine approximations without recourse to continuum assumptions \cite{ElKhettabi2025AoO, WikipediaSternBrocot}. Finally, the formal analogy between Farey trees and the nested powerset combinatorics suggests potential generalizations to higher-dimensional hypercomplex structures and their associated finite geometries.
In unifying the Farey sequence and the Arithmetic of Order framework, we highlight a powerful theme: the apparent paradox of continuum structures in finite physical systems dissolves when seen as the limit of simple, local, finitistic rules. Mathematics, under this lens, is not an edifice built on the infinite, but a revelation of emergent order rooted in the finite --- one degree of freedom at a time. This report stands as one finite tessellation of the infinite --- dedicated to John Horton Conway, who taught us that mathematics is best revealed not by assuming the infinite, but by playing joyfully at its edge.
\vspace{1em}
\noindent
\textbf{Faysal El Khettabi} \
\emph{Ensemble AIs} \
\texttt{faysal.el.khettabi@gmail.com} \
July 2025
\appendix
\section{Appendix}
\subsection{Proof of the Unimodular Relation via Pick's Theorem}
A key property of Farey sequences is that if a/b and c/d are adjacent terms (neighbors), then bc−ad=1. A geometric proof utilizes Pick's Theorem, which relates the area of a simple polygon whose vertices are points on the integer lattice to the number of integer points on its boundary and in its interior \cite{DUMMIT}. The area A is given by A=I+2B​−1, where I is the number of interior lattice points and B is the number of boundary lattice points.
Consider the triangle △ with vertices at the origin O(0,0), P1​(b,a), and P2​(d,c). The area of this triangle can be calculated using the determinant formula, which gives A=21​∣bc−ad∣. Since a/b<c/d, we have bc>ad, so the area is A=21​(bc−ad).
Now, we apply Pick's Theorem to this triangle:
\begin{itemize}
\item \textbf{Boundary Points (B):} The vertices O, P1​, and P2​ are lattice points. Since the fractions a/b and c/d are irreducible, gcdf​unc(a,b)=1 and gcdf​unc(c,d)=1. This implies there are no other lattice points on the segments OP1​ and OP2​. If there were a lattice point on the segment P1​P2​, it would represent a rational fraction that should lie between a/b and c/d in the Farey sequence but with a smaller denominator, which is impossible for neighbors. Thus, the only lattice points on the boundary are the three vertices, so B=3 \cite{DUMMIT}.
\item \textbf{Interior Points (I):} Suppose there were a lattice point (x,y) in the interior of △. The fraction y/x would have a value strictly between a/b and c/d. Since b≤n and d≤n, it would follow that x<n. This would mean that y/x is a fraction with a denominator smaller than n that lies between a/b and c/d, contradicting the assumption that they are consecutive terms in Fn​. Therefore, there can be no lattice points in the interior of the triangle, and I=0 \cite{DUMMIT}.
\end{itemize}
Applying Pick's Theorem with I=0 and B=3, the area of the triangle is A=0+23​−1=21​. Equating this with our determinant formula, we get 21​(bc−ad)=21​, which directly implies bc−ad=1. This provides a purely combinatorial confirmation that the mediant always yields an irreducible fraction when the unimodular condition is met.
\subsection{The Isomorphism (P(Ωn​),Δ)≅F2n​}
The powerset P(Ωn​) of a set Ωn​={1,2,…,n}, when equipped with the operation of symmetric difference (Δ), forms a finite abelian group \cite{WikipediaPowerset, SymmetricDifferenceGroup}.
\begin{itemize}
\item \textbf{Closure:} For any two subsets A,B⊆Ωn​, their symmetric difference AΔB=(A∪B)∖(A∩B) is also a subset of Ωn​.
\item \textbf{Associativity:} The operation is associative: (AΔB)ΔC=AΔ(BΔC).
\item \textbf{Identity Element:} The empty set ∅ serves as the identity element, as AΔ∅=A.
\item \textbf{Inverse Element:} Every element is its own inverse, as AΔA=∅.
\end{itemize}
This group is isomorphic to the n-dimensional vector space over the field of two elements, F2​={0,1}, denoted F2n​ \cite{F2nVectorSpace, BooleanRing}. The isomorphism is established by mapping each subset S⊆Ωn​ to its characteristic function (or binary vector) of length n. The symmetric difference of two subsets corresponds precisely to the component-wise addition (XOR operation) of their corresponding vectors in F2n​. This mapping makes the group operation identical to vector addition modulo 2, clarifying the isomorphism’s computational meaning. The set of singleton subsets {{1},{2},…,{n}} forms a basis for this vector space \cite{F2nVectorSpace}.
\begin{thebibliography}{99}
\bibitem{WikipediaFoundations}
Wikipedia contributors. (2024). \emph{Foundations of mathematics}. Wikipedia, The Free Encyclopedia.
\bibitem{ElKhettabi2025AoO}
El Khettabi, F. (2025). \emph{The Arithmetic of Order: A Finitistic Foundation for Mathematics, Emergent Structures, and Intelligent Systems}. viXra.
\bibitem{ElKhettabi2024HCN}
El Khettabi, F. (2024). \emph{A Comprehensive Modern Mathematical Foundation for Hypercomplex Numbers with Recollection of Sir William Rowan Hamilton, John T. Graves, and Arthur Cayley}.
\bibitem{SEP_Constructivism}
Iemhoff, R. (2023). \emph{Constructive Mathematics}. The Stanford Encyclopedia of Philosophy (Fall 2023 Edition), Edward N. Zalta & Uri Nodelman (eds.).
\bibitem{WikipediaFarey}
Wikipedia contributors. (2024). \emph{Farey sequence}. Wikipedia, The Free Encyclopedia.
\bibitem{SEP_Finitism}
Ye, F. (2021). \emph{Finitism in Geometry}. The Stanford Encyclopedia of Philosophy (Winter 2021 Edition), Edward N. Zalta (ed.).
\bibitem{Zukin2016}
Zukin, M. (2016). \emph{The Farey Sequence}. Whitman College.
\bibitem{DUMMIT}
Dummit, D. S., & Foote, R. M. (2004). \emph{Abstract Algebra}. John Wiley & Sons.
\bibitem{ElKhettabi2024PLOS}
El Khettabi, F. (2024). \emph{On the hypercomplex numbers and normed division algebra of all dimensions: A unified multiplication}. PLOS ONE, 19(6), e0312502.
\bibitem{WikipediaPowerset}
Wikipedia contributors. (2024). \emph{Power set}. Wikipedia, The Free Encyclopedia.
\bibitem{ElKhettabi2025AoO_vixra}
El Khettabi, F. (2025). The Arithmetic of Order: A Finitistic Foundation for Mathematics, Emergent Structures, and Intelligent Systems. \emph{viXra:2505.0064}.
\bibitem{JNSFarey}
Tamang, B. B., et al. (2022). Some characteristics of the Farey sequences with Ford circles. \emph{Nepal Journal of Mathematical Sciences}, 4(1), 69-76.
\bibitem{RanickiFareyProject}
Ranicki, A. (n.d.). \emph{The Farey sequence and its applications}.
\bibitem{KnottFarey}
Knott, R. \emph{Farey Series and the Stern-Brocot Tree}. University of Surrey.
\bibitem{NumberanalyticsFarey}
Number Analytics. (n.d.). \emph{Farey Sequences: A Deep Dive into Additive Number Theory}.
\bibitem{WikipediaSternBrocot}
Wikipedia contributors. (2024). \emph{Stern–Brocot tree}. Wikipedia, The Free Encyclopedia.
\bibitem{CPAlgorithmsSternBrocot}
CP-Algorithms. \emph{Stern-Brocot Tree and Farey Sequences}.
\bibitem{CutTheKnotFarey}
Bogomolny, A. \emph{Farey Series}. Cut-the-Knot.
\bibitem{CutTheKnotSternBrocot}
Bogomolny, A. \emph{Stern-Brocot Tree}. Cut-the-Knot.
\bibitem{ConradSL2Z}
Conrad, K. \emph{SL(2,Z)}. University of Connecticut.
\bibitem{WikipediaSL2Z}
Wikipedia contributors. (2024). \emph{Modular group}. Wikipedia, The Free Encyclopedia.
\bibitem{FareyGraphSymmetry}
Lutsko, C. (2021). \emph{Generalized Farey sequences}. International Mathematics Research Notices.
\bibitem{BooleanRing}
Wikipedia contributors. (2024). \emph{Boolean ring}. Wikipedia, The Free Encyclopedia.
\bibitem{F2nVectorSpace}
Stack Exchange. (2018). \emph{P(X) with symmetric difference as addition as a vector space over Z2}.
\bibitem{SymmetricDifferenceGroup}
ProofWiki. \emph{Symmetric Difference on Power Set forms Abelian Group}.
\bibitem{HardyWright}
Hardy, G. H., & Wright, E. M. (1979). \emph{An Introduction to the Theory of Numbers}. Oxford University Press.
\bibitem{ConwaySloane}
Conway, J. H., & Sloane, N. J. A. (1999). \emph{Sphere Packings, Lattices, and Groups}. Springer.
\bibitem{ConwayTopograph1985}
Conway, J. H. (1985). The Topograph: An Algebraist's Picturesque View of Quadratic Forms. \emph{Mathematical Intelligencer}, 7(4), 7-20.
\bibitem{ConwayCoxeter1973}
Conway, J. H., & Coxeter, H. S. M. (1973). Triangulated Polygons and Frieze Patterns. \emph{The Mathematical Gazette}, 57(401), 87-94.
\end{thebibliography}
\end{document}















\documentclass[12pt,a4paper]{article}% PACKAGES\usepackage[margin=1in]{geometry}\usepackage{amsmath, amssymb, amsthm}\usepackage{authblk}\usepackage[utf8]{inputenc}\usepackage{fontenc}\usepackage{hyperref}\hypersetup{colorlinks=true,linkcolor=blue,filecolor=magenta,urlcolor=cyan,pdftitle={The Farey Sequence as a Realization of the Arithmetic of Order},pdfpagemode=FullScreen,}% MATH OPERATORS AND ENVIRONMENTS\DeclareMathOperator{\im}{Im}\DeclareMathOperator{\re}{Re}\DeclareMathOperator{\SL}{SL}\DeclareMathOperator{\PSL}{PSL}\DeclareMathOperator{\GL}{GL}\DeclareMathOperator{\gcd_func}{gcd}\newtheorem{theorem}{Theorem}[section]\newtheorem{proposition}[theorem]{Proposition}\newtheorem{lemma}[theorem]{Lemma}\newtheorem{corollary}[theorem]{Corollary}\theoremstyle{definition}\newtheorem{definition}[theorem]{Definition}\newtheorem{example}[theorem]{Example}\theoremstyle{remark}\newtheorem{remark}[theorem]{Remark}% TITLE AND AUTHOR\title{Physics as Quantized Measurement: The Farey Sequence as a Realization of the Arithmetic of Order}\author{Faysal El Khettabi}\affil{Ensemble AIs\ \texttt{faysal.el.khettabi@gmail.com}}\date{July 2025}\begin{document}\maketitle\begin{center}\textit{To the memory of John Horton Conway, \ whose playful spirit revealed worlds within worlds.}\end{center}\newpage\tableofcontents\begin{abstract}This article formalizes the Farey sequence hierarchy (Fn​) and the universal Stern-Brocot tree (Tn​) as not merely analogies but explicit constructive realizations of the ``Arithmetic of Order'' (AoO) framework. It synthesizes the generative principles of constraint-guided differentiation, the progression 1→n→n+1, and the nested powerset structure, demonstrating that the Farey system embodies the finitistic, emergent continuum perspective central to the AoO thesis. We propose that the Farey numbers, generated by a finite, deterministic process, provide a natural model for physical quantities and measurements, replacing the need for an a priori infinite continuum. This synthesis reinforces a unified finitistic approach to foundational mathematics, hypercomplex number theory, projective geometries, and the algorithmic design of intelligent systems.\end{abstract}\section{Introduction: A Finitistic Recasting of the Continuum}The core argument of this report is that the Farey system provides a complete and rigorous mathematical prototype for the "Arithmetic of Order" framework \cite{ElKhettabi2025AoO}. This is not a relationship of analogy, but of instantiation, where the abstract principles of the latter find their direct, operational expression in the former.The foundations of modern mathematics and physics have been shaped by a foundational crisis that emerged in the late 19th and early 20th centuries \cite{WikipediaFoundations}. A central feature of the resolution to this crisis was the widespread adoption of infinitary concepts, most notably the continuum of real and complex numbers. The imaginary unit i=−1​∈C, for instance, is now central to the formulation of quantum theory \cite{ElKhettabi2024HCN, ElKhettabi2025AoO}. Yet, this reliance introduces a profound conceptual paradox: how can a finite physical system—a quantum register, a molecule, or any object composed of a finite number of components—fundamentally require an infinite mathematical construct for its description? \cite{ElKhettabi2025AoO}.The "Arithmetic of Order" (AoO) framework directly confronts this paradox by proposing a mathematics built from the finite and observable, where complexity and structure emerge constructively \cite{ElKhettabi2025AoO}. This approach aligns with the philosophical traditions of finitism and constructivism \cite{SEP_Finitism, SEP_Constructivism}. Finitism, in its various forms, questions or rejects the existence of actual infinite objects, such as the set of all natural numbers, proposing that mathematics should be grounded in objects that are, at least in principle, finite \cite{WikipediaFarey, SEP_Finitism}. Constructivism insists that mathematical existence is tied to algorithmic constructibility; to prove an object exists, one must provide a method of finding (“constructing”) such an object \cite{SEP_Constructivism, Zukin2016}. The AoO framework synthesizes these views by critiquing the traditional reliance on a priori infinities and proposing a new foundation where the continuum itself is not a given axiom but an emergent property \cite{ElKhettabi2025AoO}.This report demonstrates that the Farey sequence system offers a perfect, non-trivial model for this finitistic philosophy. The Farey sequence, generated by a simple, finite, and deterministic algorithm, provides a concrete pathway to the rational numbers and, by extension, to the real continuum. Its generation does not presuppose the existence of the continuum but rather builds it step-by-step, deriving it as an asymptotic limit of a sequence of finite structures \cite{WikipediaFarey, Zukin2016}. This aligns perfectly with the AoO's reinterpretation of the continuum as a limit of a nested hierarchy of finite sets \cite{ElKhettabi2025AoO}.The structure of this analysis will proceed as follows. First, the abstract principles of the AoO framework will be detailed. Second, a rigorous exposition of the Farey system will be provided. The central sections will then establish the formal mapping between the generative processes and hierarchical structures of these two domains. This is followed by a deep analysis of their corresponding algebraic structures, a discussion of the hierarchical perception of the continuum, and finally, an exploration of the implications of this synthesis.\section{The "Arithmetic of Order": A Framework of Emergent Complexity}The "Arithmetic of Order" (AoO) is a foundational framework that seeks to re-establish mathematics on finite, constructive principles. It posits that the intricate structures of mathematics are not arbitrary inventions but are natural consequences of the most fundamental process of ordered progression \cite{ElKhettabi2025AoO}.\subsection{The Generative Progression 1→n→n+1}The central axiom of the AoO framework is that all of mathematics can be understood as a "natural revelation of the intrinsic structure embedded in the progression 1→n→n+1" \cite{ElKhettabi2025AoO}. This progression is not merely a representation of counting; it embodies the fundamental process of incrementally adding a new degree of freedom to a system \cite{ElKhettabi2025AoO, ElKhettabi2024HCN}. In physical terms, each step n→n+1 can be interpreted as the introduction of a new quantized degree of freedom — an additional state, bit, or mode that refines the system’s resolution of measurable quantities. For example, each new fraction in the Farey hierarchy refines an interval with finite rational steps, mirroring how physical measurement adds finite resolution at each scale. This reflects the same principle that underlies Planck’s constant in quantum mechanics: physical observables do not vary continuously in theory but are constrained by finite quanta of action and resolution. In the context of physics, a system's state is defined by a finite number of such degrees of freedom, and understanding how the system's properties change as this number increases is paramount \cite{ElKhettabi2024HCN}.The framework emphasizes that it is the ordered nature of the underlying set Ωn​={1,2,…,n} that is crucial \cite{ElKhettabi2025AoO}. This ordering ensures that the hierarchy of structures built upon it is well-defined, recursive, and nested. Each step from n to n+1 represents a deterministic expansion of the system's potential, opening up new combinatorial possibilities and enabling the emergence of higher-order symmetries and more complex structures \cite{ElKhettabi2025AoO}.\subsection{The Powerset Hierarchy P(Ωn​) as the Canonical State Space}Within the AoO, the canonical state space of a system with n degrees of freedom is identified with the powerset of Ωn​, denoted P(Ωn​) \cite{ElKhettabi2025AoO}. The powerset is the set of all subsets of Ωn​, including the empty set and Ωn​ itself \cite{WikipediaPowerset}. Each element of P(Ωn​)—that is, each subset S⊆Ωn​—represents a distinct configuration of the system. This configuration can be encoded by a characteristic function, a binary vector indicating which degrees of freedom are "active" or "present" in that state \cite{ElKhettabi2025AoO, WikipediaPowerset}.The generative progression 1→n→n+1 manifests directly as a nested hierarchy of these powerset state spaces: P(Ωn​)⊂P(Ωn+1​) \cite{ElKhettabi2024HCN}. The construction of the next level of the hierarchy is both recursive and fully deterministic. Given P(Ωn​), the powerset P(Ωn+1​) is formed by taking all existing subsets in P(Ωn​) and adding to them a new collection of subsets, each formed by taking an element of P(Ωn​) and adjoining the new element {n+1} \cite{WikipediaPowerset}. Formally, this is expressed as:The cardinality of the powerset, ∣P(Ωn​)∣=2n, grows exponentially, highlighting the rapid increase in structural complexity that arises from the addition of each new degree of freedom \cite{ElKhettabi2024HCN}.\subsection{The Continuum as an Asymptotic Limit}A cornerstone of the AoO framework is its explicit rejection of the continuum as an a priori entity \cite{ElKhettabi2025AoO}. Instead, the real continuum is reinterpreted as an asymptotic limit of the nested powerset hierarchy. As the number of degrees of freedom n tends towards infinity, the combinatorial and topological properties of the finite space P(Ωn​) approach those traditionally ascribed to continuous systems \cite{ElKhettabi2025AoO}.This perspective aligns with finitist philosophies that are skeptical of actual, completed infinities but are comfortable with the concept of potential infinity or unbounded processes \cite{WikipediaFarey, SEP_Finitism}. The framework provides a concrete model for how this limit is approached. The cardinality of the powerset, 2n, naturally approaches the cardinality of the continuum, 2ℵ0​, as n approaches the first infinite cardinal, ℵ0​. The Cantor space, which is homeomorphic to the set of infinite binary sequences {0,1}N, serves as a topological bridge, as it is in one-to-one correspondence with P(N) and has the cardinality of the continuum \cite{WikipediaPowerset}.\subsection{Constraint-Guided Differentiation and Emergent Structures}A key generative mechanism within the AoO framework is termed "constraint-guided differentiation" \cite{ElKhettabi2025AoO}. This principle posits that by applying specific combinatorial rules or constraints to the powerset hierarchy, certain "optimal" mathematical structures can emerge deterministically. Examples cited within the framework's literature include the emergence of exceptional structures like the Golay code G24​ and the Leech lattice Λ24​ from P(Ω24​) when specific constraints are applied \cite{ElKhettabi2025AoO}.\section{The Farey System: A Deterministic Path to the Rationals}The Farey sequence and its related structures, such as the Stern-Brocot tree, form a cornerstone of elementary number theory, providing a constructive method for generating the set of rational numbers.\subsection{The Farey Sequence Fn​: Definition and Hierarchical Construction}The Farey sequence of order n, denoted Fn​, is formally defined as the sequence of irreducible fractions a/b in the closed interval (0,1) such that the denominator b is a positive integer satisfying 1≤b≤n, arranged in ascending order of magnitude \cite{WikipediaFarey, Zukin2016}. Each sequence conventionally starts with 0/1 and ends with 1/1 \cite{WikipediaFarey}.The first few Farey sequences are \cite{Zukin2016, RanickiFareyProject}:\begin{itemize}\item F1​:{10​,11​}\item F2​:{10​,21​,11​}\item F3​:{10​,31​,21​,32​,11​}\item F4​:{10​,41​,31​,21​,32​,43​,11​}\end{itemize}A fundamental property of this construction is its hierarchical nature: Fn​⊂Fn+1​ for all n≥1 \cite{WikipediaFarey, Zukin2016}. The number of new fractions added at step n+1 is precisely given by Euler's totient function, ϕ(n+1) \cite{WikipediaFarey, Zukin2016}.\subsection{Mediant Refinement: The Algorithmic Engine}The generation of Farey sequences is driven by the mediant operation \cite{RanickiFareyProject, NumberanalyticsFarey}. The mediant of two fractions a/b and c/d is defined as (a+c)/(b+d). If a/b<c/d, their mediant always lies strictly between them \cite{KnottFarey, RanickiFareyProject, NumberanalyticsFarey}.A fundamental theorem states that the sequence Fn+1​ can be constructed from Fn​ by identifying all adjacent pairs of fractions a/b and c/d in Fn​ and inserting their mediant (a+c)/(b+d) between them if and only if the denominator of the mediant satisfies b+d=n+1 \cite{DUMMIT, Zukin2016, JNSFarey}. This provides a fully algorithmic and constructive method for generating the entire hierarchy \cite{Zukin2016, JNSFarey}.\subsection{The Stern-Brocot Tree: The Universal Genealogy of Rationals}The mediant operation, when applied recursively without the denominator constraint, generates the Stern-Brocot tree, an infinite binary tree that contains every positive rational number exactly once \cite{WikipediaSternBrocot, CPAlgorithmsSternBrocot}. It is constructed by starting with the "ancestors" 0/1 and 1/0 (representing 0 and infinity) and iteratively inserting the mediant between adjacent fractions \cite{WikipediaSternBrocot, CutTheKnotSternBrocot}. The Farey sequence Fn​ can be recovered by an in-order traversal of the tree, pruning any branch where a denominator exceeds n \cite{WikipediaSternBrocot, CPAlgorithmsSternBrocot}.\subsection{Foundational Properties and their Significance}\begin{itemize}\item \textbf{The Unimodular Relation:} If two fractions a/b and c/d are adjacent in any Farey sequence Fn​, they satisfy the unimodular relation: bc−ad=1 \cite{DUMMIT, Zukin2016}. This invariant guarantees that any mediant formed from two neighbors is itself irreducible \cite{Zukin2016, JNSFarey}.\item \textbf{Optimality in Diophantine Approximation:} Farey sequences contain the set of "best rational approximations" of the first kind to any real number for a given denominator bound n \cite{Zukin2016, JNSFarey}.\item \textbf{The Emergent Continuum:} The union of all Farey sequences, ⋃n=1∞​Fn​, constitutes the set of all rational numbers in $$\cite{CutTheKnotFarey}. The real continuum$$ is the topological closure of this constructively generated set \cite{Zukin2016}.\end{itemize}\section{The Isomorphism of Process: Unifying the Framework and its Realization}The parallels between the AoO framework and the Farey system are not merely superficial. The core generative processes of both systems are structurally identical.\subsection{Local Refinement as Constrained Expansion}The fundamental mechanism of evolution in both systems is a process of local refinement governed by a global constraint.\begin{itemize}\item In the Farey system, an interval (a/b,c/d) in Fn​ is refined by the mediant (a+c)/(b+d) only when the denominator constraint b+d=n+1 is met \cite{Zukin2016, JNSFarey}.\item In the powerset hierarchy, the transition from P(Ωn​) to P(Ωn+1​) introduces new subsets defined by the inclusion of the single new element {n+1} \cite{WikipediaPowerset, ElKhettabi2024HCN}.\end{itemize}The mediant insertion rule is a perfect instantiation of the AoO's abstract principle of "constraint-guided differentiation" \cite{ElKhettabi2025AoO}. The parameter n acts as a universal filter, determining which new elements are actualized at each stage.\subsection{The Arrow of Complexity in Nested Hierarchies}There is a formal mapping between the set inclusion Fn​⊂Fn+1​ and the powerset inclusion P(Ωn​)⊂P(Ωn+1​). Despite different growth rates (polynomial for ∣Fn​∣∼3n2/π2 vs. exponential for ∣P(Ωn​)∣=2n), both systems exhibit an irreversible "arrow of complexity" driven by the same underlying progression, 1→n→n+1 \cite{ElKhettabi2025AoO, WikipediaFarey, Zukin2016}.\section{A Unification of Algebraic Structures}In algebraic thinking, the “rules” are not just about solving for unknowns, but about determining the space of all possible solutions and how they interrelate. The correspondence between the Farey and powerset systems extends to the very algebraic structures they embody, revealing a profound duality.\begin{table}[h!]\centering\caption{Structural Parallels between Farey and Powerset Systems}\label{tab:duality}\begin{tabular}{|p{2.5cm}|p{4cm}|p{4cm}|p{3.5cm}|}\hline\textbf{Aspect} & \textbf{Farey/Stern-Brocot System} & \textbf{Powerset Hierarchy} & \textbf{Framework Link} \\hline\textbf{Elements} & Irreducible fractions p/q & Subsets S⊆Ωn​ & Configurations of n degrees of freedom \cite{ElKhettabi2025AoO, Zukin2016} \\hline\textbf{Ordering} & Standard numerical order < & Subset inclusion ⊂ & Hierarchical nesting \cite{ElKhettabi2025AoO, Zukin2016} \\hline\textbf{Refinement Rule} & Mediant b+da+c​ if b+d≤n+1 \cite{Zukin2016, JNSFarey} & Add subsets containing n+1 \cite{WikipediaPowerset} & Constraint-guided differentiation \cite{ElKhettabi2025AoO} \\hline\textbf{Local Algebra} & Neighbors a/b,c/d form a matrix in $\SL(2,\mathbb{Z})$ \cite{Zukin2016, DUMMIT} & --- & Structure of local interactions \\hline\textbf{Global Algebra} & --- & (P(Ωn​),Δ) is isomorphic to F2n​ \cite{WikipediaPowerset} & Algebra of system states \cite{ElKhettabi2024HCN} \\hline\textbf{Geometric View} & Ford Circles, Farey Tessellation of H2 \cite{WikipediaFarey, Zukin2016} & Projective planes/spaces over F2​ \cite{ElKhettabi2025AoO} & Emergent geometry \cite{ElKhettabi2025AoO} \\hline\textbf{Asymptotic Limit} & The real continuum (0,1) \cite{Zukin2016} & The Cantor space 2N, cardinality of continuum \cite{WikipediaPowerset} & The emergent continuum \cite{ElKhettabi2025AoO} \\hline\end{tabular}\end{table}\subsection{The Modular Group $\SL(2,\mathbb{Z})$: The Invariant Algebra of Farey Adjacency}The unimodular relation bc−ad=1 is the defining characteristic of the special linear group $\SL(2,\mathbb{Z})$ \cite{ConradSL2Z, WikipediaSL2Z}. For any pair of adjacent Farey neighbors a/b and c/d, the matrix (ab​cd​) is an element of $\SL(2,\mathbb{Z})$ \cite{Zukin2016, DUMMIT}. This group acts as the group of orientation-preserving automorphisms of the Farey graph, which tessellates the hyperbolic plane \cite{ConradSL2Z, FareyGraphSymmetry}. This connection is further deepened by the study of frieze patterns, where the work of Conway and Coxeter revealed combinatorial interpretations of the modular group related to polygon dissections and the Farey tessellation. $\SL(2,\mathbb{Z})$ is thus the infinite, non-abelian group that governs the local, dynamic structure of the Farey sequence \cite{ConradSL2Z}.\subsection{The Vector Space F2n​: The Canonical Algebra of System Configurations}The powerset P(Ωn​), when equipped with the operation of symmetric difference (Δ), forms a finite abelian group \cite{WikipediaPowerset, BooleanRing}. This group is isomorphic to the n-dimensional vector space over the finite field of two elements, F2n​ \cite{F2nVectorSpace, BooleanRing}. This algebraic structure is presented in the AoO framework as the canonical algebra for describing the global, static set of all possible configurations of a system with n binary degrees of freedom \cite{ElKhettabi2025AoO, ElKhettabi2024HCN}.\subsection{The Duality of Local (Non-Abelian) and Global (Abelian) Algebra}The synthesis of these systems reveals a profound duality. The AoO framework, by asserting that the Farey system is a concrete instantiation of its principles, implicitly predicts this juxtaposition of the infinite, non-abelian group $\SL(2,\mathbb{Z})$ with the sequence of finite, abelian groups F2n​. This key observation can be summarized as follows: The Farey system’s local, non-abelian symmetry ($\SL(2,\mathbb{Z})$) governs the dynamic refinement of the space \cite{Zukin2016, ConradSL2Z}, while the powerset’s global, abelian symmetry (F2n​) governs the static configuration of all possible states \cite{WikipediaPowerset, BooleanRing}. This duality between dynamic refinement and static configuration is a central feature of the AoO framework's explanatory power.\section{Hierarchical Perception of the Continuum through Degrees of Freedom}The relationship between the sets of mediants generated at different stages of the Farey hierarchy captures how systems with different degrees of freedom "perceive" the continuum. This section formalizes this concept, demonstrating that each degree of freedom unlocks a distinct, non-overlapping "vision" of the emergent continuum.\subsection{Defining the "Vision" of a Degree of Freedom: The Sets Wn​ and Wm​}For a system with a given degree of freedom, represented by the integer n, we can define its specific interval of focus and the set of rational numbers it generates to refine that interval.\begin{itemize}\item \textbf{Interval of Focus for degree n:} The interval In​=(n+11​,n1​) is defined by two fractions that become Farey neighbors in the sequence Fn+1​ \cite{Zukin2016}.\item \textbf{Mediant Refinement Set Wn​:} We define Wn​ as the set of all rational numbers that recursively refine the interval In​. This set is generated by the iterative application of the mediant operation, a process identical to the construction of a local Stern-Brocot tree within that interval \cite{Zukin2016, NumberanalyticsFarey}.\begin{itemize}\item \textbf{Properties of Wn​:}\item Wn​ is \textbf{dense} in (n+11​,n1​), as the mediant operation eventually fills every gap between rational numbers \cite{DUMMIT}.\item Wn​ is \textbf{infinite} and \textbf{ordered} according to the structure of the Stern-Brocot process \cite{WikipediaSternBrocot, CPAlgorithmsSternBrocot}.\end{itemize}\end{itemize}For a system with a higher degree of freedom m>n:\begin{itemize}\item \textbf{Interval of Focus for degree m:} Im​=(m+11​,m1​).\item \textbf{Mediant Refinement Set Wm​:} Wm​ is similarly defined as the set of all mediants that recursively refine the interval Im​.\end{itemize}\begin{theorem}Let In​=(n+11​,n1​) and define Wn​ as the set of all rationals generated by repeated mediant refinement within In​. Then for any m>n+1, the sets Wn​ and Wm​ are disjoint:[W_n \cap W_m = \emptyset.]Moreover, the rational continuum in (0,1) satisfies[\bigcup_{n=1}^{\infty} W_n = \mathbb{Q} \cap (0,1),\quad \text{and} \quad\overline{\bigcup_{n=1}^{\infty} W_n} = (0,1).]\end{theorem}\begin{remark}The non-overlapping property of the Wn​ sets formalizes the notion that a physical system with n quantized degrees of freedom cannot access or resolve measurement bands defined by higher m>n+1. This mirrors how discrete energy levels, finite sampling, or Planck-scale limits constrain physical measurement in quantum theory.\end{remark}\begin{corollary}Within the Arithmetic of Order (AoO) framework, the concept of ``zero'' is not assumed as a primitive numerical constant but emerges constructively as a relational boundary condition. The ordered powerset hierarchy begins with the empty set {} as a symbolic holder, not a numerical zero. Its powerset P({})={{}} generates the first distinguishable container. The introduction of the first degree of freedom {1} yields P({1})={{},{1}}, whose cardinality 21=2=1+1 demonstrates that counting is rooted in nesting and ordering, not in a naked zero.In parallel, the Farey sequence never invokes an a priori continuum zero but uses 0/1 as a finite measurement anchor. Its mediant operation,[\frac{a}{b} \oplus \frac{c}{d} = \frac{a+c}{b+d},]ensures that each refinement step preserves the integer basis for measurement, approaching the continuum limit only asymptotically via the nested bands Wn​.Thus, the continuum’s “nude zero” appears solely as the relational closure of finitely constructed measurement steps: “zero” is never primitive but always contextualized by the ordered degrees of freedom that define its limit.\end{corollary}\subsection{Implications for the Arithmetic of Order Framework}This model of hierarchical, disjoint perception has significant implications for the AoO framework.\begin{enumerate}\item \textbf{Degree-of-Freedom-Dependent Reality:} The continuum is not perceived uniformly but as a hierarchy of refinements, where each "vision" is tied to the observer’s available degrees of freedom (n or m) \cite{ElKhettabi2025AoO}. A physical system with n variables cannot "see" the finer structure revealed by m>n variables.\item \textbf{Finitistic Continuum:} The "complete" continuum is the limit of all Wk​, but no system with a finite number of degrees of freedom can fully capture it. This aligns with the core philosophy of finitism \cite{ElKhettabi2025AoO, SEP_Finitism}.\item \textbf{Algebraic Consistency:} The disjointness of Wn​ and Wm​ is a concrete example of the AoO’s principle of constraint-guided differentiation \cite{ElKhettabi2025AoO}. Each degree of freedom k introduces structure only in its designated interval (k+11​,k1​).\end{enumerate}\section{Conclusion and Outlook}The rigorous analysis presented here confirms that the Farey sequence hierarchy Fn​ and the infinite Stern-Brocot tree provide a direct and operational instantiation of the core principles articulated in the \emph{Arithmetic of Order} framework. By demonstrating a formal isomorphism between the Farey system’s mediant-based local refinement and the AoO’s principle of \emph{constraint-guided differentiation}, this work extends the finitistic, constructivist agenda championed in El Khettabi’s foundational reports on powerset combinatorics, hypercomplex numbers, and emergent geometries \cite{ElKhettabi2025AoO, ElKhettabi2024HCN, ElKhettabi2024PLOS}.Crucially, the Farey system shows that the real continuum --- historically assumed as an \emph{a priori} infinite structure --- can instead be understood as the asymptotic closure of a fully discrete, algorithmically generable process \cite{SEP_Finitism, Zukin2016}. Every rational in the unit interval is positioned within a nested, well-ordered hierarchy whose generation is entirely finitistic and transparent \cite{WikipediaFarey, Zukin2016}. The Farey mediant mechanism mirrors the XOR-bitwise operation on powersets: both systems employ a local, deterministic rule to refine an initial configuration space under a global constraint parameter, here the denominator bound n \cite{WikipediaPowerset, JNSFarey}.The duality between local non-abelian modular symmetries ($\SL(2,\mathbb{Z})$) and the global abelian powerset algebra (F2n​) reinforces the AoO’s insight that the richness of mathematical and physical structure can emerge from the simple arithmetic of ordered degrees of freedom \cite{ElKhettabi2025AoO, ConradSL2Z, BooleanRing}. Just as the Golay code G24​ and the Leech lattice Λ24​ arise from sieving the powerset P(Ω24​), a process whose outputs were famously used by Conway and others to construct the Leech lattice, so too does the Farey hierarchy filter the Stern-Brocot tree into optimally ordered sets of best rational approximations \cite{DUMMIT}.By placing the Farey system within this finitistic combinatorial paradigm, this article not only strengthens the AoO’s theoretical claims but demonstrates its flexibility and universality across number theory, coding theory, and the modeling of continuum phenomena within strictly finite means. In this sense, the Farey hierarchy provides an explicit model of physics as quantized measurement: each mediant, each denominator constraint, each combinatorial power represents a discrete quantum of possible states. The real continuum, as the limit of these finite steps, is not a physical prerequisite but an emergent idealization of finitely resolved measurements.\textbf{Future Directions.} This synthesis invites further exploration along several lines. First, the deep interplay between Farey adjacency, modular tessellations, and hyperbolic geometry (Ford circles, Farey tessellation of H2) deserves to be mapped explicitly onto the projective geometries naturally emerging from the ordered powerset hierarchy \cite{ElKhettabi2025AoO, WikipediaFarey, FareyGraphSymmetry}. Second, the constructive mediant process suggests algorithmic avenues for designing finite, resource-bounded AI systems capable of performing Diophantine approximations without recourse to continuum assumptions \cite{ElKhettabi2025AoO, WikipediaSternBrocot}. Finally, the formal analogy between Farey trees and the nested powerset combinatorics suggests potential generalizations to higher-dimensional hypercomplex structures and their associated finite geometries.In unifying the Farey sequence and the Arithmetic of Order framework, we highlight a powerful theme: the apparent paradox of continuum structures in finite physical systems dissolves when seen as the limit of simple, local, finitistic rules. Mathematics, under this lens, is not an edifice built on the infinite, but a revelation of emergent order rooted in the finite --- one degree of freedom at a time.\vspace{1em}\noindent\textbf{Faysal El Khettabi} \\emph{Ensemble AIs} \\texttt{faysal.el.khettabi@gmail.com} \July 2025\appendix\section{Appendix}\subsection{Proof of the Unimodular Relation via Pick's Theorem}A key property of Farey sequences is that if a/b and c/d are adjacent terms (neighbors), then bc−ad=1. A geometric proof utilizes Pick's Theorem, which relates the area of a simple polygon whose vertices are points on the integer lattice to the number of integer points on its boundary and in its interior \cite{DUMMIT}. The area A is given by A=I+2B​−1, where I is the number of interior lattice points and B is the number of boundary lattice points.Consider the triangle △ with vertices at the origin O(0,0), P1​(b,a), and P2​(d,c). The area of this triangle can be calculated using the determinant formula, which gives A=21​∣bc−ad∣. Since a/b<c/d, we have bc>ad, so the area is A=21​(bc−ad).Now, we apply Pick's Theorem to this triangle:\begin{itemize}\item \textbf{Boundary Points (B):} The vertices O, P1​, and P2​ are lattice points. Since the fractions a/b and c/d are irreducible, gcdf​unc(a,b)=1 and gcdf​unc(c,d)=1. This implies there are no other lattice points on the segments OP1​ and OP2​. If there were a lattice point on the segment P1​P2​, it would represent a rational fraction that should lie between a/b and c/d in the Farey sequence but with a smaller denominator, which is impossible for neighbors. Thus, the only lattice points on the boundary are the three vertices, so B=3 \cite{DUMMIT}.\item \textbf{Interior Points (I):} Suppose there were a lattice point (x,y) in the interior of △. The fraction y/x would have a value strictly between a/b and c/d. Since b≤n and d≤n, it would follow that x<n. This would mean that y/x is a fraction with a denominator smaller than n that lies between a/b and c/d, contradicting the assumption that they are consecutive terms in Fn​. Therefore, there can be no lattice points in the interior of the triangle, and I=0 \cite{DUMMIT}.\end{itemize}Applying Pick's Theorem with I=0 and B=3, the area of the triangle is A=0+23​−1=21​. Equating this with our determinant formula, we get 21​(bc−ad)=21​, which directly implies bc−ad=1. This provides a purely combinatorial confirmation that the mediant always yields an irreducible fraction when the unimodular condition is met.\subsection{The Isomorphism (P(Ωn​),Δ)≅F2n​}The powerset P(Ωn​) of a set Ωn​={1,2,…,n}, when equipped with the operation of symmetric difference (Δ), forms a finite abelian group \cite{WikipediaPowerset, SymmetricDifferenceGroup}.\begin{itemize}\item \textbf{Closure:} For any two subsets A,B⊆Ωn​, their symmetric difference AΔB=(A∪B)∖(A∩B) is also a subset of Ωn​.\item \textbf{Associativity:} The operation is associative: (AΔB)ΔC=AΔ(BΔC).\item \textbf{Identity Element:} The empty set ∅ serves as the identity element, as AΔ∅=A.\item \textbf{Inverse Element:} Every element is its own inverse, as AΔA=∅.\end{itemize}This group is isomorphic to the n-dimensional vector space over the field of two elements, F2​={0,1}, denoted F2n​ \cite{F2nVectorSpace, BooleanRing}. The isomorphism is established by mapping each subset S⊆Ωn​ to its characteristic function (or binary vector) of length n. The symmetric difference of two subsets corresponds precisely to the component-wise addition (XOR operation) of their corresponding vectors in F2n​. This mapping makes the group operation identical to vector addition modulo 2, clarifying the isomorphism’s computational meaning. The set of singleton subsets {{1},{2},…,{n}} forms a basis for this vector space \cite{F2nVectorSpace}.\begin{thebibliography}{99}\bibitem{WikipediaFoundations}Wikipedia contributors. (2024). \emph{Foundations of mathematics}. Wikipedia, The Free Encyclopedia.\bibitem{ElKhettabi2025AoO}El Khettabi, F. (2025). \emph{The Arithmetic of Order: A Finitistic Foundation for Mathematics, Emergent Structures, and Intelligent Systems}. viXra.\bibitem{ElKhettabi2024HCN}El Khettabi, F. (2024). \emph{A Comprehensive Modern Mathematical Foundation for Hypercomplex Numbers with Recollection of Sir William Rowan Hamilton, John T. Graves, and Arthur Cayley}.\bibitem{SEP_Constructivism}Iemhoff, R. (2023). \emph{Constructive Mathematics}. The Stanford Encyclopedia of Philosophy (Fall 2023 Edition), Edward N. Zalta & Uri Nodelman (eds.).\bibitem{WikipediaFarey}Wikipedia contributors. (2024). \emph{Farey sequence}. Wikipedia, The Free Encyclopedia.\bibitem{SEP_Finitism}Ye, F. (2021). \emph{Finitism in Geometry}. The Stanford Encyclopedia of Philosophy (Winter 2021 Edition), Edward N. Zalta (ed.).\bibitem{Zukin2016}Zukin, M. (2016). \emph{The Farey Sequence}. Whitman College.\bibitem{DUMMIT}Dummit, D. S., & Foote, R. M. (2004). \emph{Abstract Algebra}. John Wiley & Sons.\bibitem{ElKhettabi2024PLOS}El Khettabi, F. (2024). \emph{On the hypercomplex numbers and normed division algebra of all dimensions: A unified multiplication}. PLOS ONE, 19(6), e0312502.\bibitem{WikipediaPowerset}Wikipedia contributors. (2024). \emph{Power set}. Wikipedia, The Free Encyclopedia.\bibitem{ElKhettabi2025AoO_vixra}El Khettabi, F. (2025). The Arithmetic of Order: A Finitistic Foundation for Mathematics, Emergent Structures, and Intelligent Systems. \emph{viXra:2505.0064}.\bibitem{JNSFarey}Tamang, B. B., et al. (2022). Some characteristics of the Farey sequences with Ford circles. \emph{Nepal Journal of Mathematical Sciences}, 4(1), 69-76.\bibitem{RanickiFareyProject}Ranicki, A. (n.d.). \emph{The Farey sequence and its applications}.\bibitem{KnottFarey}Knott, R. \emph{Farey Series and the Stern-Brocot Tree}. University of Surrey.\bibitem{NumberanalyticsFarey}Number Analytics. (n.d.). \emph{Farey Sequences: A Deep Dive into Additive Number Theory}.\bibitem{WikipediaSternBrocot}Wikipedia contributors. (2024). \emph{Stern–Brocot tree}. Wikipedia, The Free Encyclopedia.\bibitem{CPAlgorithmsSternBrocot}CP-Algorithms. \emph{Stern-Brocot Tree and Farey Sequences}.\bibitem{CutTheKnotFarey}Bogomolny, A. \emph{Farey Series}. Cut-the-Knot.\bibitem{CutTheKnotSternBrocot}Bogomolny, A. \emph{Stern-Brocot Tree}. Cut-the-Knot.\bibitem{ConradSL2Z}Conrad, K. \emph{SL(2,Z)}. University of Connecticut.\bibitem{WikipediaSL2Z}Wikipedia contributors. (2024). \emph{Modular group}. Wikipedia, The Free Encyclopedia.\bibitem{FareyGraphSymmetry}Lutsko, C. (2021). \emph{Generalized Farey sequences}. International Mathematics Research Notices.\bibitem{BooleanRing}Wikipedia contributors. (2024). \emph{Boolean ring}. Wikipedia, The Free Encyclopedia.\bibitem{F2nVectorSpace}Stack Exchange. (2018). \emph{P(X) with symmetric difference as addition as a vector space over Z2}.\bibitem{SymmetricDifferenceGroup}ProofWiki. \emph{Symmetric Difference on Power Set forms Abelian Group}.\bibitem{HardyWright}Hardy, G. H., & Wright, E. M. (1979). \emph{An Introduction to the Theory of Numbers}. Oxford University Press.\bibitem{ConwaySloane}Conway, J. H., & Sloane, N. J. A. (1999). \emph{Sphere Packings, Lattices, and Groups}. Springer.\end{thebibliography}\end{document}














\documentclass[12pt,a4paper]{article}
% PACKAGES
\usepackage[margin=1in]{geometry}
\usepackage{amsmath, amssymb, amsthm}
\usepackage{authblk}
\usepackage[utf8]{inputenc}
\usepackage{fontenc}
\usepackage{hyperref}
\hypersetup{
colorlinks=true,
linkcolor=blue,
filecolor=magenta,
urlcolor=cyan,
pdftitle={The Farey Sequence as a Realization of the Arithmetic of Order},
pdfpagemode=FullScreen,
}
% MATH OPERATORS AND ENVIRONMENTS
\DeclareMathOperator{\im}{Im}
\DeclareMathOperator{\re}{Re}
\DeclareMathOperator{\SL}{SL}
\DeclareMathOperator{\PSL}{PSL}
\DeclareMathOperator{\GL}{GL}
\DeclareMathOperator{\gcd_func}{gcd}
\newtheorem{theorem}{Theorem}[section]
\newtheorem{proposition}[theorem]{Proposition}
\newtheorem{lemma}[theorem]{Lemma}
\newtheorem{corollary}[theorem]{Corollary}
\theoremstyle{definition}
\newtheorem{definition}[theorem]{Definition}
\newtheorem{example}[theorem]{Example}
\theoremstyle{remark}
\newtheorem{remark}[theorem]{Remark}
% TITLE AND AUTHOR
\title{Physics as Quantized Measurement: The Farey Sequence as a Realization of the Arithmetic of Order}
\author{Faysal El Khettabi}
\affil{Ensemble AIs\ \texttt{faysal.el.khettabi@gmail.com}}
\date{July 2025}
\begin{document}
\maketitle
\tableofcontents
\begin{abstract}
This article formalizes the Farey sequence hierarchy (Fn​) and the universal Stern-Brocot tree (Tn​) as not merely analogies but explicit constructive realizations of the ``Arithmetic of Order'' (AoO) framework. It synthesizes the generative principles of constraint-guided differentiation, the progression 1→n→n+1, and the nested powerset structure, demonstrating that the Farey system embodies the finitistic, emergent continuum perspective central to the AoO thesis. We propose that the Farey numbers, generated by a finite, deterministic process, provide a natural model for physical quantities and measurements, replacing the need for an a priori infinite continuum. This synthesis reinforces a unified finitistic approach to foundational mathematics, hypercomplex number theory, projective geometries, and the algorithmic design of intelligent systems.
\end{abstract}
\section{Introduction: A Finitistic Recasting of the Continuum}
The core argument of this report is that the Farey system provides a complete and rigorous mathematical prototype for the "Arithmetic of Order" framework \cite{ElKhettabi2025AoO}. This is not a relationship of analogy, but of instantiation, where the abstract principles of the latter find their direct, operational expression in the former.
The foundations of modern mathematics and physics have been shaped by a foundational crisis that emerged in the late 19th and early 20th centuries \cite{WikipediaFoundations}. A central feature of the resolution to this crisis was the widespread adoption of infinitary concepts, most notably the continuum of real and complex numbers. The imaginary unit i=−1​∈C, for instance, is now central to the formulation of quantum theory \cite{ElKhettabi2024HCN, ElKhettabi2025AoO}. Yet, this reliance introduces a profound conceptual paradox: how can a finite physical system—a quantum register, a molecule, or any object composed of a finite number of components—fundamentally require an infinite mathematical construct for its description? \cite{ElKhettabi2025AoO}.
The "Arithmetic of Order" (AoO) framework directly confronts this paradox by proposing a mathematics built from the finite and observable, where complexity and structure emerge constructively \cite{ElKhettabi2025AoO}. This approach aligns with the philosophical traditions of finitism and constructivism \cite{SEP_Finitism, SEP_Constructivism}. Finitism, in its various forms, questions or rejects the existence of actual infinite objects, such as the set of all natural numbers, proposing that mathematics should be grounded in objects that are, at least in principle, finite \cite{WikipediaFarey, SEP_Finitism}. Constructivism insists that mathematical existence is tied to algorithmic constructibility; to prove an object exists, one must provide a method of finding (“constructing”) such an object \cite{SEP_Constructivism, Zukin2016}. The AoO framework synthesizes these views by critiquing the traditional reliance on a priori infinities and proposing a new foundation where the continuum itself is not a given axiom but an emergent property \cite{ElKhettabi2025AoO}.
This report demonstrates that the Farey sequence system offers a perfect, non-trivial model for this finitistic philosophy. The Farey sequence, generated by a simple, finite, and deterministic algorithm, provides a concrete pathway to the rational numbers and, by extension, to the real continuum. Its generation does not presuppose the existence of the continuum but rather builds it step-by-step, deriving it as an asymptotic limit of a sequence of finite structures \cite{WikipediaFarey, Zukin2016}. This aligns perfectly with the AoO's reinterpretation of the continuum as a limit of a nested hierarchy of finite sets \cite{ElKhettabi2025AoO}.
The structure of this analysis will proceed as follows. First, the abstract principles of the AoO framework will be detailed. Second, a rigorous exposition of the Farey system will be provided. The central sections will then establish the formal mapping between the generative processes and hierarchical structures of these two domains. This is followed by a deep analysis of their corresponding algebraic structures, a discussion of the hierarchical perception of the continuum, and finally, an exploration of the implications of this synthesis.
\section{The "Arithmetic of Order": A Framework of Emergent Complexity}
The "Arithmetic of Order" (AoO) is a foundational framework that seeks to re-establish mathematics on finite, constructive principles. It posits that the intricate structures of mathematics are not arbitrary inventions but are natural consequences of the most fundamental process of ordered progression \cite{ElKhettabi2025AoO}.
\subsection{The Generative Progression 1→n→n+1}
The central axiom of the AoO framework is that all of mathematics can be understood as a "natural revelation of the intrinsic structure embedded in the progression 1→n→n+1" \cite{ElKhettabi2025AoO}. This progression is not merely a representation of counting; it embodies the fundamental process of incrementally adding a new degree of freedom to a system \cite{ElKhettabi2025AoO, ElKhettabi2024HCN}. In physical terms, each step n→n+1 can be interpreted as the introduction of a new quantized degree of freedom — an additional state, bit, or mode that refines the system’s resolution of measurable quantities. For example, each new fraction in the Farey hierarchy refines an interval with finite rational steps, mirroring how physical measurement adds finite resolution at each scale. This reflects the same principle that underlies Planck’s constant in quantum mechanics: physical observables do not vary continuously in theory but are constrained by finite quanta of action and resolution. In the context of physics, a system's state is defined by a finite number of such degrees of freedom, and understanding how the system's properties change as this number increases is paramount \cite{ElKhettabi2024HCN}.
The framework emphasizes that it is the ordered nature of the underlying set Ωn​={1,2,…,n} that is crucial \cite{ElKhettabi2025AoO}. This ordering ensures that the hierarchy of structures built upon it is well-defined, recursive, and nested. Each step from n to n+1 represents a deterministic expansion of the system's potential, opening up new combinatorial possibilities and enabling the emergence of higher-order symmetries and more complex structures \cite{ElKhettabi2025AoO}.
\subsection{The Powerset Hierarchy P(Ωn​) as the Canonical State Space}
Within the AoO, the canonical state space of a system with n degrees of freedom is identified with the powerset of Ωn​, denoted P(Ωn​) \cite{ElKhettabi2025AoO}. The powerset is the set of all subsets of Ωn​, including the empty set and Ωn​ itself \cite{WikipediaPowerset}. Each element of P(Ωn​)—that is, each subset S⊆Ωn​—represents a distinct configuration of the system. This configuration can be encoded by a characteristic function, a binary vector indicating which degrees of freedom are "active" or "present" in that state \cite{ElKhettabi2025AoO, WikipediaPowerset}.
The generative progression 1→n→n+1 manifests directly as a nested hierarchy of these powerset state spaces: P(Ωn​)⊂P(Ωn+1​) \cite{ElKhettabi2024HCN}. The construction of the next level of the hierarchy is both recursive and fully deterministic. Given P(Ωn​), the powerset P(Ωn+1​) is formed by taking all existing subsets in P(Ωn​) and adding to them a new collection of subsets, each formed by taking an element of P(Ωn​) and adjoining the new element {n+1} \cite{WikipediaPowerset}. Formally, this is expressed as:

The cardinality of the powerset, ∣P(Ωn​)∣=2n, grows exponentially, highlighting the rapid increase in structural complexity that arises from the addition of each new degree of freedom \cite{ElKhettabi2024HCN}.
\subsection{The Continuum as an Asymptotic Limit}
A cornerstone of the AoO framework is its explicit rejection of the continuum as an a priori entity \cite{ElKhettabi2025AoO}. Instead, the real continuum is reinterpreted as an asymptotic limit of the nested powerset hierarchy. As the number of degrees of freedom n tends towards infinity, the combinatorial and topological properties of the finite space P(Ωn​) approach those traditionally ascribed to continuous systems \cite{ElKhettabi2025AoO}.
This perspective aligns with finitist philosophies that are skeptical of actual, completed infinities but are comfortable with the concept of potential infinity or unbounded processes \cite{WikipediaFarey, SEP_Finitism}. The framework provides a concrete model for how this limit is approached. The cardinality of the powerset, 2n, naturally approaches the cardinality of the continuum, 2ℵ0​, as n approaches the first infinite cardinal, ℵ0​. The Cantor space, which is homeomorphic to the set of infinite binary sequences {0,1}N, serves as a topological bridge, as it is in one-to-one correspondence with P(N) and has the cardinality of the continuum \cite{WikipediaPowerset}.
\subsection{Constraint-Guided Differentiation and Emergent Structures}
A key generative mechanism within the AoO framework is termed "constraint-guided differentiation" \cite{ElKhettabi2025AoO}. This principle posits that by applying specific combinatorial rules or constraints to the powerset hierarchy, certain "optimal" mathematical structures can emerge deterministically. Examples cited within the framework's literature include the emergence of exceptional structures like the Golay code G24​ and the Leech lattice Λ24​ from P(Ω24​) when specific constraints are applied \cite{ElKhettabi2025AoO}.
\section{The Farey System: A Deterministic Path to the Rationals}
The Farey sequence and its related structures, such as the Stern-Brocot tree, form a cornerstone of elementary number theory, providing a constructive method for generating the set of rational numbers.
\subsection{The Farey Sequence Fn​: Definition and Hierarchical Construction}
The Farey sequence of order n, denoted Fn​, is formally defined as the sequence of irreducible fractions a/b in the closed interval (0,1) such that the denominator b is a positive integer satisfying 1≤b≤n, arranged in ascending order of magnitude \cite{WikipediaFarey, Zukin2016}. Each sequence conventionally starts with 0/1 and ends with 1/1 \cite{WikipediaFarey}.
The first few Farey sequences are \cite{Zukin2016, RanickiFareyProject}:
\begin{itemize}
\item F1​:{10​,11​}
\item F2​:{10​,21​,11​}
\item F3​:{10​,31​,21​,32​,11​}
\item F4​:{10​,41​,31​,21​,32​,43​,11​}
\end{itemize}
A fundamental property of this construction is its hierarchical nature: Fn​⊂Fn+1​ for all n≥1 \cite{WikipediaFarey, Zukin2016}. The number of new fractions added at step n+1 is precisely given by Euler's totient function, ϕ(n+1) \cite{WikipediaFarey, Zukin2016}.
\subsection{Mediant Refinement: The Algorithmic Engine}
The generation of Farey sequences is driven by the mediant operation \cite{RanickiFareyProject, NumberanalyticsFarey}. The mediant of two fractions a/b and c/d is defined as (a+c)/(b+d). If a/b<c/d, their mediant always lies strictly between them \cite{KnottFarey, RanickiFareyProject, NumberanalyticsFarey}.
A fundamental theorem states that the sequence Fn+1​ can be constructed from Fn​ by identifying all adjacent pairs of fractions a/b and c/d in Fn​ and inserting their mediant (a+c)/(b+d) between them if and only if the denominator of the mediant satisfies b+d=n+1 \cite{DUMMIT, Zukin2016, JNSFarey}. This provides a fully algorithmic and constructive method for generating the entire hierarchy \cite{Zukin2016, JNSFarey}.
\subsection{The Stern-Brocot Tree: The Universal Genealogy of Rationals}
The mediant operation, when applied recursively without the denominator constraint, generates the Stern-Brocot tree, an infinite binary tree that contains every positive rational number exactly once \cite{WikipediaSternBrocot, CPAlgorithmsSternBrocot}. It is constructed by starting with the "ancestors" 0/1 and 1/0 (representing 0 and infinity) and iteratively inserting the mediant between adjacent fractions \cite{WikipediaSternBrocot, CutTheKnotSternBrocot}. The Farey sequence Fn​ can be recovered by an in-order traversal of the tree, pruning any branch where a denominator exceeds n \cite{WikipediaSternBrocot, CPAlgorithmsSternBrocot}.
\subsection{Foundational Properties and their Significance}
\begin{itemize}
\item \textbf{The Unimodular Relation:} If two fractions a/b and c/d are adjacent in any Farey sequence Fn​, they satisfy the unimodular relation: bc−ad=1 \cite{DUMMIT, Zukin2016}. This invariant guarantees that any mediant formed from two neighbors is itself irreducible \cite{Zukin2016, JNSFarey}.
\item \textbf{Optimality in Diophantine Approximation:} Farey sequences contain the set of "best rational approximations" of the first kind to any real number for a given denominator bound n \cite{Zukin2016, JNSFarey}.
\item \textbf{The Emergent Continuum:} The union of all Farey sequences, ⋃n=1∞​Fn​, constitutes the set of all rational numbers in
$$\cite{CutTheKnotFarey}. The real continuum$$
is the topological closure of this constructively generated set \cite{Zukin2016}.
\end{itemize}
\section{The Isomorphism of Process: Unifying the Framework and its Realization}
The parallels between the AoO framework and the Farey system are not merely superficial. The core generative processes of both systems are structurally identical.
\subsection{Local Refinement as Constrained Expansion}
The fundamental mechanism of evolution in both systems is a process of local refinement governed by a global constraint.
\begin{itemize}
\item In the Farey system, an interval (a/b,c/d) in Fn​ is refined by the mediant (a+c)/(b+d) only when the denominator constraint b+d=n+1 is met \cite{Zukin2016, JNSFarey}.
\item In the powerset hierarchy, the transition from P(Ωn​) to P(Ωn+1​) introduces new subsets defined by the inclusion of the single new element {n+1} \cite{WikipediaPowerset, ElKhettabi2024HCN}.
\end{itemize}
The mediant insertion rule is a perfect instantiation of the AoO's abstract principle of "constraint-guided differentiation" \cite{ElKhettabi2025AoO}. The parameter n acts as a universal filter, determining which new elements are actualized at each stage.
\subsection{The Arrow of Complexity in Nested Hierarchies}
There is a formal mapping between the set inclusion Fn​⊂Fn+1​ and the powerset inclusion P(Ωn​)⊂P(Ωn+1​). Despite different growth rates (polynomial for ∣Fn​∣∼3n2/π2 vs. exponential for ∣P(Ωn​)∣=2n), both systems exhibit an irreversible "arrow of complexity" driven by the same underlying progression, 1→n→n+1 \cite{ElKhettabi2025AoO, WikipediaFarey, Zukin2016}.
\section{A Unification of Algebraic Structures}
In algebraic thinking, the “rules” are not just about solving for unknowns, but about determining the space of all possible solutions and how they interrelate. The correspondence between the Farey and powerset systems extends to the very algebraic structures they embody, revealing a profound duality.
\begin{table}[h!]
\centering
\caption{Structural Parallels between Farey and Powerset Systems}
\label{tab:duality}
\begin{tabular}{|p{2.5cm}|p{4cm}|p{4cm}|p{3.5cm}|}
\hline
\textbf{Aspect} & \textbf{Farey/Stern-Brocot System} & \textbf{Powerset Hierarchy} & \textbf{Framework Link} \
\hline
\textbf{Elements} & Irreducible fractions p/q & Subsets S⊆Ωn​ & Configurations of n degrees of freedom \cite{ElKhettabi2025AoO, Zukin2016} \
\hline
\textbf{Ordering} & Standard numerical order < & Subset inclusion ⊂ & Hierarchical nesting \cite{ElKhettabi2025AoO, Zukin2016} \
\hline
\textbf{Refinement Rule} & Mediant b+da+c​ if b+d≤n+1 \cite{Zukin2016, JNSFarey} & Add subsets containing n+1 \cite{WikipediaPowerset} & Constraint-guided differentiation \cite{ElKhettabi2025AoO} \
\hline
\textbf{Local Algebra} & Neighbors a/b,c/d form a matrix in $\SL(2,\mathbb{Z})$ \cite{Zukin2016, DUMMIT} & --- & Structure of local interactions \
\hline
\textbf{Global Algebra} & --- & (P(Ωn​),Δ) is isomorphic to F2n​ \cite{WikipediaPowerset} & Algebra of system states \cite{ElKhettabi2024HCN} \
\hline
\textbf{Geometric View} & Ford Circles, Farey Tessellation of H2 \cite{WikipediaFarey, Zukin2016} & Projective planes/spaces over F2​ \cite{ElKhettabi2025AoO} & Emergent geometry \cite{ElKhettabi2025AoO} \
\hline
\textbf{Asymptotic Limit} & The real continuum (0,1) \cite{Zukin2016} & The Cantor space 2N, cardinality of continuum \cite{WikipediaPowerset} & The emergent continuum \cite{ElKhettabi2025AoO} \
\hline
\end{tabular}
\end{table}
\subsection{The Modular Group $\SL(2,\mathbb{Z})$: The Invariant Algebra of Farey Adjacency}
The unimodular relation bc−ad=1 is the defining characteristic of the special linear group $\SL(2,\mathbb{Z})$ \cite{ConradSL2Z, WikipediaSL2Z}. For any pair of adjacent Farey neighbors a/b and c/d, the matrix (ab​cd​) is an element of $\SL(2,\mathbb{Z})$ \cite{Zukin2016, DUMMIT}. This group acts as the group of orientation-preserving automorphisms of the Farey graph, which tessellates the hyperbolic plane \cite{ConradSL2Z, FareyGraphSymmetry}. $\SL(2,\mathbb{Z})$ is thus the infinite, non-abelian group that governs the local, dynamic structure of the Farey sequence \cite{ConradSL2Z}.
\subsection{The Vector Space F2n​: The Canonical Algebra of System Configurations}
The powerset P(Ωn​), when equipped with the operation of symmetric difference (Δ), forms a finite abelian group \cite{WikipediaPowerset, BooleanRing}. This group is isomorphic to the n-dimensional vector space over the finite field of two elements, F2n​ \cite{F2nVectorSpace, BooleanRing}. This algebraic structure is presented in the AoO framework as the canonical algebra for describing the global, static set of all possible configurations of a system with n binary degrees of freedom \cite{ElKhettabi2025AoO, ElKhettabi2024HCN}.
\subsection{The Duality of Local (Non-Abelian) and Global (Abelian) Algebra}
The synthesis of these systems reveals a profound duality. The AoO framework, by asserting that the Farey system is a concrete instantiation of its principles, implicitly predicts this juxtaposition of the infinite, non-abelian group $\SL(2,\mathbb{Z})$ with the sequence of finite, abelian groups F2n​. This key observation can be summarized as follows: The Farey system’s local, non-abelian symmetry ($\SL(2,\mathbb{Z})$) governs the dynamic refinement of the space \cite{Zukin2016, ConradSL2Z}, while the powerset’s global, abelian symmetry (F2n​) governs the static configuration of all possible states \cite{WikipediaPowerset, BooleanRing}. This duality between dynamic refinement and static configuration is a central feature of the AoO framework's explanatory power.
\section{Hierarchical Perception of the Continuum through Degrees of Freedom}
The relationship between the sets of mediants generated at different stages of the Farey hierarchy captures how systems with different degrees of freedom "perceive" the continuum. This section formalizes this concept, demonstrating that each degree of freedom unlocks a distinct, non-overlapping "vision" of the emergent continuum.
\subsection{Defining the "Vision" of a Degree of Freedom: The Sets Wn​ and Wm​}
For a system with a given degree of freedom, represented by the integer n, we can define its specific interval of focus and the set of rational numbers it generates to refine that interval.
\begin{itemize}
\item \textbf{Interval of Focus for degree n:} The interval In​=(n+11​,n1​) is defined by two fractions that become Farey neighbors in the sequence Fn+1​ \cite{Zukin2016}.
\item \textbf{Mediant Refinement Set Wn​:} We define Wn​ as the set of all rational numbers that recursively refine the interval In​. This set is generated by the iterative application of the mediant operation, a process identical to the construction of a local Stern-Brocot tree within that interval \cite{Zukin2016, NumberanalyticsFarey}.
\begin{itemize}
\item \textbf{Properties of Wn​:}
\item Wn​ is \textbf{dense} in (n+11​,n1​), as the mediant operation eventually fills every gap between rational numbers \cite{DUMMIT}.
\item Wn​ is \textbf{infinite} and \textbf{ordered} according to the structure of the Stern-Brocot process \cite{WikipediaSternBrocot, CPAlgorithmsSternBrocot}.
\end{itemize}
\end{itemize}
For a system with a higher degree of freedom m>n:
\begin{itemize}
\item \textbf{Interval of Focus for degree m:} Im​=(m+11​,m1​).
\item \textbf{Mediant Refinement Set Wm​:} Wm​ is similarly defined as the set of all mediants that recursively refine the interval Im​.
\end{itemize}
\begin{theorem}
Let In​=(n+11​,n1​) and define Wn​ as the set of all rationals generated by repeated mediant refinement within In​. Then for any m>n+1, the sets Wn​ and Wm​ are disjoint:
[
W_n \cap W_m = \emptyset.
]
Moreover, the rational continuum in (0,1) satisfies
[
\bigcup_{n=1}^{\infty} W_n = \mathbb{Q} \cap (0,1),
\quad \text{and} \quad
\overline{\bigcup_{n=1}^{\infty} W_n} = (0,1).
]
\end{theorem}
\begin{remark}
The non-overlapping property of the Wn​ sets formalizes the notion that a physical system with n quantized degrees of freedom cannot access or resolve measurement bands defined by higher m>n+1. This mirrors how discrete energy levels, finite sampling, or Planck-scale limits constrain physical measurement in quantum theory.
\end{remark}
\begin{corollary}
Within the Arithmetic of Order (AoO) framework, the concept of “zero” is not assumed as a primitive numerical constant but emerges constructively as a relational boundary condition. The ordered powerset hierarchy begins with the empty set {} as a symbolic holder, not a numerical zero. Its powerset P({})={{}} generates the first distinguishable container. The introduction of the first degree of freedom {1} yields P({1})={{},{1}}, whose cardinality 21=2=1+1 demonstrates that counting is rooted in nesting and ordering, not in a naked zero.
In parallel, the Farey sequence never invokes an a priori continuum zero but uses 0/1 as a finite measurement anchor. Its mediant operation,
[
\frac{a}{b} \oplus \frac{c}{d} = \frac{a+c}{b+d},
]
ensures that each refinement step preserves the integer basis for measurement, approaching the continuum limit only asymptotically via the nested bands Wn​.
Thus, the continuum’s “nude zero” appears solely as the relational closure of finitely constructed measurement steps: “zero” is never primitive but always contextualized by the ordered degrees of freedom that define its limit.
\end{corollary}
\subsection{Implications for the Arithmetic of Order Framework}
This model of hierarchical, disjoint perception has significant implications for the AoO framework.
\begin{enumerate}
\item \textbf{Degree-of-Freedom-Dependent Reality:} The continuum is not perceived uniformly but as a hierarchy of refinements, where each "vision" is tied to the observer’s available degrees of freedom (n or m) \cite{ElKhettabi2025AoO}. A physical system with n variables cannot "see" the finer structure revealed by m>n variables.
\item \textbf{Finitistic Continuum:} The "complete" continuum is the limit of all Wk​, but no system with a finite number of degrees of freedom can fully capture it. This aligns with the core philosophy of finitism \cite{ElKhettabi2025AoO, SEP_Finitism}.
\item \textbf{Algebraic Consistency:} The disjointness of Wn​ and Wm​ is a concrete example of the AoO’s principle of constraint-guided differentiation \cite{ElKhettabi2025AoO}. Each degree of freedom k introduces structure only in its designated interval (k+11​,k1​).
\end{enumerate}
\section{Conclusion and Outlook}
The rigorous analysis presented here confirms that the Farey sequence hierarchy Fn​ and the infinite Stern-Brocot tree provide a direct and operational instantiation of the core principles articulated in the \emph{Arithmetic of Order} framework. By demonstrating a formal isomorphism between the Farey system’s mediant-based local refinement and the AoO’s principle of \emph{constraint-guided differentiation}, this work extends the finitistic, constructivist agenda championed in El Khettabi’s foundational reports on powerset combinatorics, hypercomplex numbers, and emergent geometries \cite{ElKhettabi2025AoO, ElKhettabi2024HCN, ElKhettabi2024PLOS}.
Crucially, the Farey system shows that the real continuum --- historically assumed as an \emph{a priori} infinite structure --- can instead be understood as the asymptotic closure of a fully discrete, algorithmically generable process \cite{SEP_Finitism, Zukin2016}. Every rational in the unit interval is positioned within a nested, well-ordered hierarchy whose generation is entirely finitistic and transparent \cite{WikipediaFarey, Zukin2016}. The Farey mediant mechanism mirrors the XOR-bitwise operation on powersets: both systems employ a local, deterministic rule to refine an initial configuration space under a global constraint parameter, here the denominator bound n \cite{WikipediaPowerset, JNSFarey}.
The duality between local non-abelian modular symmetries ($\SL(2,\mathbb{Z})$) and the global abelian powerset algebra (F2n​) reinforces the AoO’s insight that the richness of mathematical and physical structure can emerge from the simple arithmetic of ordered degrees of freedom \cite{ElKhettabi2025AoO, ConradSL2Z, BooleanRing}. Just as the Golay code G24​, the Leech lattice Λ24​, and the Mathieu group M24​ arise from sieving the powerset P(Ω24​), so too does the Farey hierarchy filter the Stern-Brocot tree into optimally ordered sets of best rational approximations \cite{ElKhettabi2025AoO, DUMMIT}.
By placing the Farey system within this finitistic combinatorial paradigm, this article not only strengthens the AoO’s theoretical claims but demonstrates its flexibility and universality across number theory, coding theory, and the modeling of continuum phenomena within strictly finite means. In this sense, the Farey hierarchy provides an explicit model of physics as quantized measurement: each mediant, each denominator constraint, each combinatorial power represents a discrete quantum of possible states. The real continuum, as the limit of these finite steps, is not a physical prerequisite but an emergent idealization of finitely resolved measurements.
\textbf{Future Directions.} This synthesis invites further exploration along several lines. First, the deep interplay between Farey adjacency, modular tessellations, and hyperbolic geometry (Ford circles, Farey tessellation of H2) deserves to be mapped explicitly onto the projective geometries naturally emerging from the ordered powerset hierarchy \cite{ElKhettabi2025AoO, WikipediaFarey, FareyGraphSymmetry}. Second, the constructive mediant process suggests algorithmic avenues for designing finite, resource-bounded AI systems capable of performing Diophantine approximations without recourse to continuum assumptions \cite{ElKhettabi2025AoO, WikipediaSternBrocot}. Finally, the formal analogy between Farey trees and the nested powerset combinatorics suggests potential generalizations to higher-dimensional hypercomplex structures and their associated finite geometries.
In unifying the Farey sequence and the Arithmetic of Order framework, we highlight a powerful theme: the apparent paradox of continuum structures in finite physical systems dissolves when seen as the limit of simple, local, finitistic rules. Mathematics, under this lens, is not an edifice built on the infinite, but a revelation of emergent order rooted in the finite --- one degree of freedom at a time.
\vspace{1em}
\noindent
\textbf{Faysal El Khettabi} \
\emph{Ensemble AIs} \
\texttt{faysal.el.khettabi@gmail.com} \
July 2025
\appendix
\section{Appendix}
\subsection{Proof of the Unimodular Relation via Pick's Theorem}
A key property of Farey sequences is that if a/b and c/d are adjacent terms (neighbors), then bc−ad=1. A geometric proof utilizes Pick's Theorem, which relates the area of a simple polygon whose vertices are points on the integer lattice to the number of integer points on its boundary and in its interior \cite{DUMMIT}. The area A is given by A=I+2B​−1, where I is the number of interior lattice points and B is the number of boundary lattice points.
Consider the triangle △ with vertices at the origin O(0,0), P1​(b,a), and P2​(d,c). The area of this triangle can be calculated using the determinant formula, which gives A=21​∣bc−ad∣. Since a/b<c/d, we have bc>ad, so the area is A=21​(bc−ad).
Now, we apply Pick's Theorem to this triangle:
\begin{itemize}
\item \textbf{Boundary Points (B):} The vertices O, P1​, and P2​ are lattice points. Since the fractions a/b and c/d are irreducible, gcdf​unc(a,b)=1 and gcdf​unc(c,d)=1. This implies there are no other lattice points on the segments OP1​ and OP2​. If there were a lattice point on the segment P1​P2​, it would represent a rational fraction that should lie between a/b and c/d in the Farey sequence but with a smaller denominator, which is impossible for neighbors. Thus, the only lattice points on the boundary are the three vertices, so B=3 \cite{DUMMIT}.
\item \textbf{Interior Points (I):} Suppose there were a lattice point (x,y) in the interior of △. The fraction y/x would have a value strictly between a/b and c/d. Since b≤n and d≤n, it would follow that x<n. This would mean that y/x is a fraction with a denominator smaller than n that lies between a/b and c/d, contradicting the assumption that they are consecutive terms in Fn​. Therefore, there can be no lattice points in the interior of the triangle, and I=0 \cite{DUMMIT}.
\end{itemize}
Applying Pick's Theorem with I=0 and B=3, the area of the triangle is A=0+23​−1=21​. Equating this with our determinant formula, we get 21​(bc−ad)=21​, which directly implies bc−ad=1. This provides a purely combinatorial confirmation that the mediant always yields an irreducible fraction when the unimodular condition is met.
\subsection{The Isomorphism (P(Ωn​),Δ)≅F2n​}
The powerset P(Ωn​) of a set Ωn​={1,2,…,n}, when equipped with the operation of symmetric difference (Δ), forms a finite abelian group \cite{WikipediaPowerset, SymmetricDifferenceGroup}.
\begin{itemize}
\item \textbf{Closure:} For any two subsets A,B⊆Ωn​, their symmetric difference AΔB=(A∪B)∖(A∩B) is also a subset of Ωn​.
\item \textbf{Associativity:} The operation is associative: (AΔB)ΔC=AΔ(BΔC).
\item \textbf{Identity Element:} The empty set ∅ serves as the identity element, as AΔ∅=A.
\item \textbf{Inverse Element:} Every element is its own inverse, as AΔA=∅.
\end{itemize}
This group is isomorphic to the n-dimensional vector space over the field of two elements, F2​={0,1}, denoted F2n​ \cite{F2nVectorSpace, BooleanRing}. The isomorphism is established by mapping each subset S⊆Ωn​ to its characteristic function (or binary vector) of length n. The symmetric difference of two subsets corresponds precisely to the component-wise addition (XOR operation) of their corresponding vectors in F2n​. This mapping makes the group operation identical to vector addition modulo 2, clarifying the isomorphism’s computational meaning. The set of singleton subsets {{1},{2},…,{n}} forms a basis for this vector space \cite{F2nVectorSpace}.
\begin{thebibliography}{99}
\bibitem{WikipediaFoundations}
Wikipedia contributors. (2024). \emph{Foundations of mathematics}. Wikipedia, The Free Encyclopedia.
\bibitem{ElKhettabi2025AoO}
El Khettabi, F. (2025). \emph{The Arithmetic of Order: A Finitistic Foundation for Mathematics, Emergent Structures, and Intelligent Systems}. viXra.
\bibitem{ElKhettabi2024HCN}
El Khettabi, F. (2024). \emph{A Comprehensive Modern Mathematical Foundation for Hypercomplex Numbers with Recollection of Sir William Rowan Hamilton, John T. Graves, and Arthur Cayley}.
\bibitem{SEP_Constructivism}
Iemhoff, R. (2023). \emph{Constructive Mathematics}. The Stanford Encyclopedia of Philosophy (Fall 2023 Edition), Edward N. Zalta & Uri Nodelman (eds.).
\bibitem{WikipediaFarey}
Wikipedia contributors. (2024). \emph{Farey sequence}. Wikipedia, The Free Encyclopedia.
\bibitem{SEP_Finitism}
Ye, F. (2021). \emph{Finitism in Geometry}. The Stanford Encyclopedia of Philosophy (Winter 2021 Edition), Edward N. Zalta (ed.).
\bibitem{Zukin2016}
Zukin, M. (2016). \emph{The Farey Sequence}. Whitman College.
\bibitem{DUMMIT}
Dummit, D. S., & Foote, R. M. (2004). \emph{Abstract Algebra}. John Wiley & Sons.
\bibitem{ElKhettabi2024PLOS}
El Khettabi, F. (2024). \emph{On the hypercomplex numbers and normed division algebra of all dimensions: A unified multiplication}. PLOS ONE, 19(6), e0312502.
\bibitem{WikipediaPowerset}
Wikipedia contributors. (2024). \emph{Power set}. Wikipedia, The Free Encyclopedia.
\bibitem{ElKhettabi2025AoO_vixra}
El Khettabi, F. (2025). The Arithmetic of Order: A Finitistic Foundation for Mathematics, Emergent Structures, and Intelligent Systems. \emph{viXra:2505.0064}.
\bibitem{JNSFarey}
Tamang, B. B., et al. (2022). Some characteristics of the Farey sequences with Ford circles. \emph{Nepal Journal of Mathematical Sciences}, 4(1), 69-76.
\bibitem{RanickiFareyProject}
Ranicki, A. (n.d.). \emph{The Farey sequence and its applications}.
\bibitem{KnottFarey}
Knott, R. \emph{Farey Series and the Stern-Brocot Tree}. University of Surrey.
\bibitem{NumberanalyticsFarey}
Number Analytics. (n.d.). \emph{Farey Sequences: A Deep Dive into Additive Number Theory}.
\bibitem{WikipediaSternBrocot}
Wikipedia contributors. (2024). \emph{Stern–Brocot tree}. Wikipedia, The Free Encyclopedia.
\bibitem{CPAlgorithmsSternBrocot}
CP-Algorithms. \emph{Stern-Brocot Tree and Farey Sequences}.
\bibitem{CutTheKnotFarey}
Bogomolny, A. \emph{Farey Series}. Cut-the-Knot.
\bibitem{CutTheKnotSternBrocot}
Bogomolny, A. \emph{Stern-Brocot Tree}. Cut-the-Knot.
\bibitem{ConradSL2Z}
Conrad, K. \emph{SL(2,Z)}. University of Connecticut.
\bibitem{WikipediaSL2Z}
Wikipedia contributors. (2024). \emph{Modular group}. Wikipedia, The Free Encyclopedia.
\bibitem{FareyGraphSymmetry}
Lutsko, C. (2021). \emph{Generalized Farey sequences}. International Mathematics Research Notices.
\bibitem{BooleanRing}
Wikipedia contributors. (2024). \emph{Boolean ring}. Wikipedia, The Free Encyclopedia.
\bibitem{F2nVectorSpace}
Stack Exchange. (2018). \emph{P(X) with symmetric difference as addition as a vector space over Z2}.
\bibitem{SymmetricDifferenceGroup}
ProofWiki. \emph{Symmetric Difference on Power Set forms Abelian Group}.
\bibitem{HardyWright}
Hardy, G. H., & Wright, E. M. (1979). \emph{An Introduction to the Theory of Numbers}. Oxford University Press.
\bibitem{ConwaySloane}
Conway, J. H., & Sloane, N. J. A. (1999). \emph{Sphere Packings, Lattices, and Groups}. Springer.
\end{thebibliography}
\end{document}


















\documentclass[12pt,a4paper]{article}% PACKAGES\usepackage[margin=1in]{geometry}\usepackage{amsmath, amssymb, amsthm}\usepackage{authblk}\usepackage[utf8]{inputenc}\usepackage{fontenc}\usepackage{hyperref}\hypersetup{colorlinks=true,linkcolor=blue,filecolor=magenta,urlcolor=cyan,pdftitle={The Farey Sequence as a Realization of the Arithmetic of Order},pdfpagemode=FullScreen,}% MATH OPERATORS AND ENVIRONMENTS\DeclareMathOperator{\im}{Im}\DeclareMathOperator{\re}{Re}\DeclareMathOperator{\SL}{SL}\DeclareMathOperator{\PSL}{PSL}\DeclareMathOperator{\GL}{GL}\DeclareMathOperator{\gcd_func}{gcd}\newtheorem{theorem}{Theorem}[section]\newtheorem{proposition}[theorem]{Proposition}\newtheorem{lemma}[theorem]{Lemma}\newtheorem{corollary}[theorem]{Corollary}\theoremstyle{definition}\newtheorem{definition}[theorem]{Definition}\newtheorem{example}[theorem]{Example}\theoremstyle{remark}\newtheorem{remark}[theorem]{Remark}% TITLE AND AUTHOR\title{Physics as Quantized Measurement: The Farey Sequence as a Realization of the Arithmetic of Order}\author{Faysal El Khettabi}\affil{Ensemble AIs\ \texttt{faysal.el.khettabi@gmail.com}}\date{July 2025}\begin{document}\maketitle\tableofcontents\begin{abstract}This article formalizes the Farey sequence hierarchy (Fn​) and the universal Stern-Brocot tree (Tn​) as not merely analogies but explicit constructive realizations of the ``Arithmetic of Order'' (AoO) framework. It synthesizes the generative principles of constraint-guided differentiation, the progression 1→n→n+1, and the nested powerset structure, demonstrating that the Farey system embodies the finitistic, emergent continuum perspective central to the AoO thesis. We propose that the Farey numbers, generated by a finite, deterministic process, provide a natural model for physical quantities and measurements, replacing the need for an a priori infinite continuum. This synthesis reinforces a unified finitistic approach to foundational mathematics, hypercomplex number theory, projective geometries, and the algorithmic design of intelligent systems.\end{abstract}\section{Introduction: A Finitistic Recasting of the Continuum}The core argument of this report is that the Farey system provides a complete and rigorous mathematical prototype for the "Arithmetic of Order" framework \cite{ElKhettabi2025AoO}. This is not a relationship of analogy, but of instantiation, where the abstract principles of the latter find their direct, operational expression in the former.The foundations of modern mathematics and physics have been shaped by a foundational crisis that emerged in the late 19th and early 20th centuries \cite{WikipediaFoundations}. A central feature of the resolution to this crisis was the widespread adoption of infinitary concepts, most notably the continuum of real and complex numbers. The imaginary unit i=−1​∈C, for instance, is now central to the formulation of quantum theory \cite{ElKhettabi2024HCN, ElKhettabi2025AoO}. Yet, this reliance introduces a profound conceptual paradox: how can a finite physical system—a quantum register, a molecule, or any object composed of a finite number of components—fundamentally require an infinite mathematical construct for its description? \cite{ElKhettabi2025AoO}.The "Arithmetic of Order" (AoO) framework directly confronts this paradox by proposing a mathematics built from the finite and observable, where complexity and structure emerge constructively \cite{ElKhettabi2025AoO}. This approach aligns with the philosophical traditions of finitism and constructivism \cite{SEP_Finitism, SEP_Constructivism}. Finitism, in its various forms, questions or rejects the existence of actual infinite objects, such as the set of all natural numbers, proposing that mathematics should be grounded in objects that are, at least in principle, finite \cite{WikipediaFarey, SEP_Finitism}. Constructivism insists that mathematical existence is tied to algorithmic constructibility; to prove an object exists, one must provide a method of finding (“constructing”) such an object \cite{SEP_Constructivism, Zukin2016}. The AoO framework synthesizes these views by critiquing the traditional reliance on a priori infinities and proposing a new foundation where the continuum itself is not a given axiom but an emergent property \cite{ElKhettabi2025AoO}.This report demonstrates that the Farey sequence system offers a perfect, non-trivial model for this finitistic philosophy. The Farey sequence, generated by a simple, finite, and deterministic algorithm, provides a concrete pathway to the rational numbers and, by extension, to the real continuum. Its generation does not presuppose the existence of the continuum but rather builds it step-by-step, deriving it as an asymptotic limit of a sequence of finite structures \cite{WikipediaFarey, Zukin2016}. This aligns perfectly with the AoO's reinterpretation of the continuum as a limit of a nested hierarchy of finite sets \cite{ElKhettabi2025AoO}.The structure of this analysis will proceed as follows. First, the abstract principles of the AoO framework will be detailed. Second, a rigorous exposition of the Farey system will be provided. The central sections will then establish the formal mapping between the generative processes and hierarchical structures of these two domains. This is followed by a deep analysis of their corresponding algebraic structures, a discussion of the hierarchical perception of the continuum, and finally, an exploration of the implications of this synthesis.\section{The "Arithmetic of Order": A Framework of Emergent Complexity}The "Arithmetic of Order" (AoO) is a foundational framework that seeks to re-establish mathematics on finite, constructive principles. It posits that the intricate structures of mathematics are not arbitrary inventions but are natural consequences of the most fundamental process of ordered progression \cite{ElKhettabi2025AoO}.\subsection{The Generative Progression 1→n→n+1}The central axiom of the AoO framework is that all of mathematics can be understood as a "natural revelation of the intrinsic structure embedded in the progression 1→n→n+1" \cite{ElKhettabi2025AoO}. This progression is not merely a representation of counting; it embodies the fundamental process of incrementally adding a new degree of freedom to a system \cite{ElKhettabi2025AoO, ElKhettabi2024HCN}. In physical terms, each step n→n+1 can be interpreted as the introduction of a new quantized degree of freedom — an additional state, bit, or mode that refines the system’s resolution of measurable quantities. For example, each new fraction in the Farey hierarchy refines an interval with finite rational steps, mirroring how physical measurement adds finite resolution at each scale. This reflects the same principle that underlies Planck’s constant in quantum mechanics: physical observables do not vary continuously in theory but are constrained by finite quanta of action and resolution. In the context of physics, a system's state is defined by a finite number of such degrees of freedom, and understanding how the system's properties change as this number increases is paramount \cite{ElKhettabi2024HCN}.The framework emphasizes that it is the ordered nature of the underlying set Ωn​={1,2,…,n} that is crucial \cite{ElKhettabi2025AoO}. This ordering ensures that the hierarchy of structures built upon it is well-defined, recursive, and nested. Each step from n to n+1 represents a deterministic expansion of the system's potential, opening up new combinatorial possibilities and enabling the emergence of higher-order symmetries and more complex structures \cite{ElKhettabi2025AoO}.\subsection{The Powerset Hierarchy P(Ωn​) as the Canonical State Space}Within the AoO, the canonical state space of a system with n degrees of freedom is identified with the powerset of Ωn​, denoted P(Ωn​) \cite{ElKhettabi2025AoO}. The powerset is the set of all subsets of Ωn​, including the empty set and Ωn​ itself \cite{WikipediaPowerset}. Each element of P(Ωn​)—that is, each subset S⊆Ωn​—represents a distinct configuration of the system. This configuration can be encoded by a characteristic function, a binary vector indicating which degrees of freedom are "active" or "present" in that state \cite{ElKhettabi2025AoO, WikipediaPowerset}.The generative progression 1→n→n+1 manifests directly as a nested hierarchy of these powerset state spaces: P(Ωn​)⊂P(Ωn+1​) \cite{ElKhettabi2024HCN}. The construction of the next level of the hierarchy is both recursive and fully deterministic. Given P(Ωn​), the powerset P(Ωn+1​) is formed by taking all existing subsets in P(Ωn​) and adding to them a new collection of subsets, each formed by taking an element of P(Ωn​) and adjoining the new element {n+1} \cite{WikipediaPowerset}. Formally, this is expressed as:The cardinality of the powerset, ∣P(Ωn​)∣=2n, grows exponentially, highlighting the rapid increase in structural complexity that arises from the addition of each new degree of freedom \cite{ElKhettabi2024HCN}.\subsection{The Continuum as an Asymptotic Limit}A cornerstone of the AoO framework is its explicit rejection of the continuum as an a priori entity \cite{ElKhettabi2025AoO}. Instead, the real continuum is reinterpreted as an asymptotic limit of the nested powerset hierarchy. As the number of degrees of freedom n tends towards infinity, the combinatorial and topological properties of the finite space P(Ωn​) approach those traditionally ascribed to continuous systems \cite{ElKhettabi2025AoO}.This perspective aligns with finitist philosophies that are skeptical of actual, completed infinities but are comfortable with the concept of potential infinity or unbounded processes \cite{WikipediaFarey, SEP_Finitism}. The framework provides a concrete model for how this limit is approached. The cardinality of the powerset, 2n, naturally approaches the cardinality of the continuum, 2ℵ0​, as n approaches the first infinite cardinal, ℵ0​. The Cantor space, which is homeomorphic to the set of infinite binary sequences {0,1}N, serves as a topological bridge, as it is in one-to-one correspondence with P(N) and has the cardinality of the continuum \cite{WikipediaPowerset}.\subsection{Constraint-Guided Differentiation and Emergent Structures}A key generative mechanism within the AoO framework is termed "constraint-guided differentiation" \cite{ElKhettabi2025AoO}. This principle posits that by applying specific combinatorial rules or constraints to the powerset hierarchy, certain "optimal" mathematical structures can emerge deterministically. Examples cited within the framework's literature include the emergence of exceptional structures like the Golay code G24​ and the Leech lattice Λ24​ from P(Ω24​) when specific constraints are applied \cite{ElKhettabi2025AoO}.\section{The Farey System: A Deterministic Path to the Rationals}The Farey sequence and its related structures, such as the Stern-Brocot tree, form a cornerstone of elementary number theory, providing a constructive method for generating the set of rational numbers.\subsection{The Farey Sequence Fn​: Definition and Hierarchical Construction}The Farey sequence of order n, denoted Fn​, is formally defined as the sequence of irreducible fractions a/b in the closed interval (0,1) such that the denominator b is a positive integer satisfying 1≤b≤n, arranged in ascending order of magnitude \cite{WikipediaFarey, Zukin2016}. Each sequence conventionally starts with 0/1 and ends with 1/1 \cite{WikipediaFarey}.The first few Farey sequences are \cite{Zukin2016, RanickiFareyProject}:\begin{itemize}\item F1​:{10​,11​}\item F2​:{10​,21​,11​}\item F3​:{10​,31​,21​,32​,11​}\item F4​:{10​,41​,31​,21​,32​,43​,11​}\end{itemize}A fundamental property of this construction is its hierarchical nature: Fn​⊂Fn+1​ for all n≥1 \cite{WikipediaFarey, Zukin2016}. The number of new fractions added at step n+1 is precisely given by Euler's totient function, ϕ(n+1) \cite{WikipediaFarey, Zukin2016}.\subsection{Mediant Refinement: The Algorithmic Engine}The generation of Farey sequences is driven by the mediant operation \cite{RanickiFareyProject, NumberanalyticsFarey}. The mediant of two fractions a/b and c/d is defined as (a+c)/(b+d). If a/b<c/d, their mediant always lies strictly between them \cite{KnottFarey, RanickiFareyProject, NumberanalyticsFarey}.A fundamental theorem states that the sequence Fn+1​ can be constructed from Fn​ by identifying all adjacent pairs of fractions a/b and c/d in Fn​ and inserting their mediant (a+c)/(b+d) between them if and only if the denominator of the mediant satisfies b+d=n+1 \cite{DUMMIT, Zukin2016, JNSFarey}. This provides a fully algorithmic and constructive method for generating the entire hierarchy \cite{Zukin2016, JNSFarey}.\subsection{The Stern-Brocot Tree: The Universal Genealogy of Rationals}The mediant operation, when applied recursively without the denominator constraint, generates the Stern-Brocot tree, an infinite binary tree that contains every positive rational number exactly once \cite{WikipediaSternBrocot, CPAlgorithmsSternBrocot}. It is constructed by starting with the "ancestors" 0/1 and 1/0 (representing 0 and infinity) and iteratively inserting the mediant between adjacent fractions \cite{WikipediaSternBrocot, CutTheKnotSternBrocot}. The Farey sequence Fn​ can be recovered by an in-order traversal of the tree, pruning any branch where a denominator exceeds n \cite{WikipediaSternBrocot, CPAlgorithmsSternBrocot}.\subsection{Foundational Properties and their Significance}\begin{itemize}\item \textbf{The Unimodular Relation:} If two fractions a/b and c/d are adjacent in any Farey sequence Fn​, they satisfy the unimodular relation: bc−ad=1 \cite{DUMMIT, Zukin2016}. This invariant guarantees that any mediant formed from two neighbors is itself irreducible \cite{Zukin2016, JNSFarey}.\item \textbf{Optimality in Diophantine Approximation:} Farey sequences contain the set of "best rational approximations" of the first kind to any real number for a given denominator bound n \cite{Zukin2016, JNSFarey}.\item \textbf{The Emergent Continuum:} The union of all Farey sequences, ⋃n=1∞​Fn​, constitutes the set of all rational numbers in $$\cite{CutTheKnotFarey}. The real continuum$$ is the topological closure of this constructively generated set \cite{Zukin2016}.\end{itemize}\section{The Isomorphism of Process: Unifying the Framework and its Realization}The parallels between the AoO framework and the Farey system are not merely superficial. The core generative processes of both systems are structurally identical.\subsection{Local Refinement as Constrained Expansion}The fundamental mechanism of evolution in both systems is a process of local refinement governed by a global constraint.\begin{itemize}\item In the Farey system, an interval (a/b,c/d) in Fn​ is refined by the mediant (a+c)/(b+d) only when the denominator constraint b+d=n+1 is met \cite{Zukin2016, JNSFarey}.\item In the powerset hierarchy, the transition from P(Ωn​) to P(Ωn+1​) introduces new subsets defined by the inclusion of the single new element {n+1} \cite{WikipediaPowerset, ElKhettabi2024HCN}.\end{itemize}The mediant insertion rule is a perfect instantiation of the AoO's abstract principle of "constraint-guided differentiation" \cite{ElKhettabi2025AoO}. The parameter n acts as a universal filter, determining which new elements are actualized at each stage.\subsection{The Arrow of Complexity in Nested Hierarchies}There is a formal mapping between the set inclusion Fn​⊂Fn+1​ and the powerset inclusion P(Ωn​)⊂P(Ωn+1​). Despite different growth rates (polynomial for ∣Fn​∣∼3n2/π2 vs. exponential for ∣P(Ωn​)∣=2n), both systems exhibit an irreversible "arrow of complexity" driven by the same underlying progression, 1→n→n+1 \cite{ElKhettabi2025AoO, WikipediaFarey, Zukin2016}.\section{A Unification of Algebraic Structures}In algebraic thinking, the “rules” are not just about solving for unknowns, but about determining the space of all possible solutions and how they interrelate. The correspondence between the Farey and powerset systems extends to the very algebraic structures they embody, revealing a profound duality.\begin{table}[h!]\centering\caption{Structural Parallels between Farey and Powerset Systems}\label{tab:duality}\begin{tabular}{|p{2.5cm}|p{4cm}|p{4cm}|p{3.5cm}|}\hline\textbf{Aspect} & \textbf{Farey/Stern-Brocot System} & \textbf{Powerset Hierarchy} & \textbf{Framework Link} \\hline\textbf{Elements} & Irreducible fractions p/q & Subsets S⊆Ωn​ & Configurations of n degrees of freedom \cite{ElKhettabi2025AoO, Zukin2016} \\hline\textbf{Ordering} & Standard numerical order < & Subset inclusion ⊂ & Hierarchical nesting \cite{ElKhettabi2025AoO, Zukin2016} \\hline\textbf{Refinement Rule} & Mediant b+da+c​ if b+d≤n+1 \cite{Zukin2016, JNSFarey} & Add subsets containing n+1 \cite{WikipediaPowerset} & Constraint-guided differentiation \cite{ElKhettabi2025AoO} \\hline\textbf{Local Algebra} & Neighbors a/b,c/d form a matrix in $\SL(2,\mathbb{Z})$ \cite{Zukin2016, DUMMIT} & --- & Structure of local interactions \\hline\textbf{Global Algebra} & --- & (P(Ωn​),Δ) is isomorphic to F2n​ \cite{WikipediaPowerset} & Algebra of system states \cite{ElKhettabi2024HCN} \\hline\textbf{Geometric View} & Ford Circles, Farey Tessellation of H2 \cite{WikipediaFarey, Zukin2016} & Projective planes/spaces over F2​ \cite{ElKhettabi2025AoO} & Emergent geometry \cite{ElKhettabi2025AoO} \\hline\textbf{Asymptotic Limit} & The real continuum (0,1) \cite{Zukin2016} & The Cantor space 2N, cardinality of continuum \cite{WikipediaPowerset} & The emergent continuum \cite{ElKhettabi2025AoO} \\hline\end{tabular}\end{table}\subsection{The Modular Group $\SL(2,\mathbb{Z})$: The Invariant Algebra of Farey Adjacency}The unimodular relation bc−ad=1 is the defining characteristic of the special linear group $\SL(2,\mathbb{Z})$ \cite{ConradSL2Z, WikipediaSL2Z}. For any pair of adjacent Farey neighbors a/b and c/d, the matrix (ab​cd​) is an element of $\SL(2,\mathbb{Z})$ \cite{Zukin2016, DUMMIT}. This group acts as the group of orientation-preserving automorphisms of the Farey graph, which tessellates the hyperbolic plane \cite{ConradSL2Z, FareyGraphSymmetry}. $\SL(2,\mathbb{Z})$ is thus the infinite, non-abelian group that governs the local, dynamic structure of the Farey sequence \cite{ConradSL2Z}.\subsection{The Vector Space F2n​: The Canonical Algebra of System Configurations}The powerset P(Ωn​), when equipped with the operation of symmetric difference (Δ), forms a finite abelian group \cite{WikipediaPowerset, BooleanRing}. This group is isomorphic to the n-dimensional vector space over the finite field of two elements, F2n​ \cite{F2nVectorSpace, BooleanRing}. This algebraic structure is presented in the AoO framework as the canonical algebra for describing the global, static set of all possible configurations of a system with n binary degrees of freedom \cite{ElKhettabi2025AoO, ElKhettabi2024HCN}.\subsection{The Duality of Local (Non-Abelian) and Global (Abelian) Algebra}The synthesis of these systems reveals a profound duality. The AoO framework, by asserting that the Farey system is a concrete instantiation of its principles, implicitly predicts this juxtaposition of the infinite, non-abelian group $\SL(2,\mathbb{Z})$ with the sequence of finite, abelian groups F2n​. This key observation can be summarized as follows: The Farey system’s local, non-abelian symmetry ($\SL(2,\mathbb{Z})$) governs the dynamic refinement of the space \cite{Zukin2016, ConradSL2Z}, while the powerset’s global, abelian symmetry (F2n​) governs the static configuration of all possible states \cite{WikipediaPowerset, BooleanRing}. This duality between dynamic refinement and static configuration is a central feature of the AoO framework's explanatory power.\section{Hierarchical Perception of the Continuum through Degrees of Freedom}The relationship between the sets of mediants generated at different stages of the Farey hierarchy captures how systems with different degrees of freedom "perceive" the continuum. This section formalizes this concept, demonstrating that each degree of freedom unlocks a distinct, non-overlapping "vision" of the emergent continuum.\subsection{Defining the "Vision" of a Degree of Freedom: The Sets Wn​ and Wm​}For a system with a given degree of freedom, represented by the integer n, we can define its specific interval of focus and the set of rational numbers it generates to refine that interval.\begin{itemize}\item \textbf{Interval of Focus for degree n:} The interval In​=(n+11​,n1​) is defined by two fractions that become Farey neighbors in the sequence Fn+1​ \cite{Zukin2016}.\item \textbf{Mediant Refinement Set Wn​:} We define Wn​ as the set of all rational numbers that recursively refine the interval In​. This set is generated by the iterative application of the mediant operation, a process identical to the construction of a local Stern-Brocot tree within that interval \cite{Zukin2016, NumberanalyticsFarey}.\begin{itemize}\item \textbf{Properties of Wn​:}\item Wn​ is \textbf{dense} in (n+11​,n1​), as the mediant operation eventually fills every gap between rational numbers \cite{DUMMIT}.\item Wn​ is \textbf{infinite} and \textbf{ordered} according to the structure of the Stern-Brocot process \cite{WikipediaSternBrocot, CPAlgorithmsSternBrocot}.\end{itemize}\end{itemize}For a system with a higher degree of freedom m>n:\begin{itemize}\item \textbf{Interval of Focus for degree m:} Im​=(m+11​,m1​).\item \textbf{Mediant Refinement Set Wm​:} Wm​ is similarly defined as the set of all mediants that recursively refine the interval Im​.\end{itemize}\begin{theorem}Let In​=(n+11​,n1​) and define Wn​ as the set of all rationals generated by repeated mediant refinement within In​. Then for any m>n+1, the sets Wn​ and Wm​ are disjoint:[W_n \cap W_m = \emptyset.]Moreover, the rational continuum in (0,1) satisfies[\bigcup_{n=1}^{\infty} W_n = \mathbb{Q} \cap (0,1),\quad \text{and} \quad\overline{\bigcup_{n=1}^{\infty} W_n} = (0,1).]\end{theorem}\begin{remark}The non-overlapping property of the Wn​ sets formalizes the notion that a physical system with n quantized degrees of freedom cannot access or resolve measurement bands defined by higher m>n+1. This mirrors how discrete energy levels, finite sampling, or Planck-scale limits constrain physical measurement in quantum theory.\end{remark}\begin{corollary}Within the Arithmetic of Order (AoO) framework, the concept of “zero” is not assumed as a primitive numerical constant but emerges constructively as a relational boundary condition. The ordered powerset hierarchy begins with the empty set {} as a symbolic holder, not a numerical zero. Its powerset P({})={{}} generates the first distinguishable container. The introduction of the first degree of freedom {1} yields P({1})={{},{1}}, whose cardinality 21=2=1+1 demonstrates that counting is rooted in nesting and ordering, not in a naked zero.In parallel, the Farey sequence never invokes an a priori continuum zero but uses 0/1 as a finite measurement anchor. Its mediant operation,[\frac{a}{b} \oplus \frac{c}{d} = \frac{a+c}{b+d},]ensures that each refinement step preserves the integer basis for measurement, approaching the continuum limit only asymptotically via the nested bands Wn​.Thus, the continuum’s “nude zero” appears solely as the relational closure of finitely constructed measurement steps: “zero” is never primitive but always contextualized by the ordered degrees of freedom that define its limit.\end{corollary}\subsection{Implications for the Arithmetic of Order Framework}This model of hierarchical, disjoint perception has significant implications for the AoO framework.\begin{enumerate}\item \textbf{Degree-of-Freedom-Dependent Reality:} The continuum is not perceived uniformly but as a hierarchy of refinements, where each "vision" is tied to the observer’s available degrees of freedom (n or m) \cite{ElKhettabi2025AoO}. A physical system with n variables cannot "see" the finer structure revealed by m>n variables.\item \textbf{Finitistic Continuum:} The "complete" continuum is the limit of all Wk​, but no system with a finite number of degrees of freedom can fully capture it. This aligns with the core philosophy of finitism \cite{ElKhettabi2025AoO, SEP_Finitism}.\item \textbf{Algebraic Consistency:} The disjointness of Wn​ and Wm​ is a concrete example of the AoO’s principle of constraint-guided differentiation \cite{ElKhettabi2025AoO}. Each degree of freedom k introduces structure only in its designated interval (k+11​,k1​).\end{enumerate}\section{Conclusion and Outlook}The rigorous analysis presented here confirms that the Farey sequence hierarchy Fn​ and the infinite Stern-Brocot tree provide a direct and operational instantiation of the core principles articulated in the \emph{Arithmetic of Order} framework. By demonstrating a formal isomorphism between the Farey system’s mediant-based local refinement and the AoO’s principle of \emph{constraint-guided differentiation}, this work extends the finitistic, constructivist agenda championed in El Khettabi’s foundational reports on powerset combinatorics, hypercomplex numbers, and emergent geometries \cite{ElKhettabi2025AoO, ElKhettabi2024HCN, ElKhettabi2024PLOS}.Crucially, the Farey system shows that the real continuum --- historically assumed as an \emph{a priori} infinite structure --- can instead be understood as the asymptotic closure of a fully discrete, algorithmically generable process \cite{SEP_Finitism, Zukin2016}. Every rational in the unit interval is positioned within a nested, well-ordered hierarchy whose generation is entirely finitistic and transparent \cite{WikipediaFarey, Zukin2016}. The Farey mediant mechanism mirrors the XOR-bitwise operation on powersets: both systems employ a local, deterministic rule to refine an initial configuration space under a global constraint parameter, here the denominator bound n \cite{WikipediaPowerset, JNSFarey}.The duality between local non-abelian modular symmetries ($\SL(2,\mathbb{Z})$) and the global abelian powerset algebra (F2n​) reinforces the AoO’s insight that the richness of mathematical and physical structure can emerge from the simple arithmetic of ordered degrees of freedom \cite{ElKhettabi2025AoO, ConradSL2Z, BooleanRing}. Just as the Golay code G24​, the Leech lattice Λ24​, and the Mathieu group M24​ arise from sieving the powerset P(Ω24​), so too does the Farey hierarchy filter the Stern-Brocot tree into optimally ordered sets of best rational approximations \cite{ElKhettabi2025AoO, DUMMIT}.By placing the Farey system within this finitistic combinatorial paradigm, this article not only strengthens the AoO’s theoretical claims but demonstrates its flexibility and universality across number theory, coding theory, and the modeling of continuum phenomena within strictly finite means. In this sense, the Farey hierarchy provides an explicit model of physics as quantized measurement: each mediant, each denominator constraint, each combinatorial power represents a discrete quantum of possible states. The real continuum, as the limit of these finite steps, is not a physical prerequisite but an emergent idealization of finitely resolved measurements.\textbf{Future Directions.} This synthesis invites further exploration along several lines. First, the deep interplay between Farey adjacency, modular tessellations, and hyperbolic geometry (Ford circles, Farey tessellation of H2) deserves to be mapped explicitly onto the projective geometries naturally emerging from the ordered powerset hierarchy \cite{ElKhettabi2025AoO, WikipediaFarey, FareyGraphSymmetry}. Second, the constructive mediant process suggests algorithmic avenues for designing finite, resource-bounded AI systems capable of performing Diophantine approximations without recourse to continuum assumptions \cite{ElKhettabi2025AoO, WikipediaSternBrocot}. Finally, the formal analogy between Farey trees and the nested powerset combinatorics suggests potential generalizations to higher-dimensional hypercomplex structures and their associated finite geometries.In unifying the Farey sequence and the Arithmetic of Order framework, we highlight a powerful theme: the apparent paradox of continuum structures in finite physical systems dissolves when seen as the limit of simple, local, finitistic rules. Mathematics, under this lens, is not an edifice built on the infinite, but a revelation of emergent order rooted in the finite --- one degree of freedom at a time.\vspace{1em}\noindent\textbf{Faysal El Khettabi} \\emph{Ensemble AIs} \\texttt{faysal.el.khettabi@gmail.com} \July 2025\appendix\section{Appendix}\subsection{Proof of the Unimodular Relation via Pick's Theorem}A key property of Farey sequences is that if a/b and c/d are adjacent terms (neighbors), then bc−ad=1. A geometric proof utilizes Pick's Theorem, which relates the area of a simple polygon whose vertices are points on the integer lattice to the number of integer points on its boundary and in its interior \cite{DUMMIT}. The area A is given by A=I+2B​−1, where I is the number of interior lattice points and B is the number of boundary lattice points.Consider the triangle △ with vertices at the origin O(0,0), P1​(b,a), and P2​(d,c). The area of this triangle can be calculated using the determinant formula, which gives A=21​∣bc−ad∣. Since a/b<c/d, we have bc>ad, so the area is A=21​(bc−ad).Now, we apply Pick's Theorem to this triangle:\begin{itemize}\item \textbf{Boundary Points (B):} The vertices O, P1​, and P2​ are lattice points. Since the fractions a/b and c/d are irreducible, gcdf​unc(a,b)=1 and gcdf​unc(c,d)=1. This implies there are no other lattice points on the segments OP1​ and OP2​. If there were a lattice point on the segment P1​P2​, it would represent a rational fraction that should lie between a/b and c/d in the Farey sequence but with a smaller denominator, which is impossible for neighbors. Thus, the only lattice points on the boundary are the three vertices, so B=3 \cite{DUMMIT}.\item \textbf{Interior Points (I):} Suppose there were a lattice point (x,y) in the interior of △. The fraction y/x would have a value strictly between a/b and c/d. Since b≤n and d≤n, it would follow that x<n. This would mean that y/x is a fraction with a denominator smaller than n that lies between a/b and c/d, contradicting the assumption that they are consecutive terms in Fn​. Therefore, there can be no lattice points in the interior of the triangle, and I=0 \cite{DUMMIT}.\end{itemize}Applying Pick's Theorem with I=0 and B=3, the area of the triangle is A=0+23​−1=21​. Equating this with our determinant formula, we get 21​(bc−ad)=21​, which directly implies bc−ad=1. This provides a purely combinatorial confirmation that the mediant always yields an irreducible fraction when the unimodular condition is met.\subsection{The Isomorphism (P(Ωn​),Δ)≅F2n​}The powerset P(Ωn​) of a set Ωn​={1,2,…,n}, when equipped with the operation of symmetric difference (Δ), forms a finite abelian group \cite{WikipediaPowerset, SymmetricDifferenceGroup}.\begin{itemize}\item \textbf{Closure:} For any two subsets A,B⊆Ωn​, their symmetric difference AΔB=(A∪B)∖(A∩B) is also a subset of Ωn​.\item \textbf{Associativity:} The operation is associative: (AΔB)ΔC=AΔ(BΔC).\item \textbf{Identity Element:} The empty set ∅ serves as the identity element, as AΔ∅=A.\item \textbf{Inverse Element:} Every element is its own inverse, as AΔA=∅.\end{itemize}This group is isomorphic to the n-dimensional vector space over the field of two elements, F2​={0,1}, denoted F2n​ \cite{F2nVectorSpace, BooleanRing}. The isomorphism is established by mapping each subset S⊆Ωn​ to its characteristic function (or binary vector) of length n. The symmetric difference of two subsets corresponds precisely to the component-wise addition (XOR operation) of their corresponding vectors in F2n​. This mapping makes the group operation identical to vector addition modulo 2, clarifying the isomorphism’s computational meaning. The set of singleton subsets {{1},{2},…,{n}} forms a basis for this vector space \cite{F2nVectorSpace}.\begin{thebibliography}{99}\bibitem{WikipediaFoundations}Wikipedia contributors. (2024). \emph{Foundations of mathematics}. Wikipedia, The Free Encyclopedia.\bibitem{ElKhettabi2025AoO}El Khettabi, F. (2025). \emph{The Arithmetic of Order: A Finitistic Foundation for Mathematics, Emergent Structures, and Intelligent Systems}. viXra.\bibitem{ElKhettabi2024HCN}El Khettabi, F. (2024). \emph{A Comprehensive Modern Mathematical Foundation for Hypercomplex Numbers with Recollection of Sir William Rowan Hamilton, John T. Graves, and Arthur Cayley}.\bibitem{SEP_Constructivism}Iemhoff, R. (2023). \emph{Constructive Mathematics}. The Stanford Encyclopedia of Philosophy (Fall 2023 Edition), Edward N. Zalta & Uri Nodelman (eds.).\bibitem{WikipediaFarey}Wikipedia contributors. (2024). \emph{Farey sequence}. Wikipedia, The Free Encyclopedia.\bibitem{SEP_Finitism}Ye, F. (2021). \emph{Finitism in Geometry}. The Stanford Encyclopedia of Philosophy (Winter 2021 Edition), Edward N. Zalta (ed.).\bibitem{Zukin2016}Zukin, M. (2016). \emph{The Farey Sequence}. Whitman College.\bibitem{DUMMIT}Dummit, D. S., & Foote, R. M. (2004). \emph{Abstract Algebra}. John Wiley & Sons.\bibitem{ElKhettabi2024PLOS}El Khettabi, F. (2024). \emph{On the hypercomplex numbers and normed division algebra of all dimensions: A unified multiplication}. PLOS ONE, 19(6), e0312502.\bibitem{WikipediaPowerset}Wikipedia contributors. (2024). \emph{Power set}. Wikipedia, The Free Encyclopedia.\bibitem{ElKhettabi2025AoO_vixra}El Khettabi, F. (2025). The Arithmetic of Order: A Finitistic Foundation for Mathematics, Emergent Structures, and Intelligent Systems. \emph{viXra:2505.0064}.\bibitem{JNSFarey}Tamang, B. B., et al. (2022). Some characteristics of the Farey sequences with Ford circles. \emph{Nepal Journal of Mathematical Sciences}, 4(1), 69-76.\bibitem{RanickiFareyProject}Ranicki, A. (n.d.). \emph{The Farey sequence and its applications}.\bibitem{KnottFarey}Knott, R. \emph{Farey Series and the Stern-Brocot Tree}. University of Surrey.\bibitem{NumberanalyticsFarey}Number Analytics. (n.d.). \emph{Farey Sequences: A Deep Dive into Additive Number Theory}.\bibitem{WikipediaSternBrocot}Wikipedia contributors. (2024). \emph{Stern–Brocot tree}. Wikipedia, The Free Encyclopedia.\bibitem{CPAlgorithmsSternBrocot}CP-Algorithms. \emph{Stern-Brocot Tree and Farey Sequences}.\bibitem{CutTheKnotFarey}Bogomolny, A. \emph{Farey Series}. Cut-the-Knot.\bibitem{CutTheKnotSternBrocot}Bogomolny, A. \emph{Stern-Brocot Tree}. Cut-the-Knot.\bibitem{ConradSL2Z}Conrad, K. \emph{SL(2,Z)}. University of Connecticut.\bibitem{WikipediaSL2Z}Wikipedia contributors. (2024). \emph{Modular group}. Wikipedia, The Free Encyclopedia.\bibitem{FareyGraphSymmetry}Lutsko, C. (2021). \emph{Generalized Farey sequences}. International Mathematics Research Notices.\bibitem{BooleanRing}Wikipedia contributors. (2024). \emph{Boolean ring}. Wikipedia, The Free Encyclopedia.\bibitem{F2nVectorSpace}Stack Exchange. (2018). \emph{P(X) with symmetric difference as addition as a vector space over Z2}.\bibitem{SymmetricDifferenceGroup}ProofWiki. \emph{Symmetric Difference on Power Set forms Abelian Group}.\bibitem{HardyWright}Hardy, G. H., & Wright, E. M. (1979). \emph{An Introduction to the Theory of Numbers}. Oxford University Press.\bibitem{ConwaySloane}Conway, J. H., & Sloane, N. J. A. (1999). \emph{Sphere Packings, Lattices, and Groups}. Springer.\end{thebibliography}\end{document}










\documentclass[12pt,a4paper]{article}% PACKAGES\usepackage[margin=1in]{geometry}\usepackage{amsmath, amssymb, amsthm}\usepackage{authblk}\usepackage[utf8]{inputenc}\usepackage{fontenc}\usepackage{hyperref}\hypersetup{colorlinks=true,linkcolor=blue,filecolor=magenta,urlcolor=cyan,pdftitle={The Farey Sequence as a Realization of the Arithmetic of Order},pdfpagemode=FullScreen,}% MATH OPERATORS AND ENVIRONMENTS\DeclareMathOperator{\im}{Im}\DeclareMathOperator{\re}{Re}\DeclareMathOperator{\SL}{SL}\DeclareMathOperator{\PSL}{PSL}\DeclareMathOperator{\GL}{GL}\DeclareMathOperator{\gcd_func}{gcd}\newtheorem{theorem}{Theorem}[section]\newtheorem{proposition}[theorem]{Proposition}\newtheorem{lemma}[theorem]{Lemma}\newtheorem{corollary}[theorem]{Corollary}\theoremstyle{definition}\newtheorem{definition}[theorem]{Definition}\newtheorem{example}[theorem]{Example}\theoremstyle{remark}\newtheorem{remark}[theorem]{Remark}% TITLE AND AUTHOR\title{Physics as Quantized Measurement: The Farey Sequence as a Realization of the Arithmetic of Order}\author{Faysal El Khettabi}\affil{Ensemble AIs\ \texttt{faysal.el.khettabi@gmail.com}}\date{July 2025}\begin{document}\maketitle\tableofcontents\begin{abstract}This article formalizes the Farey sequence hierarchy (Fn​) and the universal Stern-Brocot tree (Tn​) as not merely analogies but explicit constructive realizations of the ``Arithmetic of Order'' (AoO) framework. It synthesizes the generative principles of constraint-guided differentiation, the progression 1→n→n+1, and the nested powerset structure, demonstrating that the Farey system embodies the finitistic, emergent continuum perspective central to the AoO thesis. We propose that the Farey numbers, generated by a finite, deterministic process, provide a natural model for physical quantities and measurements, replacing the need for an a priori infinite continuum. This synthesis reinforces a unified finitistic approach to foundational mathematics, hypercomplex number theory, projective geometries, and the algorithmic design of intelligent systems.\end{abstract}\section{Introduction: A Finitistic Recasting of the Continuum}The core argument of this report is that the Farey system provides a complete and rigorous mathematical prototype for the "Arithmetic of Order" framework \cite{ElKhettabi2025AoO}. This is not a relationship of analogy, but of instantiation, where the abstract principles of the latter find their direct, operational expression in the former.The foundations of modern mathematics and physics have been shaped by a foundational crisis that emerged in the late 19th and early 20th centuries \cite{WikipediaFoundations}. A central feature of the resolution to this crisis was the widespread adoption of infinitary concepts, most notably the continuum of real and complex numbers. The imaginary unit i=−1​∈C, for instance, is now central to the formulation of quantum theory \cite{ElKhettabi2024HCN, ElKhettabi2025AoO}. Yet, this reliance introduces a profound conceptual paradox: how can a finite physical system—a quantum register, a molecule, or any object composed of a finite number of components—fundamentally require an infinite mathematical construct for its description? \cite{ElKhettabi2025AoO}.The "Arithmetic of Order" (AoO) framework directly confronts this paradox by proposing a mathematics built from the finite and observable, where complexity and structure emerge constructively \cite{ElKhettabi2025AoO}. This approach aligns with the philosophical traditions of finitism and constructivism \cite{SEP_Finitism, SEP_Constructivism}. Finitism, in its various forms, questions or rejects the existence of actual infinite objects, such as the set of all natural numbers, proposing that mathematics should be grounded in objects that are, at least in principle, finite \cite{WikipediaFarey, SEP_Finitism}. Constructivism insists that mathematical existence is tied to algorithmic constructibility; to prove an object exists, one must provide a method of finding (“constructing”) such an object \cite{SEP_Constructivism, Zukin2016}. The AoO framework synthesizes these views by critiquing the traditional reliance on a priori infinities and proposing a new foundation where the continuum itself is not a given axiom but an emergent property \cite{ElKhettabi2025AoO}.This report demonstrates that the Farey sequence system offers a perfect, non-trivial model for this finitistic philosophy. The Farey sequence, generated by a simple, finite, and deterministic algorithm, provides a concrete pathway to the rational numbers and, by extension, to the real continuum. Its generation does not presuppose the existence of the continuum but rather builds it step-by-step, deriving it as an asymptotic limit of a sequence of finite structures \cite{WikipediaFarey, Zukin2016}. This aligns perfectly with the AoO's reinterpretation of the continuum as a limit of a nested hierarchy of finite sets \cite{ElKhettabi2025AoO}.The structure of this analysis will proceed as follows. First, the abstract principles of the AoO framework will be detailed. Second, a rigorous exposition of the Farey system will be provided. The central sections will then establish the formal mapping between the generative processes and hierarchical structures of these two domains. This is followed by a deep analysis of their corresponding algebraic structures, a discussion of the hierarchical perception of the continuum, and finally, an exploration of the implications of this synthesis.\section{The "Arithmetic of Order": A Framework of Emergent Complexity}The "Arithmetic of Order" (AoO) is a foundational framework that seeks to re-establish mathematics on finite, constructive principles. It posits that the intricate structures of mathematics are not arbitrary inventions but are natural consequences of the most fundamental process of ordered progression \cite{ElKhettabi2025AoO}.\subsection{The Generative Progression 1→n→n+1}The central axiom of the AoO framework is that all of mathematics can be understood as a "natural revelation of the intrinsic structure embedded in the progression 1→n→n+1" \cite{ElKhettabi2025AoO}. This progression is not merely a representation of counting; it embodies the fundamental process of incrementally adding a new degree of freedom to a system \cite{ElKhettabi2025AoO, ElKhettabi2024HCN}. In physical terms, each step n→n+1 can be interpreted as the introduction of a new quantized degree of freedom — an additional state, bit, or mode that refines the system’s resolution of measurable quantities. For example, each new fraction in the Farey hierarchy refines an interval with finite rational steps, mirroring how physical measurement adds finite resolution at each scale. This reflects the same principle that underlies Planck’s constant in quantum mechanics: physical observables do not vary continuously in theory but are constrained by finite quanta of action and resolution. In the context of physics, a system's state is defined by a finite number of such degrees of freedom, and understanding how the system's properties change as this number increases is paramount \cite{ElKhettabi2024HCN}.The framework emphasizes that it is the ordered nature of the underlying set Ωn​={1,2,…,n} that is crucial \cite{ElKhettabi2025AoO}. This ordering ensures that the hierarchy of structures built upon it is well-defined, recursive, and nested. Each step from n to n+1 represents a deterministic expansion of the system's potential, opening up new combinatorial possibilities and enabling the emergence of higher-order symmetries and more complex structures \cite{ElKhettabi2025AoO}.\subsection{The Powerset Hierarchy P(Ωn​) as the Canonical State Space}Within the AoO, the canonical state space of a system with n degrees of freedom is identified with the powerset of Ωn​, denoted P(Ωn​) \cite{ElKhettabi2025AoO}. The powerset is the set of all subsets of Ωn​, including the empty set and Ωn​ itself \cite{WikipediaPowerset}. Each element of P(Ωn​)—that is, each subset S⊆Ωn​—represents a distinct configuration of the system. This configuration can be encoded by a characteristic function, a binary vector indicating which degrees of freedom are "active" or "present" in that state \cite{ElKhettabi2025AoO, WikipediaPowerset}.The generative progression 1→n→n+1 manifests directly as a nested hierarchy of these powerset state spaces: P(Ωn​)⊂P(Ωn+1​) \cite{ElKhettabi2024HCN}. The construction of the next level of the hierarchy is both recursive and fully deterministic. Given P(Ωn​), the powerset P(Ωn+1​) is formed by taking all existing subsets in P(Ωn​) and adding to them a new collection of subsets, each formed by taking an element of P(Ωn​) and adjoining the new element {n+1} \cite{WikipediaPowerset}. Formally, this is expressed as:The cardinality of the powerset, ∣P(Ωn​)∣=2n, grows exponentially, highlighting the rapid increase in structural complexity that arises from the addition of each new degree of freedom \cite{ElKhettabi2024HCN}.\subsection{The Continuum as an Asymptotic Limit}A cornerstone of the AoO framework is its explicit rejection of the continuum as an a priori entity \cite{ElKhettabi2025AoO}. Instead, the real continuum is reinterpreted as an asymptotic limit of the nested powerset hierarchy. As the number of degrees of freedom n tends towards infinity, the combinatorial and topological properties of the finite space P(Ωn​) approach those traditionally ascribed to continuous systems \cite{ElKhettabi2025AoO}.This perspective aligns with finitist philosophies that are skeptical of actual, completed infinities but are comfortable with the concept of potential infinity or unbounded processes \cite{WikipediaFarey, SEP_Finitism}. The framework provides a concrete model for how this limit is approached. The cardinality of the powerset, 2n, naturally approaches the cardinality of the continuum, 2ℵ0​, as n approaches the first infinite cardinal, ℵ0​. The Cantor space, which is homeomorphic to the set of infinite binary sequences {0,1}N, serves as a topological bridge, as it is in one-to-one correspondence with P(N) and has the cardinality of the continuum \cite{WikipediaPowerset}.\subsection{Constraint-Guided Differentiation and Emergent Structures}A key generative mechanism within the AoO framework is termed "constraint-guided differentiation" \cite{ElKhettabi2025AoO}. This principle posits that by applying specific combinatorial rules or constraints to the powerset hierarchy, certain "optimal" mathematical structures can emerge deterministically. Examples cited within the framework's literature include the emergence of exceptional structures like the Golay code G24​ and the Leech lattice Λ24​ from P(Ω24​) when specific constraints are applied \cite{ElKhettabi2025AoO}.\section{The Farey System: A Deterministic Path to the Rationals}The Farey sequence and its related structures, such as the Stern-Brocot tree, form a cornerstone of elementary number theory, providing a constructive method for generating the set of rational numbers.\subsection{The Farey Sequence Fn​: Definition and Hierarchical Construction}The Farey sequence of order n, denoted Fn​, is formally defined as the sequence of irreducible fractions a/b in the closed interval (0,1) such that the denominator b is a positive integer satisfying 1≤b≤n, arranged in ascending order of magnitude \cite{WikipediaFarey, Zukin2016}. Each sequence conventionally starts with 0/1 and ends with 1/1 \cite{WikipediaFarey}.The first few Farey sequences are \cite{Zukin2016, RanickiFareyProject}:\begin{itemize}\item F1​:{10​,11​}\item F2​:{10​,21​,11​}\item F3​:{10​,31​,21​,32​,11​}\item F4​:{10​,41​,31​,21​,32​,43​,11​}\end{itemize}A fundamental property of this construction is its hierarchical nature: Fn​⊂Fn+1​ for all n≥1 \cite{WikipediaFarey, Zukin2016}. The number of new fractions added at step n+1 is precisely given by Euler's totient function, ϕ(n+1) \cite{WikipediaFarey, Zukin2016}.\subsection{Mediant Refinement: The Algorithmic Engine}The generation of Farey sequences is driven by the mediant operation \cite{RanickiFareyProject, NumberanalyticsFarey}. The mediant of two fractions a/b and c/d is defined as (a+c)/(b+d). If a/b<c/d, their mediant always lies strictly between them \cite{KnottFarey, RanickiFareyProject, NumberanalyticsFarey}.A fundamental theorem states that the sequence Fn+1​ can be constructed from Fn​ by identifying all adjacent pairs of fractions a/b and c/d in Fn​ and inserting their mediant (a+c)/(b+d) between them if and only if the denominator of the mediant satisfies b+d=n+1 \cite{DUMMIT, Zukin2016, JNSFarey}. This provides a fully algorithmic and constructive method for generating the entire hierarchy \cite{Zukin2016, JNSFarey}.\subsection{The Stern-Brocot Tree: The Universal Genealogy of Rationals}The mediant operation, when applied recursively without the denominator constraint, generates the Stern-Brocot tree, an infinite binary tree that contains every positive rational number exactly once \cite{WikipediaSternBrocot, CPAlgorithmsSternBrocot}. It is constructed by starting with the "ancestors" 0/1 and 1/0 (representing 0 and infinity) and iteratively inserting the mediant between adjacent fractions \cite{WikipediaSternBrocot, CutTheKnotSternBrocot}. The Farey sequence Fn​ can be recovered by an in-order traversal of the tree, pruning any branch where a denominator exceeds n \cite{WikipediaSternBrocot, CPAlgorithmsSternBrocot}.\subsection{Foundational Properties and their Significance}\begin{itemize}\item \textbf{The Unimodular Relation:} If two fractions a/b and c/d are adjacent in any Farey sequence Fn​, they satisfy the unimodular relation: bc−ad=1 \cite{DUMMIT, Zukin2016}. This invariant guarantees that any mediant formed from two neighbors is itself irreducible \cite{Zukin2016, JNSFarey}.\item \textbf{Optimality in Diophantine Approximation:} Farey sequences contain the set of "best rational approximations" of the first kind to any real number for a given denominator bound n \cite{Zukin2016, JNSFarey}.\item \textbf{The Emergent Continuum:} The union of all Farey sequences, ⋃n=1∞​Fn​, constitutes the set of all rational numbers in $$\cite{CutTheKnotFarey}. The real continuum$$ is the topological closure of this constructively generated set \cite{Zukin2016}.\end{itemize}\section{The Isomorphism of Process: Unifying the Framework and its Realization}The parallels between the AoO framework and the Farey system are not merely superficial. The core generative processes of both systems are structurally identical.\subsection{Local Refinement as Constrained Expansion}The fundamental mechanism of evolution in both systems is a process of local refinement governed by a global constraint.\begin{itemize}\item In the Farey system, an interval (a/b,c/d) in Fn​ is refined by the mediant (a+c)/(b+d) only when the denominator constraint b+d=n+1 is met \cite{Zukin2016, JNSFarey}.\item In the powerset hierarchy, the transition from P(Ωn​) to P(Ωn+1​) introduces new subsets defined by the inclusion of the single new element {n+1} \cite{WikipediaPowerset, ElKhettabi2024HCN}.\end{itemize}The mediant insertion rule is a perfect instantiation of the AoO's abstract principle of "constraint-guided differentiation" \cite{ElKhettabi2025AoO}. The parameter n acts as a universal filter, determining which new elements are actualized at each stage.\subsection{The Arrow of Complexity in Nested Hierarchies}There is a formal mapping between the set inclusion Fn​⊂Fn+1​ and the powerset inclusion P(Ωn​)⊂P(Ωn+1​). Despite different growth rates (polynomial for ∣Fn​∣∼3n2/π2 vs. exponential for ∣P(Ωn​)∣=2n), both systems exhibit an irreversible "arrow of complexity" driven by the same underlying progression, 1→n→n+1 \cite{ElKhettabi2025AoO, WikipediaFarey, Zukin2016}.\section{A Unification of Algebraic Structures}In algebraic thinking, the “rules” are not just about solving for unknowns, but about determining the space of all possible solutions and how they interrelate. The correspondence between the Farey and powerset systems extends to the very algebraic structures they embody, revealing a profound duality.\begin{table}[h!]\centering\caption{Structural Parallels between Farey and Powerset Systems}\label{tab:duality}\begin{tabular}{|p{2.5cm}|p{4cm}|p{4cm}|p{3.5cm}|}\hline\textbf{Aspect} & \textbf{Farey/Stern-Brocot System} & \textbf{Powerset Hierarchy} & \textbf{Framework Link} \\hline\textbf{Elements} & Irreducible fractions p/q & Subsets S⊆Ωn​ & Configurations of n degrees of freedom \cite{ElKhettabi2025AoO, Zukin2016} \\hline\textbf{Ordering} & Standard numerical order < & Subset inclusion ⊂ & Hierarchical nesting \cite{ElKhettabi2025AoO, Zukin2016} \\hline\textbf{Refinement Rule} & Mediant b+da+c​ if b+d≤n+1 \cite{Zukin2016, JNSFarey} & Add subsets containing n+1 \cite{WikipediaPowerset} & Constraint-guided differentiation \cite{ElKhettabi2025AoO} \\hline\textbf{Local Algebra} & Neighbors a/b,c/d form a matrix in $\SL(2,\mathbb{Z})$ \cite{Zukin2016, DUMMIT} & --- & Structure of local interactions \\hline\textbf{Global Algebra} & --- & (P(Ωn​),Δ) is isomorphic to F2n​ \cite{WikipediaPowerset} & Algebra of system states \cite{ElKhettabi2024HCN} \\hline\textbf{Geometric View} & Ford Circles, Farey Tessellation of H2 \cite{WikipediaFarey, Zukin2016} & Projective planes/spaces over F2​ \cite{ElKhettabi2025AoO} & Emergent geometry \cite{ElKhettabi2025AoO} \\hline\textbf{Asymptotic Limit} & The real continuum (0,1) \cite{Zukin2016} & The Cantor space 2N, cardinality of continuum \cite{WikipediaPowerset} & The emergent continuum \cite{ElKhettabi2025AoO} \\hline\end{tabular}\end{table}\subsection{The Modular Group $\SL(2,\mathbb{Z})$: The Invariant Algebra of Farey Adjacency}The unimodular relation bc−ad=1 is the defining characteristic of the special linear group $\SL(2,\mathbb{Z})$ \cite{ConradSL2Z, WikipediaSL2Z}. For any pair of adjacent Farey neighbors a/b and c/d, the matrix (ab​cd​) is an element of $\SL(2,\mathbb{Z})$ \cite{Zukin2016, DUMMIT}. This group acts as the group of orientation-preserving automorphisms of the Farey graph, which tessellates the hyperbolic plane \cite{ConradSL2Z, FareyGraphSymmetry}. $\SL(2,\mathbb{Z})$ is thus the infinite, non-abelian group that governs the local, dynamic structure of the Farey sequence \cite{ConradSL2Z}.\subsection{The Vector Space F2n​: The Canonical Algebra of System Configurations}The powerset P(Ωn​), when equipped with the operation of symmetric difference (Δ), forms a finite abelian group \cite{WikipediaPowerset, BooleanRing}. This group is isomorphic to the n-dimensional vector space over the finite field of two elements, F2n​ \cite{F2nVectorSpace, BooleanRing}. This algebraic structure is presented in the AoO framework as the canonical algebra for describing the global, static set of all possible configurations of a system with n binary degrees of freedom \cite{ElKhettabi2025AoO, ElKhettabi2024HCN}.\subsection{The Duality of Local (Non-Abelian) and Global (Abelian) Algebra}The synthesis of these systems reveals a profound duality. The AoO framework, by asserting that the Farey system is a concrete instantiation of its principles, implicitly predicts this juxtaposition of the infinite, non-abelian group $\SL(2,\mathbb{Z})$ with the sequence of finite, abelian groups F2n​. This key observation can be summarized as follows: The Farey system’s local, non-abelian symmetry ($\SL(2,\mathbb{Z})$) governs the dynamic refinement of the space \cite{Zukin2016, ConradSL2Z}, while the powerset’s global, abelian symmetry (F2n​) governs the static configuration of all possible states \cite{WikipediaPowerset, BooleanRing}. This duality between dynamic refinement and static configuration is a central feature of the AoO framework's explanatory power.\section{Hierarchical Perception of the Continuum through Degrees of Freedom}The relationship between the sets of mediants generated at different stages of the Farey hierarchy captures how systems with different degrees of freedom "perceive" the continuum. This section formalizes this concept, demonstrating that each degree of freedom unlocks a distinct, non-overlapping "vision" of the emergent continuum.\subsection{Defining the "Vision" of a Degree of Freedom: The Sets Wn​ and Wm​}For a system with a given degree of freedom, represented by the integer n, we can define its specific interval of focus and the set of rational numbers it generates to refine that interval.\begin{itemize}\item \textbf{Interval of Focus for degree n:} The interval In​=(n+11​,n1​) is defined by two fractions that become Farey neighbors in the sequence Fn+1​ \cite{Zukin2016}.\item \textbf{Mediant Refinement Set Wn​:} We define Wn​ as the set of all rational numbers that recursively refine the interval In​. This set is generated by the iterative application of the mediant operation, a process identical to the construction of a local Stern-Brocot tree within that interval \cite{Zukin2016, NumberanalyticsFarey}.\begin{itemize}\item \textbf{Properties of Wn​:}\item Wn​ is \textbf{dense} in (n+11​,n1​), as the mediant operation eventually fills every gap between rational numbers \cite{DUMMIT}.\item Wn​ is \textbf{infinite} and \textbf{ordered} according to the structure of the Stern-Brocot process \cite{WikipediaSternBrocot, CPAlgorithmsSternBrocot}.\end{itemize}\end{itemize}For a system with a higher degree of freedom m>n:\begin{itemize}\item \textbf{Interval of Focus for degree m:} Im​=(m+11​,m1​).\item \textbf{Mediant Refinement Set Wm​:} Wm​ is similarly defined as the set of all mediants that recursively refine the interval Im​.\end{itemize}\begin{theorem}Let In​=(n+11​,n1​) and define Wn​ as the set of all rationals generated by repeated mediant refinement within In​. Then for any m>n+1, the sets Wn​ and Wm​ are disjoint:[W_n \cap W_m = \emptyset.]Moreover, the rational continuum in (0,1) satisfies[\bigcup_{n=1}^{\infty} W_n = \mathbb{Q} \cap (0,1),\quad \text{and} \quad\overline{\bigcup_{n=1}^{\infty} W_n} = (0,1).]\end{theorem}\begin{remark}The non-overlapping property of the Wn​ sets formalizes the notion that a physical system with n quantized degrees of freedom cannot access or resolve measurement bands defined by higher m>n+1. This mirrors how discrete energy levels, finite sampling, or Planck-scale limits constrain physical measurement in quantum theory.\end{remark}\begin{corollary}Within the Arithmetic of Order (AoO) framework, the concept of ``zero'' is not assumed as a primitive numerical constant but emerges constructively as a relational boundary condition. The ordered powerset hierarchy begins with the empty set {} as a symbolic holder, not a numerical zero. Its powerset P({})={{}} generates the first distinguishable container. The introduction of the first degree of freedom {1} yields P({1})={{},{1}}, whose cardinality 21=2=1+1 demonstrates that counting is rooted in nesting and ordering, not in a naked zero.In parallel, the Farey sequence never invokes an a priori continuum zero but uses 0/1 as a finite measurement anchor. Its mediant operation,[\frac{a}{b} \oplus \frac{c}{d} = \frac{a+c}{b+d},]ensures that each refinement step preserves the integer basis for measurement, approaching the continuum limit only asymptotically via the nested bands Wn​.Thus, the continuum's nude zero'' appears solely as the relational closure of finitely constructed measurement steps: zero'' is never primitive but always contextualized by the ordered degrees of freedom that define its limit.\end{corollary}\subsection{Implications for the Arithmetic of Order Framework}This model of hierarchical, disjoint perception has significant implications for the AoO framework.\begin{enumerate}\item \textbf{Degree-of-Freedom-Dependent Reality:} The continuum is not perceived uniformly but as a hierarchy of refinements, where each "vision" is tied to the observer’s available degrees of freedom (n or m) \cite{ElKhettabi2025AoO}. A physical system with n variables cannot "see" the finer structure revealed by m>n variables.\item \textbf{Finitistic Continuum:} The "complete" continuum is the limit of all Wk​, but no system with a finite number of degrees of freedom can fully capture it. This aligns with the core philosophy of finitism \cite{ElKhettabi2025AoO, SEP_Finitism}.\item \textbf{Algebraic Consistency:} The disjointness of Wn​ and Wm​ is a concrete example of the AoO’s principle of constraint-guided differentiation \cite{ElKhettabi2025AoO}. Each degree of freedom k introduces structure only in its designated interval (k+11​,k1​).\end{enumerate}\section{Conclusion and Outlook}The rigorous analysis presented here confirms that the Farey sequence hierarchy Fn​ and the infinite Stern-Brocot tree provide a direct and operational instantiation of the core principles articulated in the \emph{Arithmetic of Order} framework. By demonstrating a formal isomorphism between the Farey system’s mediant-based local refinement and the AoO’s principle of \emph{constraint-guided differentiation}, this work extends the finitistic, constructivist agenda championed in El Khettabi’s foundational reports on powerset combinatorics, hypercomplex numbers, and emergent geometries \cite{ElKhettabi2025AoO, ElKhettabi2024HCN, ElKhettabi2024PLOS}.Crucially, the Farey system shows that the real continuum --- historically assumed as an \emph{a priori} infinite structure --- can instead be understood as the asymptotic closure of a fully discrete, algorithmically generable process \cite{SEP_Finitism, Zukin2016}. Every rational in the unit interval is positioned within a nested, well-ordered hierarchy whose generation is entirely finitistic and transparent \cite{WikipediaFarey, Zukin2016}. The Farey mediant mechanism mirrors the XOR-bitwise operation on powersets: both systems employ a local, deterministic rule to refine an initial configuration space under a global constraint parameter, here the denominator bound n \cite{WikipediaPowerset, JNSFarey}.The duality between local non-abelian modular symmetries ($\SL(2,\mathbb{Z})$) and the global abelian powerset algebra (F2n​) reinforces the AoO’s insight that the richness of mathematical and physical structure can emerge from the simple arithmetic of ordered degrees of freedom \cite{ElKhettabi2025AoO, ConradSL2Z, BooleanRing}. Just as the Golay code G24​, the Leech lattice Λ24​, and the Mathieu group M24​ arise from sieving the powerset P(Ω24​), so too does the Farey hierarchy filter the Stern-Brocot tree into optimally ordered sets of best rational approximations \cite{ElKhettabi2025AoO, DUMMIT}.By placing the Farey system within this finitistic combinatorial paradigm, this article not only strengthens the AoO’s theoretical claims but demonstrates its flexibility and universality across number theory, coding theory, and the modeling of continuum phenomena within strictly finite means. In this sense, the Farey hierarchy provides an explicit model of physics as quantized measurement: each mediant, each denominator constraint, each combinatorial power represents a discrete quantum of possible states. The real continuum, as the limit of these finite steps, is not a physical prerequisite but an emergent idealization of finitely resolved measurements.\textbf{Future Directions.} This synthesis invites further exploration along several lines. First, the deep interplay between Farey adjacency, modular tessellations, and hyperbolic geometry (Ford circles, Farey tessellation of H2) deserves to be mapped explicitly onto the projective geometries naturally emerging from the ordered powerset hierarchy \cite{ElKhettabi2025AoO, WikipediaFarey, FareyGraphSymmetry}. Second, the constructive mediant process suggests algorithmic avenues for designing finite, resource-bounded AI systems capable of performing Diophantine approximations without recourse to continuum assumptions \cite{ElKhettabi2025AoO, WikipediaSternBrocot}. Finally, the formal analogy between Farey trees and the nested powerset combinatorics suggests potential generalizations to higher-dimensional hypercomplex structures and their associated finite geometries.In unifying the Farey sequence and the Arithmetic of Order framework, we highlight a powerful theme: the apparent paradox of continuum structures in finite physical systems dissolves when seen as the limit of simple, local, finitistic rules. Mathematics, under this lens, is not an edifice built on the infinite, but a revelation of emergent order rooted in the finite --- one degree of freedom at a time.\vspace{1em}\noindent\textbf{Faysal El Khettabi} \\emph{Ensemble AIs} \\texttt{faysal.el.khettabi@gmail.com} \July 2025\appendix\section{Appendix}\subsection{Proof of the Unimodular Relation via Pick's Theorem}A key property of Farey sequences is that if a/b and c/d are adjacent terms (neighbors), then bc−ad=1. A geometric proof utilizes Pick's Theorem, which relates the area of a simple polygon whose vertices are points on the integer lattice to the number of integer points on its boundary and in its interior \cite{DUMMIT}. The area A is given by A=I+2B​−1, where I is the number of interior lattice points and B is the number of boundary lattice points.Consider the triangle △ with vertices at the origin O(0,0), P1​(b,a), and P2​(d,c). The area of this triangle can be calculated using the determinant formula, which gives A=21​∣bc−ad∣. Since a/b<c/d, we have bc>ad, so the area is A=21​(bc−ad).Now, we apply Pick's Theorem to this triangle:\begin{itemize}\item \textbf{Boundary Points (B):} The vertices O, P1​, and P2​ are lattice points. Since the fractions a/b and c/d are irreducible, gcdf​unc(a,b)=1 and gcdf​unc(c,d)=1. This implies there are no other lattice points on the segments OP1​ and OP2​. If there were a lattice point on the segment P1​P2​, it would represent a rational fraction that should lie between a/b and c/d in the Farey sequence but with a smaller denominator, which is impossible for neighbors. Thus, the only lattice points on the boundary are the three vertices, so B=3 \cite{DUMMIT}.\item \textbf{Interior Points (I):} Suppose there were a lattice point (x,y) in the interior of △. The fraction y/x would have a value strictly between a/b and c/d. Since b≤n and d≤n, it would follow that x<n. This would mean that y/x is a fraction with a denominator smaller than n that lies between a/b and c/d, contradicting the assumption that they are consecutive terms in Fn​. Therefore, there can be no lattice points in the interior of the triangle, and I=0 \cite{DUMMIT}.\end{itemize}Applying Pick's Theorem with I=0 and B=3, the area of the triangle is A=0+23​−1=21​. Equating this with our determinant formula, we get 21​(bc−ad)=21​, which directly implies bc−ad=1. This provides a purely combinatorial confirmation that the mediant always yields an irreducible fraction when the unimodular condition is met.\subsection{The Isomorphism (P(Ωn​),Δ)≅F2n​}The powerset P(Ωn​) of a set Ωn​={1,2,…,n}, when equipped with the operation of symmetric difference (Δ), forms a finite abelian group \cite{WikipediaPowerset, SymmetricDifferenceGroup}.\begin{itemize}\item \textbf{Closure:} For any two subsets A,B⊆Ωn​, their symmetric difference AΔB=(A∪B)∖(A∩B) is also a subset of Ωn​.\item \textbf{Associativity:} The operation is associative: (AΔB)ΔC=AΔ(BΔC).\item \textbf{Identity Element:} The empty set ∅ serves as the identity element, as AΔ∅=A.\item \textbf{Inverse Element:} Every element is its own inverse, as AΔA=∅.\end{itemize}This group is isomorphic to the n-dimensional vector space over the field of two elements, F2​={0,1}, denoted F2n​ \cite{F2nVectorSpace, BooleanRing}. The isomorphism is established by mapping each subset S⊆Ωn​ to its characteristic function (or binary vector) of length n. The symmetric difference of two subsets corresponds precisely to the component-wise addition (XOR operation) of their corresponding vectors in F2n​. This mapping makes the group operation identical to vector addition modulo 2, clarifying the isomorphism’s computational meaning. The set of singleton subsets {{1},{2},…,{n}} forms a basis for this vector space \cite{F2nVectorSpace}.\begin{thebibliography}{99}\bibitem{WikipediaFoundations}Wikipedia contributors. (2024). \emph{Foundations of mathematics}. Wikipedia, The Free Encyclopedia.\bibitem{ElKhettabi2025AoO}El Khettabi, F. (2025). \emph{The Arithmetic of Order: A Finitistic Foundation for Mathematics, Emergent Structures, and Intelligent Systems}. viXra.\bibitem{ElKhettabi2024HCN}El Khettabi, F. (2024). \emph{A Comprehensive Modern Mathematical Foundation for Hypercomplex Numbers with Recollection of Sir William Rowan Hamilton, John T. Graves, and Arthur Cayley}.\bibitem{SEP_Constructivism}Iemhoff, R. (2023). \emph{Constructive Mathematics}. The Stanford Encyclopedia of Philosophy (Fall 2023 Edition), Edward N. Zalta & Uri Nodelman (eds.).\bibitem{WikipediaFarey}Wikipedia contributors. (2024). \emph{Farey sequence}. Wikipedia, The Free Encyclopedia.\bibitem{SEP_Finitism}Ye, F. (2021). \emph{Finitism in Geometry}. The Stanford Encyclopedia of Philosophy (Winter 2021 Edition), Edward N. Zalta (ed.).\bibitem{Zukin2016}Zukin, M. (2016). \emph{The Farey Sequence}. Whitman College.\bibitem{DUMMIT}Dummit, D. S., & Foote, R. M. (2004). \emph{Abstract Algebra}. John Wiley & Sons.\bibitem{ElKhettabi2024PLOS}El Khettabi, F. (2024). \emph{On the hypercomplex numbers and normed division algebra of all dimensions: A unified multiplication}. PLOS ONE, 19(6), e0312502.\bibitem{WikipediaPowerset}Wikipedia contributors. (2024). \emph{Power set}. Wikipedia, The Free Encyclopedia.\bibitem{ElKhettabi2025AoO_vixra}El Khettabi, F. (2025). The Arithmetic of Order: A Finitistic Foundation for Mathematics, Emergent Structures, and Intelligent Systems. \emph{viXra:2505.0064}.\bibitem{JNSFarey}Tamang, B. B., et al. (2022). Some characteristics of the Farey sequences with Ford circles. \emph{Nepal Journal of Mathematical Sciences}, 4(1), 69-76.\bibitem{RanickiFareyProject}Ranicki, A. (n.d.). \emph{The Farey sequence and its applications}.\bibitem{KnottFarey}Knott, R. \emph{Farey Series and the Stern-Brocot Tree}. University of Surrey.\bibitem{NumberanalyticsFarey}Number Analytics. (n.d.). \emph{Farey Sequences: A Deep Dive into Additive Number Theory}.\bibitem{WikipediaSternBrocot}Wikipedia contributors. (2024). \emph{Stern–Brocot tree}. Wikipedia, The Free Encyclopedia.\bibitem{CPAlgorithmsSternBrocot}CP-Algorithms. \emph{Stern-Brocot Tree and Farey Sequences}.\bibitem{CutTheKnotFarey}Bogomolny, A. \emph{Farey Series}. Cut-the-Knot.\bibitem{CutTheKnotSternBrocot}Bogomolny, A. \emph{Stern-Brocot Tree}. Cut-the-Knot.\bibitem{ConradSL2Z}Conrad, K. \emph{SL(2,Z)}. University of Connecticut.\bibitem{WikipediaSL2Z}Wikipedia contributors. (2024). \emph{Modular group}. Wikipedia, The Free Encyclopedia.\bibitem{FareyGraphSymmetry}Lutsko, C. (2021). \emph{Generalized Farey sequences}. International Mathematics Research Notices.\bibitem{BooleanRing}Wikipedia contributors. (2024). \emph{Boolean ring}. Wikipedia, The Free Encyclopedia.\bibitem{F2nVectorSpace}Stack Exchange. (2018). \emph{P(X) with symmetric difference as addition as a vector space over Z2}.\bibitem{SymmetricDifferenceGroup}ProofWiki. \emph{Symmetric Difference on Power Set forms Abelian Group}.\bibitem{HardyWright}Hardy, G. H., & Wright, E. M. (1979). \emph{An Introduction to the Theory of Numbers}. Oxford University Press.\bibitem{ConwaySloane}Conway, J. H., & Sloane, N. J. A. (1999). \emph{Sphere Packings, Lattices, and Groups}. Springer.\end{thebibliography}\end{document}















\documentclass[12pt,a4paper]{article}

% PACKAGES
\usepackage[margin=1in]{geometry}
\usepackage{amsmath, amssymb, amsthm}
\usepackage{authblk}
\usepackage[utf8]{inputenc}
\usepackage{fontenc}
\usepackage{hyperref}
\hypersetup{
    colorlinks=true,
    linkcolor=blue,
    filecolor=magenta,      
    urlcolor=cyan,
    pdftitle={The Farey Sequence as a Realization of the Arithmetic of Order},
    pdfpagemode=FullScreen,
}

% MATH OPERATORS AND ENVIRONMENTS
\DeclareMathOperator{\im}{Im}
\DeclareMathOperator{\re}{Re}
\DeclareMathOperator{\SL}{SL}
\DeclareMathOperator{\PSL}{PSL}
\DeclareMathOperator{\GL}{GL}
\DeclareMathOperator{\gcd_func}{gcd}

\newtheorem{theorem}{Theorem}[section]
\newtheorem{proposition}[theorem]{Proposition}
\newtheorem{lemma}[theorem]{Lemma}
\newtheorem{corollary}[theorem]{Corollary}
\theoremstyle{definition}
\newtheorem{definition}[theorem]{Definition}
\newtheorem{example}[theorem]{Example}
\theoremstyle{remark}
\newtheorem{remark}[theorem]{Remark}

% TITLE AND AUTHOR
\title{Physics as Quantized Measurement: The Farey Sequence as a Realization of the Arithmetic of Order}
\author{Faysal El Khettabi}
\affil{Ensemble AIs\\ \texttt{faysal.el.khettabi@gmail.com}}
\date{July 2025}

\begin{document}

\maketitle
\tableofcontents

\begin{abstract}
This article formalizes the Farey sequence hierarchy ($F_n$) and the universal Stern-Brocot tree ($T_n$) as not merely analogies but explicit constructive realizations of the ``Arithmetic of Order'' (AoO) framework. It synthesizes the generative principles of constraint-guided differentiation, the progression $1 \to n \to n+1$, and the nested powerset structure, demonstrating that the Farey system embodies the finitistic, emergent continuum perspective central to the AoO thesis. We propose that the Farey numbers, generated by a finite, deterministic process, provide a natural model for physical quantities and measurements, replacing the need for an *a priori* infinite continuum. This synthesis reinforces a unified finitistic approach to foundational mathematics, hypercomplex number theory, projective geometries, and the algorithmic design of intelligent systems.
\end{abstract}

\section{Introduction: A Finitistic Recasting of the Continuum}

The core argument of this report is that the Farey system provides a complete and rigorous mathematical prototype for the "Arithmetic of Order" framework \cite{ElKhettabi2025AoO}. This is not a relationship of analogy, but of instantiation, where the abstract principles of the latter find their direct, operational expression in the former.

The foundations of modern mathematics and physics have been shaped by a foundational crisis that emerged in the late 19th and early 20th centuries \cite{WikipediaFoundations}. A central feature of the resolution to this crisis was the widespread adoption of infinitary concepts, most notably the continuum of real and complex numbers. The imaginary unit $i = \sqrt{-1} \in \mathbb{C}$, for instance, is now central to the formulation of quantum theory \cite{ElKhettabi2024HCN, ElKhettabi2025AoO}. Yet, this reliance introduces a profound conceptual paradox: how can a finite physical system—a quantum register, a molecule, or any object composed of a finite number of components—fundamentally require an infinite mathematical construct for its description? \cite{ElKhettabi2025AoO}.

The "Arithmetic of Order" (AoO) framework directly confronts this paradox by proposing a mathematics built from the finite and observable, where complexity and structure emerge constructively \cite{ElKhettabi2025AoO}. This approach aligns with the philosophical traditions of finitism and constructivism \cite{SEP_Finitism, SEP_Constructivism}. Finitism, in its various forms, questions or rejects the existence of actual infinite objects, such as the set of all natural numbers, proposing that mathematics should be grounded in objects that are, at least in principle, finite \cite{WikipediaFarey, SEP_Finitism}. Constructivism insists that mathematical existence is tied to algorithmic constructibility; to prove an object exists, one must provide a method of finding (“constructing”) such an object \cite{SEP_Constructivism, Zukin2016}. The AoO framework synthesizes these views by critiquing the traditional reliance on *a priori* infinities and proposing a new foundation where the continuum itself is not a given axiom but an emergent property \cite{ElKhettabi2025AoO}.

This report demonstrates that the Farey sequence system offers a perfect, non-trivial model for this finitistic philosophy. The Farey sequence, generated by a simple, finite, and deterministic algorithm, provides a concrete pathway to the rational numbers and, by extension, to the real continuum. Its generation does not presuppose the existence of the continuum but rather builds it step-by-step, deriving it as an asymptotic limit of a sequence of finite structures \cite{WikipediaFarey, Zukin2016}. This aligns perfectly with the AoO's reinterpretation of the continuum as a limit of a nested hierarchy of finite sets \cite{ElKhettabi2025AoO}.

The structure of this analysis will proceed as follows. First, the abstract principles of the AoO framework will be detailed. Second, a rigorous exposition of the Farey system will be provided. The central sections will then establish the formal mapping between the generative processes and hierarchical structures of these two domains. This is followed by a deep analysis of their corresponding algebraic structures, a discussion of the hierarchical perception of the continuum, and finally, an exploration of the implications of this synthesis.

\section{The "Arithmetic of Order": A Framework of Emergent Complexity}

The "Arithmetic of Order" (AoO) is a foundational framework that seeks to re-establish mathematics on finite, constructive principles. It posits that the intricate structures of mathematics are not arbitrary inventions but are natural consequences of the most fundamental process of ordered progression \cite{ElKhettabi2025AoO}.

\subsection{The Generative Progression $1 \to n \to n+1$}
The central axiom of the AoO framework is that all of mathematics can be understood as a "natural revelation of the intrinsic structure embedded in the progression $1 \to n \to n+1$" \cite{ElKhettabi2025AoO}. This progression is not merely a representation of counting; it embodies the fundamental process of incrementally adding a new degree of freedom to a system \cite{ElKhettabi2025AoO, ElKhettabi2024HCN}. In physical terms, each step $n \to n+1$ can be interpreted as the introduction of a new quantized degree of freedom — an additional state, bit, or mode that refines the system’s resolution of measurable quantities. For example, each new fraction in the Farey hierarchy refines an interval with finite rational steps, mirroring how physical measurement adds finite resolution at each scale. This reflects the same principle that underlies Planck’s constant in quantum mechanics: physical observables do not vary continuously in theory but are constrained by finite quanta of action and resolution. In the context of physics, a system's state is defined by a finite number of such degrees of freedom, and understanding how the system's properties change as this number increases is paramount \cite{ElKhettabi2024HCN}.

The framework emphasizes that it is the *ordered* nature of the underlying set $\Omega_n = \{1, 2, \ldots, n\}$ that is crucial \cite{ElKhettabi2025AoO}. This ordering ensures that the hierarchy of structures built upon it is well-defined, recursive, and nested. Each step from $n$ to $n+1$ represents a deterministic expansion of the system's potential, opening up new combinatorial possibilities and enabling the emergence of higher-order symmetries and more complex structures \cite{ElKhettabi2025AoO}.

\subsection{The Powerset Hierarchy $\mathcal{P}(\Omega_n)$ as the Canonical State Space}
Within the AoO, the canonical state space of a system with $n$ degrees of freedom is identified with the powerset of $\Omega_n$, denoted $\mathcal{P}(\Omega_n)$ \cite{ElKhettabi2025AoO}. The powerset is the set of all subsets of $\Omega_n$, including the empty set and $\Omega_n$ itself \cite{WikipediaPowerset}. Each element of $\mathcal{P}(\Omega_n)$—that is, each subset $S \subseteq \Omega_n$—represents a distinct configuration of the system. This configuration can be encoded by a characteristic function, a binary vector indicating which degrees of freedom are "active" or "present" in that state \cite{ElKhettabi2025AoO, WikipediaPowerset}.

The generative progression $1 \to n \to n+1$ manifests directly as a nested hierarchy of these powerset state spaces: $\mathcal{P}(\Omega_n) \subset \mathcal{P}(\Omega_{n+1})$ \cite{ElKhettabi2024HCN}. The construction of the next level of the hierarchy is both recursive and fully deterministic. Given $\mathcal{P}(\Omega_n)$, the powerset $\mathcal{P}(\Omega_{n+1})$ is formed by taking all existing subsets in $\mathcal{P}(\Omega_n)$ and adding to them a new collection of subsets, each formed by taking an element of $\mathcal{P}(\Omega_n)$ and adjoining the new element $\{n+1\}$ \cite{WikipediaPowerset}. Formally, this is expressed as:
\
The cardinality of the powerset, $|\mathcal{P}(\Omega_n)| = 2^n$, grows exponentially, highlighting the rapid increase in structural complexity that arises from the addition of each new degree of freedom \cite{ElKhettabi2024HCN}.

\subsection{The Continuum as an Asymptotic Limit}
A cornerstone of the AoO framework is its explicit rejection of the continuum as an *a priori* entity \cite{ElKhettabi2025AoO}. Instead, the real continuum is reinterpreted as an asymptotic limit of the nested powerset hierarchy. As the number of degrees of freedom $n$ tends towards infinity, the combinatorial and topological properties of the finite space $\mathcal{P}(\Omega_n)$ approach those traditionally ascribed to continuous systems \cite{ElKhettabi2025AoO}.

This perspective aligns with finitist philosophies that are skeptical of actual, completed infinities but are comfortable with the concept of potential infinity or unbounded processes \cite{WikipediaFarey, SEP_Finitism}. The framework provides a concrete model for how this limit is approached. The cardinality of the powerset, $2^n$, naturally approaches the cardinality of the continuum, $2^{\aleph_0}$, as $n$ approaches the first infinite cardinal, $\aleph_0$. The Cantor space, which is homeomorphic to the set of infinite binary sequences $\{0, 1\}^\mathbb{N}$, serves as a topological bridge, as it is in one-to-one correspondence with $\mathcal{P}(\mathbb{N})$ and has the cardinality of the continuum \cite{WikipediaPowerset}.

\subsection{Constraint-Guided Differentiation and Emergent Structures}
A key generative mechanism within the AoO framework is termed "constraint-guided differentiation" \cite{ElKhettabi2025AoO}. This principle posits that by applying specific combinatorial rules or constraints to the powerset hierarchy, certain "optimal" mathematical structures can emerge deterministically. Examples cited within the framework's literature include the emergence of exceptional structures like the Golay code $G_{24}$ and the Leech lattice $\Lambda_{24}$ from $\mathcal{P}(\Omega_{24})$ when specific constraints are applied \cite{ElKhettabi2025AoO}.

\section{The Farey System: A Deterministic Path to the Rationals}

The Farey sequence and its related structures, such as the Stern-Brocot tree, form a cornerstone of elementary number theory, providing a constructive method for generating the set of rational numbers.

\subsection{The Farey Sequence $F_n$: Definition and Hierarchical Construction}
The Farey sequence of order $n$, denoted $F_n$, is formally defined as the sequence of irreducible fractions $a/b$ in the closed interval $(0,1)$ such that the denominator $b$ is a positive integer satisfying $1 \le b \le n$, arranged in ascending order of magnitude \cite{WikipediaFarey, Zukin2016}. Each sequence conventionally starts with $0/1$ and ends with $1/1$ \cite{WikipediaFarey}.

The first few Farey sequences are \cite{Zukin2016, RanickiFareyProject}:
\begin{itemize}
    \item $F_1: \{\frac{0}{1}, \frac{1}{1}\}$
    \item $F_2: \{\frac{0}{1}, \frac{1}{2}, \frac{1}{1}\}$
    \item $F_3: \{\frac{0}{1}, \frac{1}{3}, \frac{1}{2}, \frac{2}{3}, \frac{1}{1}\}$
    \item $F_4: \{\frac{0}{1}, \frac{1}{4}, \frac{1}{3}, \frac{1}{2}, \frac{2}{3}, \frac{3}{4}, \frac{1}{1}\}$
\end{itemize}
A fundamental property of this construction is its hierarchical nature: $F_n \subset F_{n+1}$ for all $n \ge 1$ \cite{WikipediaFarey, Zukin2016}. The number of new fractions added at step $n+1$ is precisely given by Euler's totient function, $\phi(n+1)$ \cite{WikipediaFarey, Zukin2016}.

\subsection{Mediant Refinement: The Algorithmic Engine}
The generation of Farey sequences is driven by the mediant operation \cite{RanickiFareyProject, NumberanalyticsFarey}. The mediant of two fractions $a/b$ and $c/d$ is defined as $(a+c)/(b+d)$. If $a/b < c/d$, their mediant always lies strictly between them \cite{KnottFarey, RanickiFareyProject, NumberanalyticsFarey}.

A fundamental theorem states that the sequence $F_{n+1}$ can be constructed from $F_n$ by identifying all adjacent pairs of fractions $a/b$ and $c/d$ in $F_n$ and inserting their mediant $(a+c)/(b+d)$ between them if and only if the denominator of the mediant satisfies $b+d = n+1$ \cite{DUMMIT, Zukin2016, JNSFarey}. This provides a fully algorithmic and constructive method for generating the entire hierarchy \cite{Zukin2016, JNSFarey}.

\subsection{The Stern-Brocot Tree: The Universal Genealogy of Rationals}
The mediant operation, when applied recursively without the denominator constraint, generates the Stern-Brocot tree, an infinite binary tree that contains every positive rational number exactly once \cite{WikipediaSternBrocot, CPAlgorithmsSternBrocot}. It is constructed by starting with the "ancestors" $0/1$ and $1/0$ (representing 0 and infinity) and iteratively inserting the mediant between adjacent fractions \cite{WikipediaSternBrocot, CutTheKnotSternBrocot}. The Farey sequence $F_n$ can be recovered by an in-order traversal of the tree, pruning any branch where a denominator exceeds $n$ \cite{WikipediaSternBrocot, CPAlgorithmsSternBrocot}.

\subsection{Foundational Properties and their Significance}
\begin{itemize}
    \item \textbf{The Unimodular Relation:} If two fractions $a/b$ and $c/d$ are adjacent in any Farey sequence $F_n$, they satisfy the unimodular relation: $bc - ad = 1$ \cite{DUMMIT, Zukin2016}. This invariant guarantees that any mediant formed from two neighbors is itself irreducible \cite{Zukin2016, JNSFarey}.
    \item \textbf{Optimality in Diophantine Approximation:} Farey sequences contain the set of "best rational approximations" of the first kind to any real number for a given denominator bound $n$ \cite{Zukin2016, JNSFarey}.
    \item \textbf{The Emergent Continuum:} The union of all Farey sequences, $\bigcup_{n=1}^{\infty} F_n$, constitutes the set of all rational numbers in $$\cite{CutTheKnotFarey}. The real continuum$$ is the topological closure of this constructively generated set \cite{Zukin2016}.
\end{itemize}

\section{The Isomorphism of Process: Unifying the Framework and its Realization}

The parallels between the AoO framework and the Farey system are not merely superficial. The core generative processes of both systems are structurally identical.

\subsection{Local Refinement as Constrained Expansion}
The fundamental mechanism of evolution in both systems is a process of local refinement governed by a global constraint.
\begin{itemize}
    \item In the **Farey system**, an interval $(a/b, c/d)$ in $F_n$ is refined by the mediant $(a+c)/(b+d)$ only when the denominator constraint $b+d = n+1$ is met \cite{Zukin2016, JNSFarey}.
    \item In the **powerset hierarchy**, the transition from $\mathcal{P}(\Omega_n)$ to $\mathcal{P}(\Omega_{n+1})$ introduces new subsets defined by the inclusion of the single new element $\{n+1\}$ \cite{WikipediaPowerset, ElKhettabi2024HCN}.
\end{itemize}
The mediant insertion rule is a perfect instantiation of the AoO's abstract principle of "constraint-guided differentiation" \cite{ElKhettabi2025AoO}. The parameter $n$ acts as a universal filter, determining which new elements are actualized at each stage.

\subsection{The Arrow of Complexity in Nested Hierarchies}
There is a formal mapping between the set inclusion $F_n \subset F_{n+1}$ and the powerset inclusion $\mathcal{P}(\Omega_n) \subset \mathcal{P}(\Omega_{n+1})$. Despite different growth rates (polynomial for $|F_n| \sim 3n^2/\pi^2$ vs. exponential for $|\mathcal{P}(\Omega_n)| = 2^n$), both systems exhibit an irreversible "arrow of complexity" driven by the same underlying progression, $1 \to n \to n+1$ \cite{ElKhettabi2025AoO, WikipediaFarey, Zukin2016}.

\section{A Unification of Algebraic Structures}
In algebraic thinking, the “rules” are not just about solving for unknowns, but about determining the space of all possible solutions and how they interrelate. The correspondence between the Farey and powerset systems extends to the very algebraic structures they embody, revealing a profound duality.

\begin{table}[h!]
\centering
\caption{Structural Parallels between Farey and Powerset Systems}
\label{tab:duality}
\begin{tabular}{|p{2.5cm}|p{4cm}|p{4cm}|p{3.5cm}|}
\hline
\textbf{Aspect} & \textbf{Farey/Stern-Brocot System} & \textbf{Powerset Hierarchy} & \textbf{Framework Link} \\
\hline
\textbf{Elements} & Irreducible fractions $p/q$ & Subsets $S \subseteq \Omega_n$ & Configurations of $n$ degrees of freedom \cite{ElKhettabi2025AoO, Zukin2016} \\
\hline
\textbf{Ordering} & Standard numerical order $<$ & Subset inclusion $\subset$ & Hierarchical nesting \cite{ElKhettabi2025AoO, Zukin2016} \\
\hline
\textbf{Refinement Rule} & Mediant $\frac{a+c}{b+d}$ if $b+d \leq n+1$ \cite{Zukin2016, JNSFarey} & Add subsets containing $n+1$ \cite{WikipediaPowerset} & Constraint-guided differentiation \cite{ElKhettabi2025AoO} \\
\hline
\textbf{Local Algebra} & Neighbors $a/b, c/d$ form a matrix in $\SL(2,\mathbb{Z})$ \cite{Zukin2016, DUMMIT} & --- & Structure of local interactions \\
\hline
\textbf{Global Algebra} & --- & $(\mathcal{P}(\Omega_n), \Delta)$ is isomorphic to $\mathbb{F}_2^n$ \cite{WikipediaPowerset} & Algebra of system states \cite{ElKhettabi2024HCN} \\
\hline
\textbf{Geometric View} & Ford Circles, Farey Tessellation of $\mathbb{H}^2$ \cite{WikipediaFarey, Zukin2016} & Projective planes/spaces over $\mathbb{F}_2$ \cite{ElKhettabi2025AoO} & Emergent geometry \cite{ElKhettabi2025AoO} \\
\hline
\textbf{Asymptotic Limit} & The real continuum $(0,1)$ \cite{Zukin2016} & The Cantor space $2^\mathbb{N}$, cardinality of continuum \cite{WikipediaPowerset} & The emergent continuum \cite{ElKhettabi2025AoO} \\
\hline
\end{tabular}
\end{table}

\subsection{The Modular Group $\SL(2,\mathbb{Z})$: The Invariant Algebra of Farey Adjacency}
The unimodular relation $bc - ad = 1$ is the defining characteristic of the special linear group $\SL(2,\mathbb{Z})$ \cite{ConradSL2Z, WikipediaSL2Z}. For any pair of adjacent Farey neighbors $a/b$ and $c/d$, the matrix $\begin{pmatrix} a & c \\ b & d \end{pmatrix}$ is an element of $\SL(2,\mathbb{Z})$ \cite{Zukin2016, DUMMIT}. This group acts as the group of orientation-preserving automorphisms of the Farey graph, which tessellates the hyperbolic plane \cite{ConradSL2Z, FareyGraphSymmetry}. $\SL(2,\mathbb{Z})$ is thus the infinite, non-abelian group that governs the *local, dynamic* structure of the Farey sequence \cite{ConradSL2Z}.

\subsection{The Vector Space $\mathbb{F}_2^n$: The Canonical Algebra of System Configurations}
The powerset $\mathcal{P}(\Omega_n)$, when equipped with the operation of symmetric difference ($\Delta$), forms a finite abelian group \cite{WikipediaPowerset, BooleanRing}. This group is isomorphic to the $n$-dimensional vector space over the finite field of two elements, $\mathbb{F}_2^n$ \cite{F2nVectorSpace, BooleanRing}. This algebraic structure is presented in the AoO framework as the canonical algebra for describing the *global, static* set of all possible configurations of a system with $n$ binary degrees of freedom \cite{ElKhettabi2025AoO, ElKhettabi2024HCN}.

\subsection{The Duality of Local (Non-Abelian) and Global (Abelian) Algebra}
The synthesis of these systems reveals a profound duality. The AoO framework, by asserting that the Farey system is a concrete instantiation of its principles, implicitly predicts this juxtaposition of the infinite, non-abelian group $\SL(2,\mathbb{Z})$ with the sequence of finite, abelian groups $\mathbb{F}_2^n$. This key observation can be summarized as follows: The Farey system’s local, non-abelian symmetry ($\SL(2,\mathbb{Z})$) governs the *dynamic refinement* of the space \cite{Zukin2016, ConradSL2Z}, while the powerset’s global, abelian symmetry ($\mathbb{F}_2^n$) governs the *static configuration* of all possible states \cite{WikipediaPowerset, BooleanRing}. This duality between dynamic refinement and static configuration is a central feature of the AoO framework's explanatory power.

\section{Hierarchical Perception of the Continuum through Degrees of Freedom}

The relationship between the sets of mediants generated at different stages of the Farey hierarchy captures how systems with different degrees of freedom "perceive" the continuum. This section formalizes this concept, demonstrating that each degree of freedom unlocks a distinct, non-overlapping "vision" of the emergent continuum.

\subsection{Defining the "Vision" of a Degree of Freedom: The Sets $W_n$ and $W_m$}
For a system with a given degree of freedom, represented by the integer $n$, we can define its specific interval of focus and the set of rational numbers it generates to refine that interval.

\begin{itemize}
    \item \textbf{Interval of Focus for degree $n$:} The interval $I_n = (\frac{1}{n+1}, \frac{1}{n})$ is defined by two fractions that become Farey neighbors in the sequence $F_{n+1}$ \cite{Zukin2016}.
    \item \textbf{Mediant Refinement Set $W_n$:} We define $W_n$ as the set of all rational numbers that recursively refine the interval $I_n$. This set is generated by the iterative application of the mediant operation, a process identical to the construction of a local Stern-Brocot tree within that interval \cite{Zukin2016, NumberanalyticsFarey}.
    \begin{itemize}
        \item \textbf{Properties of $W_n$:}
        \item $W_n$ is \textbf{dense} in $(\frac{1}{n+1}, \frac{1}{n})$, as the mediant operation eventually fills every gap between rational numbers \cite{DUMMIT}.
        \item $W_n$ is \textbf{infinite} and \textbf{ordered} according to the structure of the Stern-Brocot process \cite{WikipediaSternBrocot, CPAlgorithmsSternBrocot}.
    \end{itemize}
\end{itemize}

For a system with a higher degree of freedom $m > n$:
\begin{itemize}
    \item \textbf{Interval of Focus for degree $m$:} $I_m = (\frac{1}{m+1}, \frac{1}{m})$.
    \item \textbf{Mediant Refinement Set $W_m$:} $W_m$ is similarly defined as the set of all mediants that recursively refine the interval $I_m$.
\end{itemize}

\begin{theorem}
Let $I_n = \Big(\frac{1}{n+1}, \frac{1}{n}\Big)$ and define $W_n$ as the set of all rationals generated by repeated mediant refinement within $I_n$. Then for any $m > n+1$, the sets $W_n$ and $W_m$ are disjoint:
\[
W_n \cap W_m = \emptyset.
\]
Moreover, the rational continuum in $(0,1)$ satisfies
\[
\bigcup_{n=1}^{\infty} W_n = \mathbb{Q} \cap (0,1),
\quad \text{and} \quad
\overline{\bigcup_{n=1}^{\infty} W_n} = (0,1).
\]
\end{theorem}

\begin{remark}
The non-overlapping property of the $W_n$ sets formalizes the notion that a physical system with $n$ quantized degrees of freedom cannot access or resolve measurement bands defined by higher $m > n+1$. This mirrors how discrete energy levels, finite sampling, or Planck-scale limits constrain physical measurement in quantum theory.
\end{remark}

\subsection{Implications for the Arithmetic of Order Framework}
This model of hierarchical, disjoint perception has significant implications for the AoO framework.
\begin{enumerate}
    \item \textbf{Degree-of-Freedom-Dependent Reality:} The continuum is not perceived uniformly but as a **hierarchy of refinements**, where each "vision" is tied to the observer’s available degrees of freedom ($n$ or $m$) \cite{ElKhettabi2025AoO}. A physical system with $n$ variables cannot "see" the finer structure revealed by $m > n$ variables.
    \item \textbf{Finitistic Continuum:} The "complete" continuum is the limit of all $W_k$, but no system with a finite number of degrees of freedom can fully capture it. This aligns with the core philosophy of finitism \cite{ElKhettabi2025AoO, SEP_Finitism}.
    \item \textbf{Algebraic Consistency:} The disjointness of $W_n$ and $W_m$ is a concrete example of the AoO’s principle of **constraint-guided differentiation** \cite{ElKhettabi2025AoO}. Each degree of freedom $k$ introduces structure **only** in its designated interval $(\frac{1}{k+1}, \frac{1}{k})$.
\end{enumerate}

\section{Conclusion and Outlook}

The rigorous analysis presented here confirms that the Farey sequence hierarchy $F_n$ and the infinite Stern-Brocot tree provide a direct and operational instantiation of the core principles articulated in the \emph{Arithmetic of Order} framework. By demonstrating a formal isomorphism between the Farey system’s mediant-based local refinement and the AoO’s principle of \emph{constraint-guided differentiation}, this work extends the finitistic, constructivist agenda championed in El Khettabi’s foundational reports on powerset combinatorics, hypercomplex numbers, and emergent geometries \cite{ElKhettabi2025AoO, ElKhettabi2024HCN, ElKhettabi2024PLOS}.

Crucially, the Farey system shows that the real continuum --- historically assumed as an \emph{a priori} infinite structure --- can instead be understood as the asymptotic closure of a fully discrete, algorithmically generable process \cite{SEP_Finitism, Zukin2016}. Every rational in the unit interval is positioned within a nested, well-ordered hierarchy whose generation is entirely finitistic and transparent \cite{WikipediaFarey, Zukin2016}. The Farey mediant mechanism mirrors the XOR-bitwise operation on powersets: both systems employ a local, deterministic rule to refine an initial configuration space under a global constraint parameter, here the denominator bound $n$ \cite{WikipediaPowerset, JNSFarey}.

The duality between local non-abelian modular symmetries ($\SL(2,\mathbb{Z})$) and the global abelian powerset algebra ($\mathbb{F}_2^n$) reinforces the AoO’s insight that the richness of mathematical and physical structure can emerge from the simple arithmetic of ordered degrees of freedom \cite{ElKhettabi2025AoO, ConradSL2Z, BooleanRing}. Just as the Golay code $G_{24}$, the Leech lattice $\Lambda_{24}$, and the Mathieu group $M_{24}$ arise from sieving the powerset $\mathcal{P}(\Omega_{24})$, so too does the Farey hierarchy filter the Stern-Brocot tree into optimally ordered sets of best rational approximations \cite{ElKhettabi2025AoO, DUMMIT}.

By placing the Farey system within this finitistic combinatorial paradigm, this article not only strengthens the AoO’s theoretical claims but demonstrates its flexibility and universality across number theory, coding theory, and the modeling of continuum phenomena within strictly finite means. In this sense, the Farey hierarchy provides an explicit model of physics as quantized measurement: each mediant, each denominator constraint, each combinatorial power represents a discrete quantum of possible states. The real continuum, as the limit of these finite steps, is not a physical prerequisite but an emergent idealization of finitely resolved measurements.

\textbf{Future Directions.} This synthesis invites further exploration along several lines. First, the deep interplay between Farey adjacency, modular tessellations, and hyperbolic geometry (Ford circles, Farey tessellation of $\mathbb{H}^2$) deserves to be mapped explicitly onto the projective geometries naturally emerging from the ordered powerset hierarchy \cite{ElKhettabi2025AoO, WikipediaFarey, FareyGraphSymmetry}. Second, the constructive mediant process suggests algorithmic avenues for designing finite, resource-bounded AI systems capable of performing Diophantine approximations without recourse to continuum assumptions \cite{ElKhettabi2025AoO, WikipediaSternBrocot}. Finally, the formal analogy between Farey trees and the nested powerset combinatorics suggests potential generalizations to higher-dimensional hypercomplex structures and their associated finite geometries.

In unifying the Farey sequence and the Arithmetic of Order framework, we highlight a powerful theme: the apparent paradox of continuum structures in finite physical systems dissolves when seen as the limit of simple, local, finitistic rules. Mathematics, under this lens, is not an edifice built on the infinite, but a revelation of emergent order rooted in the finite --- one degree of freedom at a time.

\vspace{1em}

\noindent
\textbf{Faysal El Khettabi} \\
\emph{Ensemble AIs} \\
\texttt{faysal.el.khettabi@gmail.com} \\
July 2025

\appendix
\section{Appendix}

\subsection{Proof of the Unimodular Relation via Pick's Theorem}
A key property of Farey sequences is that if $a/b$ and $c/d$ are adjacent terms (neighbors), then $bc - ad = 1$. A geometric proof utilizes Pick's Theorem, which relates the area of a simple polygon whose vertices are points on the integer lattice to the number of integer points on its boundary and in its interior \cite{DUMMIT}. The area $A$ is given by $A = I + \frac{B}{2} - 1$, where $I$ is the number of interior lattice points and $B$ is the number of boundary lattice points.

Consider the triangle $\triangle$ with vertices at the origin $O(0,0)$, $P_1(b,a)$, and $P_2(d,c)$. The area of this triangle can be calculated using the determinant formula, which gives $A = \frac{1}{2}|bc - ad|$. Since $a/b < c/d$, we have $bc > ad$, so the area is $A = \frac{1}{2}(bc - ad)$.

Now, we apply Pick's Theorem to this triangle:
\begin{itemize}
    \item \textbf{Boundary Points (B):} The vertices $O$, $P_1$, and $P_2$ are lattice points. Since the fractions $a/b$ and $c/d$ are irreducible, $\gcd_func(a,b)=1$ and $\gcd_func(c,d)=1$. This implies there are no other lattice points on the segments $OP_1$ and $OP_2$. If there were a lattice point on the segment $P_1P_2$, it would represent a rational fraction that should lie between $a/b$ and $c/d$ in the Farey sequence but with a smaller denominator, which is impossible for neighbors. Thus, the only lattice points on the boundary are the three vertices, so $B=3$ \cite{DUMMIT}.
    \item \textbf{Interior Points (I):} Suppose there were a lattice point $(x,y)$ in the interior of $\triangle$. The fraction $y/x$ would have a value strictly between $a/b$ and $c/d$. Since $b \le n$ and $d \le n$, it would follow that $x < n$. This would mean that $y/x$ is a fraction with a denominator smaller than $n$ that lies between $a/b$ and $c/d$, contradicting the assumption that they are consecutive terms in $F_n$. Therefore, there can be no lattice points in the interior of the triangle, and $I=0$ \cite{DUMMIT}.
\end{itemize}
Applying Pick's Theorem with $I=0$ and $B=3$, the area of the triangle is $A = 0 + \frac{3}{2} - 1 = \frac{1}{2}$. Equating this with our determinant formula, we get $\frac{1}{2}(bc - ad) = \frac{1}{2}$, which directly implies $bc - ad = 1$. This provides a purely combinatorial confirmation that the mediant always yields an irreducible fraction when the unimodular condition is met.

\subsection{The Isomorphism $(\mathcal{P}(\Omega_n), \Delta) \cong \mathbb{F}_2^n$}
The powerset $\mathcal{P}(\Omega_n)$ of a set $\Omega_n = \{1, 2, \ldots, n\}$, when equipped with the operation of symmetric difference ($\Delta$), forms a finite abelian group \cite{WikipediaPowerset, SymmetricDifferenceGroup}.
\begin{itemize}
    \item \textbf{Closure:} For any two subsets $A, B \subseteq \Omega_n$, their symmetric difference $A \Delta B = (A \cup B) \setminus (A \cap B)$ is also a subset of $\Omega_n$.
    \item \textbf{Associativity:} The operation is associative: $(A \Delta B) \Delta C = A \Delta (B \Delta C)$.
    \item \textbf{Identity Element:} The empty set $\emptyset$ serves as the identity element, as $A \Delta \emptyset = A$.
    \item \textbf{Inverse Element:} Every element is its own inverse, as $A \Delta A = \emptyset$.
\end{itemize}
This group is isomorphic to the $n$-dimensional vector space over the field of two elements, $\mathbb{F}_2 = \{0, 1\}$, denoted $\mathbb{F}_2^n$ \cite{F2nVectorSpace, BooleanRing}. The isomorphism is established by mapping each subset $S \subseteq \Omega_n$ to its characteristic function (or binary vector) of length $n$. The symmetric difference of two subsets corresponds precisely to the component-wise addition (XOR operation) of their corresponding vectors in $\mathbb{F}_2^n$. This mapping makes the group operation identical to vector addition modulo 2, clarifying the isomorphism’s computational meaning. The set of singleton subsets $\{\{1\}, \{2\}, \ldots, \{n\}\}$ forms a basis for this vector space \cite{F2nVectorSpace}.

\begin{thebibliography}{99}

\bibitem{WikipediaFoundations}
Wikipedia contributors. (2024). \emph{Foundations of mathematics}. Wikipedia, The Free Encyclopedia.

\bibitem{ElKhettabi2025AoO}
El Khettabi, F. (2025). \emph{The Arithmetic of Order: A Finitistic Foundation for Mathematics, Emergent Structures, and Intelligent Systems}. viXra.

\bibitem{ElKhettabi2024HCN}
El Khettabi, F. (2024). \emph{A Comprehensive Modern Mathematical Foundation for Hypercomplex Numbers with Recollection of Sir William Rowan Hamilton, John T. Graves, and Arthur Cayley}.

\bibitem{SEP_Constructivism}
Iemhoff, R. (2023). \emph{Constructive Mathematics}. The Stanford Encyclopedia of Philosophy (Fall 2023 Edition), Edward N. Zalta \& Uri Nodelman (eds.).

\bibitem{WikipediaFarey}
Wikipedia contributors. (2024). \emph{Farey sequence}. Wikipedia, The Free Encyclopedia.

\bibitem{SEP_Finitism}
Ye, F. (2021). \emph{Finitism in Geometry}. The Stanford Encyclopedia of Philosophy (Winter 2021 Edition), Edward N. Zalta (ed.).

\bibitem{Zukin2016}
Zukin, M. (2016). \emph{The Farey Sequence}. Whitman College.

\bibitem{DUMMIT}
Dummit, D. S., \& Foote, R. M. (2004). \emph{Abstract Algebra}. John Wiley \& Sons.

\bibitem{ElKhettabi2024PLOS}
El Khettabi, F. (2024). \emph{On the hypercomplex numbers and normed division algebra of all dimensions: A unified multiplication}. PLOS ONE, 19(6), e0312502.

\bibitem{WikipediaPowerset}
Wikipedia contributors. (2024). \emph{Power set}. Wikipedia, The Free Encyclopedia.

\bibitem{ElKhettabi2025AoO_vixra}
El Khettabi, F. (2025). The Arithmetic of Order: A Finitistic Foundation for Mathematics, Emergent Structures, and Intelligent Systems. \emph{viXra:2505.0064}.

\bibitem{JNSFarey}
Tamang, B. B., et al. (2022). Some characteristics of the Farey sequences with Ford circles. \emph{Nepal Journal of Mathematical Sciences}, 4(1), 69-76.

\bibitem{RanickiFareyProject}
Ranicki, A. (n.d.). \emph{The Farey sequence and its applications}.

\bibitem{KnottFarey}
Knott, R. \emph{Farey Series and the Stern-Brocot Tree}. University of Surrey.

\bibitem{NumberanalyticsFarey}
Number Analytics. (n.d.). \emph{Farey Sequences: A Deep Dive into Additive Number Theory}.

\bibitem{WikipediaSternBrocot}
Wikipedia contributors. (2024). \emph{Stern–Brocot tree}. Wikipedia, The Free Encyclopedia.

\bibitem{CPAlgorithmsSternBrocot}
CP-Algorithms. \emph{Stern-Brocot Tree and Farey Sequences}.

\bibitem{CutTheKnotFarey}
Bogomolny, A. \emph{Farey Series}. Cut-the-Knot.

\bibitem{CutTheKnotSternBrocot}
Bogomolny, A. \emph{Stern-Brocot Tree}. Cut-the-Knot.

\bibitem{ConradSL2Z}
Conrad, K. \emph{SL(2,Z)}. University of Connecticut.

\bibitem{WikipediaSL2Z}
Wikipedia contributors. (2024). \emph{Modular group}. Wikipedia, The Free Encyclopedia.

\bibitem{FareyGraphSymmetry}
Lutsko, C. (2021). \emph{Generalized Farey sequences}. International Mathematics Research Notices.

\bibitem{BooleanRing}
Wikipedia contributors. (2024). \emph{Boolean ring}. Wikipedia, The Free Encyclopedia.

\bibitem{F2nVectorSpace}
Stack Exchange. (2018). \emph{P(X) with symmetric difference as addition as a vector space over Z2}.

\bibitem{SymmetricDifferenceGroup}
ProofWiki. \emph{Symmetric Difference on Power Set forms Abelian Group}.

\bibitem{HardyWright}
Hardy, G. H., \& Wright, E. M. (1979). \emph{An Introduction to the Theory of Numbers}. Oxford University Press.

\bibitem{ConwaySloane}
Conway, J. H., \& Sloane, N. J. A. (1999). \emph{Sphere Packings, Lattices, and Groups}. Springer.

\end{thebibliography}

\end{document}