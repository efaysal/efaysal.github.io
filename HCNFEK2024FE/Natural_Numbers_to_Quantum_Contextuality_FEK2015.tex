
\documentclass[11pt]{article}
\usepackage{amsmath, amssymb, amsfonts}
\usepackage{geometry}
\usepackage{hyperref}
\usepackage{graphicx}
\usepackage{booktabs}
\usepackage{authblk}
\geometry{margin=1in}

\hypersetup{
    colorlinks=true,
    linkcolor=blue,
    filecolor=magenta,      
    urlcolor=cyan,
    pdftitle={From Natural Numbers to Quantum Contextuality},
    pdfauthor={Faysal El Khettabi}
}

\title{From Natural Numbers to Quantum Contextuality:\\ The Revelations of Modularity in Finite Projective Geometries}
\author{Faysal El Khettabi}
\affil{\texttt{faysal.el.khettabi@gmail.com}}
\affil{LinkedIn: \href{https://www.linkedin.com/in/faysal-el-khettabi-ph-d-4847415}{faysal-el-khettabi-ph-d-4847415}}
\date{The Beauty of Expanding Knowledge}

\begin{document}

\maketitle

\begin{abstract}
This article explores how the arithmetic of natural numbers, when viewed through the lens of modularity, gives rise to the combinatorial and geometric structures underlying quantum contextuality and computational power. We trace the logical journey from simple modular arithmetic to the rich modular distinctions found in finite projective geometries, particularly the Fano plane, and show how these structures directly inform the distinction between stabilizer and magic states in quantum information theory.
\end{abstract}

\section{Introduction}

Mathematics, at its deepest, is not merely a formal game of symbols. It is a construction-revelation arising from the fundamental logic of the natural numbers: from \textbf{1 to n to n+1}. Arithmetic, when observed carefully, demands structures like modularity, combinatorics, and ultimately, geometry. In this note, we reveal how the modular structures found in quantum information theory emerge naturally from the combinatorial properties of finite projective geometries, particularly the Fano plane $\mathrm{PG}(2,2)$.

\section{The Arithmetic Foundation: 1 to 7}

Starting with the set $\{1,2,3,4,5,6,7\}$, we label the points of the smallest projective plane $\mathrm{PG}(2,2)$. In this labeling, an extraordinary property emerges: by considering lines as triples $(a,b,c)$ where the sum $a+b$ (mod 7) equals $c$ or not, we naturally divide the lines into two types:
\begin{itemize}
    \item \textbf{Ordinary lines}: satisfying $a+b=c \pmod{7}$
    \item \textbf{Defective lines}: not satisfying this relation
\end{itemize}
Thus, even in simple labeling, a \textbf{hidden modularity} emerges, constructed from the arithmetic of natural numbers.

\section{Emergence of Modular Classifications}

By further analyzing the structure:

\begin{center}
\begin{tabular}{cccc}
\toprule
Object in PG(2,2) & Classification & Modular Relation & Count \\
\midrule
Lines & Ordinary vs Defective & mod 2 & 2 types \\
Points & By number of defective lines & mod 4 & 4 types \\
Planes & Full labeling types & mod 8 & 8 types \\
\bottomrule
\end{tabular}
\end{center}

A hierarchy emerges:
\[
2^1,\, 2^2,\, 2^3
\]
Each level of structure reflects the powerset behavior of natural numbers and connects directly to modular arithmetic.

\section{Hidden Modularity and Lifting to Rings}

In the finite field $\mathbb{F}_2$ underlying PG(2,2), the characteristic 2 ensures that bilinear forms satisfy $B(x,x) = 0$. However, when lifting to modular rings like $\mathbb{Z}_4$, the characteristic 2 is no longer zero mod 4.

For any quadratic form $Q(x)$:
\begin{itemize}
    \item $2Q(x) \equiv 0 \pmod{4}$ if $Q(x) \equiv 0 \pmod{2}$
    \item $2Q(x) \equiv 2 \pmod{4}$ if $Q(x) \equiv 1 \pmod{2}$
\end{itemize}
Thus, only two classes for $B(x,x) = 2Q(x) \pmod{4}$ exist: 0 and 2.

\section{Quantum Information Applications}

This modular behavior mirrors quantum structures:
\begin{itemize}
    \item \textbf{Stabilizer states}: $2Q(x) = 0 \pmod{4}$
    \item \textbf{Magic states}: $2Q(x) = 2 \pmod{4}$
\end{itemize}

Quantum contextuality arises precisely because certain configurations (defective lines) resist classical explanation, based on these modular divisions.

Quantum circuits, particularly phase gates like the T-gate, cycle phases in $e^{i\pi/4}$ increments, reflecting a direct mod 4 modularity.

Thus, modular structures first revealed in finite geometries underpin quantum computational power.

\section{Natural Numbers, Modular Arithmetic, and the Construction of Quantum Reality}

\subsection{Natural Numbers and Modular Construction}

The arithmetic of natural numbers, following the sequence $1 \to 2 \to \cdots \to n \to n+1$, is the seedbed of all mathematical structures. Modular conditions, especially mod 2, mod 4, and mod 8, reveal emergent modular patterns constructed by deep combinatorial necessity.

\subsection{Finite Projective Geometries and Modular Classifications}

The Fano plane (PG(2,2)), when viewed through modular arithmetic, encodes the hidden modular order of combinatorial interactions:

\begin{center}
\begin{tabular}{ccc}
\toprule
Structure & Modular Level & Number of Types \\
\midrule
Lines & mod 2 & 2 types \\
Points & mod 4 & 4 types \\
Planes & mod 8 & 8 types \\
\bottomrule
\end{tabular}
\end{center}

Thus, modularity emerges organically from the powerset behavior of sets.

\subsection{Fields, Rings, and the Breaking of Formalism}

Fields like $\mathbb{F}_2$ preserve strict characteristics (e.g., $2=0$), where bilinear forms $B(x,x)$ always vanish. To capture richer modular distinctions, we must lift to rings like $\mathbb{Z}_4$ or $\mathbb{Z}_2 \times \mathbb{Z}_2$, where:
\begin{itemize}
    \item $2 \neq 0 \pmod{4}$,
    \item $2Q(x) \pmod{4}$ can be $0$ or $2$,
    \item Stabilizer vs magic distinction becomes visible.
\end{itemize}
Thus, the emergence of rings is a mathematical and physical necessity dictated by the combinatorics.

\subsection{Quantum Information and Modular Arithmetic}

These modular structures mirror fundamental structures in quantum information:
\begin{itemize}
    \item \textbf{Stabilizer states}: $2Q(x) = 0 \pmod{4}$ (classical-like)
    \item \textbf{Magic states}: $2Q(x) = 2 \pmod{4}$ (quantum contextual)
\end{itemize}

Quantum circuits, through phase gates like the T-gate, naturally cycle through mod 4 phases (increments of $\pi/4$).

Thus, modular arithmetic governs the operation of quantum computational models.

\subsection{Conclusion: Modular Arithmetic as the Bridge from Natural Numbers to Quantum Reality}

\begin{quote}
Modularity is not an external formalism. It is the natural language of arithmetic reality, finite geometries, and quantum structure.
\end{quote}

The path:
\[
\boxed{\text{Natural Numbers}} \longrightarrow \boxed{\text{Modular Arithmetic}} \longrightarrow \boxed{\text{Finite Geometries}} \longrightarrow \boxed{\text{Quantum Reality}}
\]
is not speculative; it is constructed, revealed, and verified by combinatorial and physical necessity.

\section*{References}
\begin{enumerate}
    \item Planat, M., "Quantum contextual finite geometries from dessins d'enfants", arXiv:1310.4267
    \item Saniga, M., "A Class of Three-Qubit Contextual Configurations Located in Fano Pentads", arXiv:2004.07517
    \item Planat, M., Saniga, M., "Quantum contextual finite geometries from dessins d'enfants", World Scientific (2015)
    \item Sengupta, A.N., "Finite Geometries with Qubit Operators"
    \item Planat, M., "Quantum computation and measurements from an exotic space-time R4", arXiv:2001.09091
    \item Exploiting Finite Geometries for Better Quantum Advantages, arXiv:2403.09512
    \item Wikipedia: Fano plane
    \item Error Correction Zoo: Modular-qudit code, Clifford hierarchy
    \item Nature npj Quantum Information: Randomised benchmarking of non-Clifford gates
\end{enumerate}

\section*{LinkedIn/ResearchGate Sharing Intro}
\begin{quote}
How does the arithmetic of natural numbers give rise to quantum contextuality and computational power? This article reveals the deep modular structures connecting counting, geometry, and the quantum world---showing how the Fano plane and modular arithmetic underpin the logic of quantum computation and contextuality. Dive in for a journey from numbers to quantum reality!
\end{quote}

\end{document}
